\begin{lemma} \label{lem:spiess-numbers}
Sei $K = (F, Z)$ eine Spießkombination. Für jede Obstsorte $o(x, i)$, wobei $x \in F$, gilt:
\begin{enumerate}[label={\upshape(\roman*)}]
  %\item $\forall x \in F\, \forall y \in Z: x \leadsto y$
  %\item $\nexists p \in F\, \forall q \in B \setminus Z: p \leadsto q$.
  \item $i \in Z$, \label{lem:spiess-numbers1}
  \item $i \notin B \setminus Z$. \label{lem:spiess-numbers2}
\end{enumerate}   
Deshalb darf man alle Kanten, die aus einem Knoten $x \in F$ 
zu einem Knoten $y \in B \setminus Z$, sowie die aus einem Knoten $p \in Z$ zu
einem Knoten $q \in A \setminus F$ führen, aus $E$ entfernen.
\end{lemma}

\begin{proof}
Nach Definition \ref{def:spiesskomb} gilt \ref{lem:spiess-numbers1}. 
Nach Axiom \ref{ax:obstsorte-index} besitzt jede Obstsorte einen einzigartigen Index $i$,
deshalb kann $i$ nicht gleichzeitig zu $Z$ und $B \setminus Z$ gehören \ref{lem:spiess-numbers2}.\\
Die Folgerung gilt, da %der Graph $G$ am Anfang vollständig ist und
jede Kante zwischen zwei beliebigen Knoten $x \in A$ und $y \in B$ die Möglichkeit darstellt,
dass $x$ einen Index $y$ besitzen kann.
Wenn eine Teilmenge von $A$ und $B$ in Form einer Spießkombination ausgegliedert wird,
schrumpft die Anzahl an möglichen Zuweisungen zwischen jedem $x$ und jedem $y$. 
\end{proof}
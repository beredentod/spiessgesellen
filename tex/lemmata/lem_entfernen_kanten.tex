\begin{lemma} \label{lem:spiess-numbers}
Sei $K = (F, Z)$ eine Spießkombination. Für jede Obstsorte $o(x, i)$, wobei $x \in F$, gilt:
\begin{enumerate}[label={\upshape(\roman*)}]
  %\item $\forall x \in F\, \forall y \in Z: x \leadsto y$
  %\item $\nexists p \in F\, \forall q \in B \setminus Z: p \leadsto q$.
  \item $i \in Z$. \label{lem:spiess-numbers1}
  \item $i \notin B \setminus Z$. \label{lem:spiess-numbers2}
\end{enumerate}   
Deshalb darf man alle Kanten, die aus jedem Knoten $x \in F$ 
zu jedem Knoten $y \in B \setminus Z$ führen, aus $E$ entfernen und nur die Kanten lassen, die
zu allen $z \in Z$ führen.
\end{lemma}

\begin{proof}
Nach Definition \ref{def:spiesskomb} gilt \ref{lem:spiess-numbers1}. 
Nach Axiom \ref{ax:obstsorte-index} besitzt jede Obstsorte einen einzigartigen Index $i$,
deshalb kann $i$ nicht gleichzeitig zu $Z$ und $B \setminus Z$ gehören \ref{lem:spiess-numbers2}.
\end{proof}
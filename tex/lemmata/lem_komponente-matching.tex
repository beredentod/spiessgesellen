\begin{lemma}\label{lem:komponente-matching}
Sei $C = (L_c \cup R_c, E_c)$ eine beliebige
Zusammenhangskomponente in $G$. Dann existiert immer ein perfektes Matching zu $C$.
\end{lemma}
\begin{proof} 
Nach Lemma \ref{lem:komponente-complete} ist jede Zusammenhangskomponente in $G$ ein vollständiger,
bipartiter Graph. Nach Satz von Hall existiert ein perfektes Matching, wenn
für alle Teilmengen $K \subseteq L_c$ gilt: $|K| \leqslant |N(K)|$.

Sei $K$ eine beliebige Teilmenge von $L_c$ mit der Mächtigkeit $|K| \leqslant |L_c|$. 
Da $C$ selbst ein vollständiger, bipartiter Graph ist,
besitzt jeder Knoten in $C$ die Kardinalität von $|L_c| = |R_c|$.
So gilt: $N(K) = |K| \cdot |L_c|$ und es gilt: $|K| \leqslant |N(K)|$.
\end{proof}


\begin{comment}
Die obere Behauptung kann für beliebig große Mächtigkeiten $|K| = k \in \mathbb{N}$
durch vollständige Induktion bewiesen werden.\\

\noindent
\tbf{Induktionsanfang:} Für $k = 1$ hat der einzelne Knoten $x \in K \subseteq L_c$
die Kardinalität $\Delta(x) = 1$.
Deshalb gilt: $|K| = 1 \leqslant |N(x)| = 1$. Damit stimmt die Behauptung für $k = 1$ und der Induktionsanfang ist erledigt.\\

\noindent
\tbf{Induktionsschritt:} Es gelte die Aussage für ein beliebiges $k \in \mathbb{N}$, also für eine Teilemenge
$K \subseteq L_c$, die aus $k$ Knoten besteht und in der jeder Knoten $x \in K$ die Kardinalität
$\Delta(x) = k$ hat.\\ Es gelte also: $|K| \leqslant |N(K)|$.\\
Zu zeigen ist die Aussage für $k + 1$, also für eine Teilmenge $K' \subseteq L_c$ der Mächtigkeit 
$|K'| = k+1$:
\[
|K'| \leqslant |N(K')|.
\] 
Wir verifizieren:

Jeder Knoten in $C$ hat den Grad $k+1$, also: $|K'| = k + 1 \leqslant |N(K')| = (k+1)^2 = k^2 + 2k + 1$.\\
Folglich stimmt die Behauptung für $k+1$.\\

Der Induktionsschritt ist damit vollzogen und es wurde bewiesen, dass die Behauptung für beliebige
Mächtigkeit von $K$ gilt.
Dadurch wurde auch bewiesen, dass es in einer Zusammenhangskomponente in $G$
 immer ein perfektes Matching gibt.
\end{comment}


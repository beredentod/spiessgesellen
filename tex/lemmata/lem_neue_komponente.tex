\begin{comment}
\begin{lemma}\label{lem:neue-komponenten}
Sei $C = (V_c, E_c) \subseteq G$ eine Zusammenhangskomponente.
Sei $K = (F, Z)$ eine Spießkombiantion.
Wir betrachten, was mit $G$ nach der Verarbeitung von $K$ passiert.
\begin{enumerate}[label={\upshape(\roman*)}]
  \item Falls gilt: $F \cup Z = V_c$, dann entsteht keine neue Zusammenhangskomponente in $G$. \label{lem:neue-komponenten1}
  \item Falls gilt: $F \cup Z \subsetneq V_c$,%\, \land\, F \cup Z \not\subset V \setminus V_c$, 
  dann entstehen zwei neue Zusammenhangskomponenten in $G$ --- $C$ wird in zwei Komponenten gespalten.\label{lem:neue-komponenten2}
  \item Seien $C_1, C_2, ..., C_k \subset G$ voneinander unterschiedliche Zusammenhangskomponenten.\\
  Falls $F \cup Z$ aus mehreren Teilmengen aus $C_1, C_2, ..., C_k$ besteht, gelten für jede Komponente $C_i$ ebenfalls \ref{lem:neue-komponenten1} und \ref{lem:neue-komponenten2}.
\end{enumerate}

\end{lemma}

\begin{proof}
\TODO{was machen wir mit dem Beweis?}
Die Beweise für die entsprechenden Punkte:
\begin{enumerate}[label={\upshape(\roman*)}]
  \item Nach der Definition einer Zusammenhangskomponente gilt: $\nexists x \in V_c: x \in V \setminus V_c$.
  Bei Bearbeiteung von $K$ werden nach Lemma \ref{lem:spiess-numbers} alle Kanten zwischen 
  $x \in V \setminus V_c$ und $y \in V_c$ aus $E$ entfernt.
  Deshalb werden bei so einer Spießkombination $K$ keine Kanten entfernt.
  \item blablabla
  %Laut dieser Bedingung gilt: $\exists x \in A \cap V_c: x \notin F$ und
  %$\exists y \in B \cap V_c: y \notin Z$.\\
\end{enumerate}
\end{proof}
\end{comment}


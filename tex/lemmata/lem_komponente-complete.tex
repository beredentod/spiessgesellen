\begin{lemma}\label{lem:komponente-complete}
Sei $C = (V_c, E_c)$ eine beliebige Zusammenhangskomponente in $G$. 
Dann bildet $C$ nach Verarbeitung jeder $k$--ten Spießkombination
selbt einen vollständigen, bipartiten Graphen.
\end{lemma}

\begin{proof}
\TODO{check proof and lemma}
Diese Aussage kann durch die vollständige Induktion für jedes $k \in \mathbb{N}$ bewiesen werden.\\

\noindent
\tbf{Induktionsanfang}: 
Die beiden Mengen $A$ und $B$ sind gleichmächtig und ganz am Anfang ist $G$ vollständig.
Sei die erste Spießkombination $K_1 = (F_1, Z_1)$, wobei $F_1 \neq A$. (Falls $F_1 = A$,
dann gilt die Aussage sofort für $k = 1$.)
Nach der Verarbeitung von $K_1$ 
entstehen zwei Zusammenhangskomponenten: $C_1 \cup C_2 = G$, wobei o.B.d.A $C_1 = F \cup Z$.
Dann sind $C_1 \cap A$ und $C_1 \cap B$ nach Definition \ref{def:spiesskomb} auch gleichmächtig. 
Ebenfalls sind dann $C_2 \cap A$ und $C_2 \cap B$ gleichmächtig.
Nach \ref{lem:spiess-numbers} gilt, dass alle Kanten zwischen $C_1$ und $C_2$
aus $E$ entfernt wurden, aber alle innerhalb von $C_1$ und innerhalb von $C_2$
beibehalten wurden.
Dies bedeutet, dass die Komponenten $C_1$ und $C_2$ selbst vollständige, bipartite Graphen sind.\\
Damit ist die Aussage für $k = 1$ bewiesen und der Induktionsanfang erledigt.\\

\noindent
\tbf{Induktionsschritt}: Es gelte die Aussage, also die \tbf{Induktionsannahme},
für $k \in \mathbb{N}$, d.h., es gelte,
dass jede Zusammenhangskomponente in $G$ nach Verarbeitung von $k$ Spießkombinationen selbst
einen vollständigen, bipartiten Graphen bildet.\\

Zu zeigen ist die Aussage für $k + 1$, also, dass jede Zusammenhangskomponente in $G$ nach Verarbeitung
von $k + 1$ Spießkombinationen selbst einen vollständigen, bipartiten Graphen bildet.\\

Sei $K_i = (F_i, Z_i)$ die $k+1$--te Spießkombination.
Zu untersuchen ist die folgende Fallunterscheidung:
\begin{enumerate}[label={\upshape(\roman*)}]
  \item Sei $D = (V_D, E_D)$ eine Zusammenhangskomponente in $G$. Sei $F_i \cup Z_i = V_D$.
  Da alle Knoten der Spießkombination sich mit allen Knoten von $D$ decken,
  können, nach Korollar \ref{kor:komponente-mengen}, keine Kanten aus $E$ entfernt werden, deshalb
  entstet keine neue Zusammenhangskomponente, also ist jede Zusammenhangskomponente
  nach der Induktionsannahme ein vollständiger, bipartiter Graph.\label{lem:komponente-complete1} 
  %Damit ist der Induktionsschritt für diesen Fall vollzogen und die Behauptung gilt für jedes
  %$k \in \mathbb{N}$.
  \item Sei $D = (V_D, E_D)$ eine Zusammenhangskomponente in $G$. Sei $F_i \cup Z_i \subsetneq V_D
  \land (F_i \cup Z_i) \not\subset (A \cup B)$, also $F_i \cup Z_i$ gehört nur zu einer
  Zusammenhangskomponente in $G$.
  $D$ ist laut Induktionsannahme selbt ein vollständiger, bipartiter Graph.
  Nach Korollar \ref{kor:komponente-mengen} werden alle Kanten zwischen 
  allen $x \in F_i$ und allen $y \in V_D \cap Z_i$ entfernt.
  So entstehen zwei neue Zusammenhangskomponenten:
  $C_1 = F_i \cup Z_i$ und $C_2 = V_D \setminus (F_i \cup Z_i)$, die ebenfalls selbt 
  vollständige, bipartite Grpahen sind.
  Jede andere Zusammenhangskomponente in $G$ ist
  nach der Induktionsannahme ein vollständiger, bipartiter Graph.\label{lem:komponente-complete2} 
  \item Sei $1 \leqslant p \leqslant n$ beliebig, aber fest.
  Seien $C_1, C_2, ..., C_p$ untereinander unterschiedliche Zusammnhangskomponenten in $G$.
  Gehöre $(F_i \cup Z_i)$ zu mehreren Komponenten $C_p, ..., C_q$.
  Dann gilt für jede Zusammenhangskomponente $C_i$ entweder \ref{lem:komponente-complete1} 
  oder \ref{lem:komponente-complete2}, abhängig davon, ob $C_i$ vollständig zu $F_i \cup Z_i$
  gehört oder nur zum Teil. Das bedeutet, entweder entsteht keine neue Zusammenhangskomponente 
  \ref{lem:komponente-complete1} oder $C_i$ wird in zwei neue Zusammenhangskomponenten gespalten
  \ref{lem:komponente-complete2}.
\end{enumerate}

Da alle mögliche untersucht wurden, ist der Induktionsschritt vollzogen und die Behauptung gilt für jedes
$k \in \mathbb{N}$.
\end{proof}

\begin{lemma}\label{lem:komponente-complete}
Sei $C = (V_c, E_c)$ eine beliebige Zusammenhangskomponente in $G$. 
Dann bildet $C$ nach Verarbeitung jeder $k$--ten Spießkombination
selbst einen vollständigen, bipartiten Graphen.
\end{lemma}

\begin{proof}
Diese Aussage kann durch vollständige Induktion über $k \in \mathbb{N}$ bewiesen werden.

\tbf{Induktionsanfang}: 
Die beiden Mengen $A$ und $B$ sind gleichmächtig und ganz am Anfang ist $G$ vollständig.
Sei die erste Spießkombination $K_1 = (F_1, Z_1)$, wobei $F_1 \neq A$. (Falls $F_1 = A$,
dann gilt sofort die Aussage für $k = 1$.)
Nach Lemma \ref{lem:spiess-numbers} werden alle Kanten zwischen 
allen $x \in F_1$ und allen $y \in B \setminus Z_1$,
sowie zwischen allen $p \in Z_1$ und allen $q \in A \setminus F_1$ entfernt.
Nach der Verarbeitung von $K_1$ 
entstehen so zwei Zusammenhangskomponenten:
$C_1 = (L_1 \cup R_1, E_1)$ und $C_2 = (L_2 \cup R_2, E_2)$,
$C_1 \cup C_2 = G$, wobei o.B.d.A $L_1 \cup R_1 = F_1 \cup Z_1$.
Dann sind $L_1$ und $R_1$
nach Definition \ref{def:spiesskomb} auch gleichmächtig. 
Ebenfalls sind dann $L_2$ und $R_2$ gleichmächtig.
Nach Lemma \ref{lem:spiess-numbers} gilt, dass alle Kanten zwischen $C_1$ und $C_2$
aus $E$ entfernt werden, aber alle Kanten innerhalb von $C_1$ und innerhalb von $C_2$
beibehalten werden.
Dies bedeutet, dass die Komponenten $C_1$ und $C_2$ selbst vollständige, bipartite Graphen sind.
Damit ist die Aussage für $k = 1$ bewiesen und der Induktionsanfang erledigt.

\tbf{Induktionsschritt}: Es gelte die Aussage, also die \tbf{Induktionsannahme},
für ein beliebiges, aber festes $k \in \mathbb{N}$, d.h., es gelte,
dass jede Zusammenhangskomponente in $G$ nach Verarbeitung von $k$ Spießkombinationen selbst
einen vollständigen, bipartiten Graphen bildet.

Zu zeigen ist die Aussage für $k + 1$, also, dass jede Zusammenhangskomponente in $G$ nach Verarbeitung
von $k + 1$ Spießkombinationen selbst einen vollständigen, bipartiten Graphen bildet.

Sei $K_i = (F_i, Z_i)$ die $k+1$--te Spießkombination.
Zu untersuchen ist die folgende Fallunterscheidung:
\begin{enumerate}[label={\upshape(\roman*)}]
  \item Sei $D = (V_D, E_D)$ eine Zusammenhangskomponente in $G$. Sei $F_i \cup Z_i = V_D$.
  Da alle Knoten der Spießkombination sich mit allen Knoten von $D$ decken,
  können keine Kanten nach Korollar \ref{kor:komponente-mengen} aus $E$ entfernt werden, deshalb
  entsteht keine neue Zusammenhangskomponente, also ist jede Zusammenhangskomponente
  nach der Induktionsannahme ein vollständiger, bipartiter Graph.\label{lem:komponente-complete1} 
  %Damit ist der Induktionsschritt für diesen Fall vollzogen und die Behauptung gilt für jedes
  %$k \in \mathbb{N}$.
  \item Sei $D = (L_D \cup R_D = V_D,\, E_D)$ eine Zusammenhangskomponente in $G$.
  Sei $F_i \cup Z_i \subsetneq V_D$, also gehört $F_i \cup Z_i$ nur zu einer
  Zusammenhangskomponente in $G$, aber deckt sich nicht mit allen Knoten.
  $D$ ist laut Induktionsannahme selbst ein vollständiger, bipartiter Graph.
  Nach Korollar \ref{kor:komponente-mengen} werden alle Kanten zwischen 
  allen $x \in F_i$ und allen $y \in R_D \setminus Z_i$,
  sowie zwischen allen $p \in Z_i$ und allen $q \in L_D \setminus F_i$ entfernt.
  So entstehen zwei neue Zusammenhangskomponenten: $C_1 = (L_1 \cup R_1, E_1)$ und $C_2 = (L_2 \cup R_2, E_2)$,
  o.B.d.A. $L_1 \cup R_1 = F_i \cup Z_i$ und $L_2 \cup R_2 = V_D \setminus (F_i \cup Z_i)$, die ebenfalls selbst 
  vollständige, bipartite Graphen sind.
  Jede andere Zusammenhangskomponente in $G$ ist
  nach der Induktionsannahme ein vollständiger, bipartiter Graph.\label{lem:komponente-complete2}

  \item Sei $1 \leqslant t \leqslant n$ beliebig ($n$ ist die Anzahl der Obstsorten), aber fest.
  Seien $C_1, C_2, ..., C_t$ paarweise verschiedene Zusammenhangskomponenten in $G$.
  Gehöre $F_i \cup Z_i$ zu mehreren Komponenten $C_p, ..., C_q$.
  Dann gilt für jede Zusammenhangskomponente $C_i$ entweder \ref{lem:komponente-complete1} 
  oder \ref{lem:komponente-complete2}, abhängig davon, ob $C_i$ vollständig zu $F_i \cup Z_i$
  gehört oder nur zum Teil. Das bedeutet, entweder entsteht keine neue Zusammenhangskomponente 
  \ref{lem:komponente-complete1} oder $C_i$ wird in zwei neue Zusammenhangskomponenten gespalten
  \ref{lem:komponente-complete2}.
\end{enumerate}

Da alle möglichen Fälle untersucht wurden, ist der Induktionsschritt vollzogen und die Behauptung gilt für jedes
$k \in \mathbb{N}$.
\end{proof}

\begin{korollar}\label{kor:komponente-mengen}
Sei $C = (L_c \cup R_c, E_c)$ eine Zusammenhangskomponente in $G$.
%$L_c \subseteq A$ und $R_c \subseteq B$ seinen die zwei Partitionen von $C$ und $E_c$ sei die Menge
%der Kanten in $C$.
Sei $K = (F, Z)$ eine Spießkombination.
%Falls $F \subsetneq L_c$ gilt, dann:
Falls $F \subseteq L_c$ gilt, dann gilt für jede Obstsorte $o(x, i)$, wobei $x \in F$:
\begin{enumerate}[label={\upshape(\roman*)}]
	\item $i \in Z$,
	\item $i \notin R_c \setminus Z$.
  %\item Für jede Obstsorte $o(p, i)$, wobei $p \in F$, gilt: $i \in Z \land i \notin R_c \setminus Z$,\label{lem:komponente-mengen1}
  %\item Für jede Obstsorte $o(q, j)$, wobei $q \in L_c \setminus F$, gilt: 
  %$(j \in R_c \setminus Z) \land j \notin Z$.\label{lem:komponente-mengen2}
\end{enumerate}
\end{korollar}

\begin{comment}
\begin{proof}
\ref{lem:komponente-mengen1} gilt nach Defintion \ref{def:spiesskomb}, Axiom \ref{ax:obstsorte-index}
und Lemma \ref{lem:spiess-numbers}.\\
\ref{lem:komponente-mengen2} gilt aus dem Grund, dass $L_c$ und $R_c$ gleichmächtig sind.
(Sonst, könnte man nicht allen $x \in A$ einen $y \in B$ zuweisen.)
\TODO{Beweis zu Ende}
\end{proof}
\end{comment}

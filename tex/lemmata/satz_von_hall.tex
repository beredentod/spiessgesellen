\begin{satz}[Satz von Hall]
Sei $\mathcal{G} = (\mathcal{L} \cup \mathcal{R}, \mathcal{E})$ ein bipartiter, ungerichteter Graph.
Es existiert ein perfektes Matching genau dann,
wenn es für alle Teilmengen $\mathcal{K} \subseteq \mathcal{L}$ gilt: $|\mathcal{K}| \leqslant |N(\mathcal{K})|$.
\end{satz}

\begin{proof}
Auf den Beweis verzichte ich. Ein Beweis ist beispielsweise
\href{https://homes.cs.washington.edu/~anuprao/pubs/CSE599sExtremal/lecture6.pdf}{hier}
\footnote{Anup Rao. Lecture 6 Hall’s Theorem. October 17, 2011. University of Washington. [Zugang 21.01.2021]\\
\url{https://homes.cs.washington.edu/~anuprao/pubs/CSE599sExtremal/lecture6.pdf}} zu finden.
\end{proof}

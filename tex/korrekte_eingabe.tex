\subsection{Prüfung auf Korrektheit der Eingabe}\label{sec:korrektheit-eingabe}
Am Ende des Teils \ref{sec:formulierung} wurde bemerkt, dass die Korrektheit und Vollständigkeit
der Lösung davon abhängt, ob alle Obstsorten in einer Eingabe Axiom \ref{ax:obstsorte-index} folgen.

Indentifizieren wir zuerst die Probleme, die auftreten können.
In den folgenden Überlegungen nehmen wir an, dass jede Spießkombination $K = (F_i, Z_i)$ so gebildet wird,
dass gilt: $|F_i| = |Z_i|$.
(Falls man dies nicht angenommen hätte, wäre eine Eingabe schon an dieser Stelle
falsch, da eine Obstsorte zwei Indizes oder zwei Obstsorten einen Index haben müssten.
Außerdem ist dieser Fehler leicht herauszufinden, indem man beim Einlesen prüft,
ob die beiden Mengen gleichmächtig sind.)
%\TODO{dopisac ten przypadek}
Im Allgemeinen kommt es zu einem Widerspruch, wenn die Eingabe
dem Axiom \ref{ax:obstsorte-index} nicht folgt. 
Das heißt, es können die folgenden 
Möglichkeiten auftreten:
\begin{enumerate}[label={(P\arabic*)}]
  \item In einer Eingabe existieren zwei Obstsorten: $o(x, i)$ und $o(y, i)$, wobei $x \neq y$,\label{probleme1}
  \item In einer Eingabe existieren zwei Obstsorten: $o(x, i)$ und $o(x, j)$, wobei $i \neq j$.\label{probleme2}
\end{enumerate}

Untersuchen wir die Situation, in der die folgenden 
zwei Obstsorten existieren: $o(x, i)$ und $o(y, j)$.
Nehmen wir an dieser Stelle an, dass $i=j$.
Betrachten wir dazu zwei Spießkombinationen: $K_1 = (F_1, Z_1)$ und $K_2 = (F_2, Z_2)$.
Es gelte: $x \in F_1$ und entsprechend $i \in Z_1$.
%\TODO{Bild vielleicht?}
\begin{enumerate}[label={\upshape(F\arabic*)}]
  \item Falls $y \in F_1$ und $i = j$, dann ist $i$ bereits in $Z_1$. Damit $|F_1| = |Z_1|$ gilt,
  muss gelten: $\exists o(z, k) : z \notin F_1 \land k \in Z_1$.
  Dann muss zwar kein Widerpsurch erfolgen, aber wir haben der Obstsorte einen Index zugewiesen,
  also kann an dieser Stelle die Beziehung zwischen $z$ und $k$ gar nicht festgestellt werden.
  Falls alle anderen Spießkombinationen widerspruchsfrei sind,
  wird $z$ ein Index $\ell \in B \setminus F_1$ zugewiesen.\label{fall1}

  \item Falls $y \notin F_1 \land y \in F_2 \land x \notin F_2 \land i =j$, dann ist $i$ bereits in $Z_1$.
  Dann muss für $i$ auch gelten: $i \in Z_2$. Am Anfang ist der bipartite Graph $G$ vollständig.
  Nach der Verarbeitung der Spießkombination $K_1$ werden alle Kanten zwischen allen
  $p \in Z_1$ und allen $q \in A \setminus F_1$, sowie alle Kanten zwischen allen
  $p \in F_1$ und allen $q \in B \setminus Z_1$ entfernt, darunter auch die Kante zwischen
  $y$ und $i$. Nach der Verarbeitung von $K_2$ wird auch die Kante zwischen $x$ und $i$
  entfernt, da $x \notin F_2$. Der Knoten $x$ hat dann eine Kardinalität um 1 kleiner
  als der Rest der Knoten auf dieser Komponente. Insbesondere: Wenn die Mengen
  $F_1$ und $F_2$ jeweils eine Mächtigkeit von 2 haben, hat $x$ dann den Grad 0.\label{fall2}
\end{enumerate}

Bei der Untersuchung der Situation für \ref{probleme2} geht man durch eine analoge Fallunterscheidung wie
in \ref{fall1} und \ref{fall2} vor.\\

Um zu prüfen, ob die Eingabe Axiom \ref{ax:obstsorte-index} widerspricht, muss man deshalb
nur untersuchen, ob die Kardinalität jedes Knotens mit der Kardinalität eines seiner 
Nachbarn nicht übereinstimmt.\\
Dazu muss man beachten, dass die Zahl $n$ in einigen Beispieldateien größer ist
als die Anzahl der in Spießkombinationen und in der Wunschliste verwendeteten Obstsorten
und Indizes. In diesem Fall muss man die nicht genutzten Obstsorten und Indizes beim Einesen
entsprechend markieren und sie beim Prüfen auf Korrektheit der Eingabe überspringen. Mehr dazu
in der \nameref{sec:umsetzung}.\\

Sehen Sie dazu die folgenden Beispiele: \nameref{example:8}, \nameref{example:9}, \nameref{example:14}.

%Untersuchen wir noch die Situation, in der die Mengen $F$ und $Z$ einer Spießkombination $K$ nicht
%gleichmächtig seien. Es gelte: $|F| = |Z| + h, h \in \mathbb{N}$.

\subsection{Zusammenhangskomponenten}
Nach der Verarbeitung aller $m$ Spießkombinationen verfügen wir über den Graphen $G$,
in dem viele Kanten in $E$ entfernt wurden.
Auf diese Weise können wir schon anfangen, die Indizes der Obstsorten aus $W$ festzulegen.
Definieren wir zunächst, was generell ein \tbf{Matching} ist.

\begin{definition}[Matching]\label{def:matching}
Sei $\mathcal{G} = (\mathcal{V}, \mathcal{E})$ ein ungerichteter Graph.
Als ein \textbf{Matching} bezeichnen wir eine Teilmenge $\mathcal{S} \subseteq \mathcal{E}$,
sodass für alle $v \in \mathcal{V}$ gilt, dass höchstens eine Kante 
aus $\mathcal{S}$ inzident zu $v$ ist.\\
Wir bezeichnen einen Knoten $v \in \mathcal{V}$ als in $\mathcal{S}$ \textbf{gematcht},
wenn eine Kante aus $\mathcal{S}$ inzident zu $v$ ist.\\\textnormal{\cite[S.~732]{cormen:matchings}}.
\end{definition}


Zwischen verschiedenen Typen des Matchings unterscheidet man auch das \tbf{perfekte Matching}.

\begin{definition}[Perfektes Matching]\label{def:perfect-matching}
Sei $\mathcal{G} = (\mathcal{V}, \mathcal{E})$ ein ungerichteter Graph. Ein \textbf{perfektes Matching} 
 ist so ein Matching, in dem alle Knoten aus $\mathcal{V}$ gematcht sind.
\end{definition}


Um die Aufgabe in der Form zu lösen, eignet sich gut der \tbf{Satz von Hall},
der als ein Ausgangspunkt der ganzen Matching--Theorie gilt. 
Um sich dieses Satzes zu bedienen, muss man noch den Begriff der \tbf{Nachbarschaft} einführen.

\begin{definition}[Nachbarschaft]\label{def:nachbarschaft}
Sei $\mathcal{G} = (\mathcal{V}, \mathcal{E})$ ein ungerichteter Graph.
Für alle $X \subseteq \mathcal{V}$ definieren wir die \textbf{Nachbarschaft}
von $X$ als $N(X) = \{y \in \mathcal{V} \mid \exists x \in X : (x, y) \in \mathcal{E}\}$.\textnormal{\cite[S.~735, Übung]{cormen:matchings}}
\end{definition}


\begin{satz}[Satz von Hall]
Sei $\mathcal{G} = (\mathcal{L} \cup \mathcal{R}, \mathcal{E})$ ein bipartiter, ungerichteter Graph.
Es existiert ein perfektes Matching genau dann,
wenn es für alle Teilmengen $\mathcal{K} \subseteq \mathcal{L}$ gilt: $|\mathcal{K}| \leqslant |N(\mathcal{K})|$.\textnormal{\cite[S.~736, Übung]{cormen:matchings}}
\end{satz}

\begin{proof}
Auf den Beweis verzichte ich. Ein Beweis ist beispielsweise
\href{https://homes.cs.washington.edu/~anuprao/pubs/CSE599sExtremal/lecture6.pdf}{hier}
\footnote{Anup Rao. Lecture 6 Hall’s Theorem. October 17, 2011. University of Washington. [Zugang 21.01.2021]\\
\url{https://homes.cs.washington.edu/~anuprao/pubs/CSE599sExtremal/lecture6.pdf}} zu finden.
\end{proof}


An dieser Stelle stellen wir Folgendes fest.

\begin{lemma}\label{lem:komponente-complete}
Sei $C = (V_c, E_c)$ eine beliebige Zusammenhangskomponente in $G$. 
Dann bildet $C$ nach Verarbeitung jeder $k$--ten Spießkombination
selbt einen vollständigen, bipartiten Graphen.
\end{lemma}

\begin{proof}
Diese Aussage kann durch die vollständige Induktion für jedes $k \in \mathbb{N}$ bewiesen werden.\\

\noindent
\tbf{Induktionsanfang}: 
Die beiden Mengen $A$ und $B$ sind gleichmächtig und ganz am Anfang ist $G$ vollständig.
Sei die erste Spießkombination $K_1 = (F_1, Z_1)$, wobei $F_1 \neq A$. (Falls $F_1 = A$,
dann gilt sofort die Aussage für $k = 1$.)
Nach der Verarbeitung von $K_1$ 
entstehen zwei Zusammenhangskomponenten: $C_1 = (L_1 \cup R_1, E_1)$ und $C_2 = (L_2 \cup R_2, E_2)$,
$C_1 \cup C_2 = G$, wobei o.B.d.A $L_1 \cup R_1 = F \cup Z$.
Dann sind $L_1$ und $R_1$
nach Definition \ref{def:spiesskomb} auch gleichmächtig. 
Ebenfalls sind dann $L_2$ und $R_2$ gleichmächtig.
Nach Lemma \ref{lem:spiess-numbers} gilt, dass alle Kanten zwischen $C_1$ und $C_2$
aus $E$ entfernt werden, aber alle Kanten innerhalb von $C_1$ und innerhalb von $C_2$
beibehalten werden.
Dies bedeutet, dass die Komponenten $C_1$ und $C_2$ selbst vollständige, bipartite Graphen sind.\\
Damit ist die Aussage für $k = 1$ bewiesen und der Induktionsanfang erledigt.\\

\noindent
\tbf{Induktionsschritt}: Es gelte die Aussage, also die \tbf{Induktionsannahme},
für ein beliebiges, aber festes $k \in \mathbb{N}$, d.h., es gelte,
dass jede Zusammenhangskomponente in $G$ nach Verarbeitung von $k$ Spießkombinationen selbst
einen vollständigen, bipartiten Graphen bildet.\\

Zu zeigen ist die Aussage für $k + 1$, also, dass jede Zusammenhangskomponente in $G$ nach Verarbeitung
von $k + 1$ Spießkombinationen selbst einen vollständigen, bipartiten Graphen bildet.\\

Sei $K_i = (F_i, Z_i)$ die $k+1$--te Spießkombination.
Zu untersuchen ist die folgende Fallunterscheidung:
\begin{enumerate}[label={\upshape(\roman*)}]
  \item Sei $D = (V_D, E_D)$ eine Zusammenhangskomponente in $G$. Sei $F_i \cup Z_i = V_D$.
  Da alle Knoten der Spießkombination sich mit allen Knoten von $D$ decken,
  können keine Kanten nach Korollar \ref{kor:komponente-mengen} aus $E$ entfernt werden, deshalb
  entsteht keine neue Zusammenhangskomponente, also ist jede Zusammenhangskomponente
  nach der Induktionsannahme ein vollständiger, bipartiter Graph.\label{lem:komponente-complete1} 
  %Damit ist der Induktionsschritt für diesen Fall vollzogen und die Behauptung gilt für jedes
  %$k \in \mathbb{N}$.
  \item Sei $D = (L_D \cup R_D = V_D,\, E_D)$ eine Zusammenhangskomponente in $G$.
  Sei $F_i \cup Z_i \subsetneq V_D$, also gehört $F_i \cup Z_i$ nur zu einer
  Zusammenhangskomponente in $G$, aber deckt sich nicht mit allen Knoten.
  $D$ ist laut Induktionsannahme selbt ein vollständiger, bipartiter Graph.
  Nach Korollar \ref{kor:komponente-mengen} werden alle Kanten zwischen 
  allen $x \in F_i$ und allen $y \in R_D \setminus Z_i$,
  sowie zwischen allen $p \in Z_i$ und allen $q \in L_D \setminus F_i$ entfernt.
  So entstehen zwei neue Zusammenhangskomponenten: $C_1 = (L_1 \cup R_1, E_1)$ und $C_2 = (L_2 \cup R_2, E_2)$,
  o.B.d.A. $L_1 \cup R_1 = F_i \cup Z_i$ und $C_2 = V_D \setminus (F_i \cup Z_i)$, die ebenfalls selbt 
  vollständige, bipartite Grpahen sind.
  Jede andere Zusammenhangskomponente in $G$ ist
  nach der Induktionsannahme ein vollständiger, bipartiter Graph.\label{lem:komponente-complete2}

  \item Sei $1 \leqslant t \leqslant n$ beliebig, aber fest.
  Seien $C_1, C_2, ..., C_t$ untereinander unterschiedliche Zusammnhangskomponenten in $G$.
  Gehöre $F_i \cup Z_i$ zu mehreren Komponenten $C_p, ..., C_q$.
  Dann gilt für jede Zusammenhangskomponente $C_i$ entweder \ref{lem:komponente-complete1} 
  oder \ref{lem:komponente-complete2}, abhängig davon, ob $C_i$ vollständig zu $F_i \cup Z_i$
  gehört oder nur zum Teil. Das bedeutet, entweder entsteht keine neue Zusammenhangskomponente 
  \ref{lem:komponente-complete1} oder $C_i$ wird in zwei neue Zusammenhangskomponenten gespalten
  \ref{lem:komponente-complete2}.
\end{enumerate}

Da alle möglichen Fälle untersucht wurden, ist der Induktionsschritt vollzogen und die Behauptung gilt für jedes
$k \in \mathbb{N}$.
\end{proof}


\begin{figure}[ht]
\caption{Abbgebildet ist das Beispiel aus der Aufgabenstellung nach
der Verarbeitung von allen $m$ Spießkombinationen.}
\label{fig:graph-after-analysis}
\centering
\begin{subfigure}[b]{.49\textwidth}
\centering
\begin{tikzpicture}
    \node[vertex] (A) {$A$};
    \node[vertex] (B) [below = 0.4cm of A] {$B$};
    \node[vertex] (Br) [below = 0.4cm of B] {$Br$};
    \node[vertex] (E) [below = 0.4cm of Br] {$E$};
    \node[vertex] (P) [below = 0.4cm of E] {$P$};
    \node[vertex] (W) [below = 0.4cm of P] {$W$};
    \node[vertex] (1) [right = 1.5cm of A] {$1$};
    \node[vertex] (2) [right = 1.5cm of B] {$2$};
    \node[vertex] (3) [right = 1.5cm of Br] {$3$};
    \node[vertex] (4) [right = 1.5cm of E] {$4$};
    \node[vertex] (5) [right = 1.5cm of P] {$5$};
    \node[vertex] (6) [right = 1.5cm of W] {$6$};

    \path[draw,thick]
    (A) edge node {} (1)
    (A) edge node {} (4)
    (B) edge node {} (5)
    (Br) edge node {} (1)
    (Br) edge node {} (4)
    (E) edge node {} (2)
    (P) edge node {} (6)
    (W) edge node {} (3);
\end{tikzpicture}

\caption{Der Graph nach der Verarbeitung aller Spießkombinationen}
\label{fig:graph-after-infos}
\end{subfigure}
\begin{subfigure}[b]{.49\textwidth}
\centering
\begin{tikzpicture}
    \node[vertex] (A) {$A$};
    \node[vertex] (Br) [below = 0.4cm of A] {$Br$};
    \node[vertex] (1) [right = 1.5cm of A] {$1$};
    \node[vertex] (4) [right = 1.5cm of Br] {$4$};

    \path[draw,thick]
    (A) edge node {} (1)
    (A) edge node {} (4)
    (Br) edge node {} (1)
    (Br) edge node {} (4);
\end{tikzpicture}
\vspace{2.5cm}

\caption{Die übrige Zusammenhangskomponente mit mehr als 2 Knoten}
\label{fig:component-left}
\end{subfigure}
\end{figure}



\begin{lemma}\label{lem:komponente-matching}
Sei $C = (L_c \cup R_c, E_c)$ eine beliebige
Zusammenhangskomponente in $G$. Dann existiert immer ein perfektes Matching zu $C$.
\end{lemma}
\begin{proof} 
Nach Lemma \ref{lem:komponente-complete} ist jede Zusammenhangskomponente in $G$ ein vollständiger,
bipartiter Graph. Nach Satz von Hall existiert ein perfektes Matching, wenn
für alle Teilmengen $K \subseteq L_c$ gilt: $|K| \leqslant |N(K)|$.
Die obere Behauptung kann für beliebig große Mächtigkeiten $|K| = k \in \mathbb{N}$
durch die vollständige Induktion bewiesen werden.\\

\noindent
\tbf{Induktionsanfang:} Für $k = 1$ hat der einzelne Knoten $x \in K \subseteq L_c$
die Kardinalität $\Delta(x) = 1$.
Deshalb gilt: $|K| = 1 \leqslant |N(x)| = 1$. Damit stimmt die Behauptung für $k = 1$ und der Induktionsanfang ist erledigt.\\

\noindent
\tbf{Induktionsschritt:} Es gelte die Aussage für ein beliebiges $k \in \mathbb{N}$, also für eine Teilemenge
$K \subseteq L_c$, die aus $k$ Knoten besteht und in der jeder Knoten $x \in K$ die Kardinalität
$\Delta(x) = k$ hat.\\ Es gelte also: $|K| \leqslant |N(K)|$.\\
Zu zeigen ist die Aussage für $k + 1$, also für eine Teilmenge $K' \subseteq L_c$ der Mächtigkeit 
$|K'| = k+1$:
\[
|K'| \leqslant |N(K')|.
\] 
Wir verifizieren:

Jeder Knoten in $C$ hat den Grad $k+1$, also: $|K'| = k + 1 \leqslant |N(K')| = (k+1)^2 = k^2 + 2k + 1$.\\
Folglich stimmt die Behauptung für $k+1$.\\

Der Induktionsschritt ist damit vollzogen und es wurde bewiesen, dass die Behauptung für beliebige
Mächtigkeit von $K$ gilt.
Dadurch wurde auch bewiesen, dass es in einer Zusammenhangskomponente in $G$
 immer ein perfektes Matching gibt.
\end{proof}


Nach der Verarbeitung aller Spießkombinationen entsteht ein Graph mit vielen Zusammenhangskomponenten
(s. \cref{fig:graph-after-analysis}).
An dieser Stelle muss man noch die Wunschliste $W$ untersuchen, um die entsprechende Menge $W'$ zu bestimmen.
Dazu muss man die folgenden zwei Beobachtungen betrachten.

\begin{lemma}\label{lem:komponente-all-wunschliste}
Sei $C = (L_c \cup R_c, E_c)$ eine Zusammenhangskomponente in $G$.
Wenn gilt: $\forall x \in L_c : x \in W$, dann werden alle $y \in R_c$ in $W'$ hinzugefügt.
\end{lemma}
\begin{proof} 
Nach Axiom \ref{ax:obstsorte-index} besitzt jede Obstsorte genau einen einzigartigen Index.
Die Zusammenhangskomponente $C$ beschreibt nach Lemmata \ref{lem:komponente-complete} und \ref{lem:komponente-matching},
dass jede Obstsorte $p \in L_c$ jeden Index $q \in R_c$ haben kann, weil $C$ ein vollständiger, bipartiter Graph ist und ein perfektes Matching stets existiert.

Dadurch, dass $\forall x \in L_c : x \in W$ gilt,
ist ohne Bedeutung, welchen Index die jeweilige Obstsorte besitzt, da
 die Lösung des Problems eine Menge $W'$ mit den Indizes der Obstsorten aus $W$ sein soll.\\
Dadurch, dass $L_c \subseteq W$ gilt, gilt auch: $R_c \subseteq W'$. 
\end{proof}


\begin{lemma}\label{lem:komponente-one-not-wunschliste}
Sei $C = (L_c \cup R_c, E_c)$ eine Zusammenhangskomponente in $G$.
Wenn gilt: $\exists x \in L_c : x \notin W$ und $\exists y \in L_c : y \in W$,
dann kann die Menge $W'$ nicht eindeutig
festgelegt werden.
\end{lemma}
\begin{proof}
Nach Axiom \ref{ax:obstsorte-index} besitzt jede Obstsorte genau einen einzigartigen Index.
Die Zusammenhangskomponente $C$ beschreibt nach Lemmata \ref{lem:komponente-complete} und \ref{lem:komponente-matching},
dass jede Obstsorte $p \in L_c$ jeden Index $q \in R_c$ haben kann, weil $C$ ein vollständiger, bipartiter Graph ist und ein perfektes Matching stets existiert.

Angenommen, $\exists r \in L_c : r \notin W$. Dann ist es unmöglich, festzustellen,
welcher Index aus $R_c$ der Obstsorte $r$ gehört.
Also ist es auch unmöglich, festzustellen, welche Indizes in $W'$ hinzugefügt werden sollen.
Deshalb
ist es unmöglich (unabhängig von allen anderen Zusammenhangskomponenten des Graphen $G$),
eine eindeutige Menge der Indizes der gewünschten Obstsorten festzulegen.
Dadurch gibt es keine eindeutige Lösung zu diesem Problem für diese Eingabe.
\end{proof}


Direkt aus \cref{lem:komponente-all-wunschliste} ergibt sich das folgende Korollar.
Man bedient sich dessen und des Lemmas \ref{lem:komponente-one-not-wunschliste}, um das ganze Problem zu lösen, also: Ob die Menge $W'$ eindeutig bestimmt werden kann.

\begin{korollar}
Seien $C_1 = (L_1 \cup R_1, E_1), ..., C_k = (L_k \cup R_k, E_k)$ alle Zusammenhangskomponenten in $G$,
für jede $i$--te von denen gilt: $\exists x \in L_i : x \in W$.
Falls für jede $i$--te von diesen Komponenten gilt: $L_i \subseteq W$, dann
kann $W'$ eindeutig und vollständig bestimmt werden.
\end{korollar}


Man stellt fest, dass man die Menge $W$ untersuchen kann und wenn ein $x \in W$ in $G$ die
Kardinalität $\Delta(x) = 1$ besitzt, kann der einzelne Nachbar von $x$ in $W'$ hinzugefügt werden
(Lemma \ref{lem:komponente-all-wunschliste}).
Im sonstigen Fall, also wenn $\Delta(x) > 1$, muss die ganze Zusammenhangskomponente
$C_x = (L_x \cup R_c, E_x)$, zu der $x$ gehört, untersucht werden, ob gilt: $\forall p \in L_x: p \in W$
(Lemmata \ref{lem:komponente-all-wunschliste} und \ref{lem:komponente-one-not-wunschliste}).

Man erstellt eine Liste $\bar{W}$ der Länge $n$,
in der die Zugehörigkeit einer Obstsorte zur Wunschliste $W$ durch 1 oder 0 gekennzeichnet wird
(s. \nameref{sec:umsetzung}). Außerdem erstellt wird eine Liste $\bar{R}$ der Länge $n$,
in der jede gewünschte Obstsorte $x$ als 1 gekennzeichnet wird, falls der Knoten $x$ in $G$ bereits
besucht wurde (s. \nameref{sec:umsetzung}).

Wenn man einen Knoten $x \in W$ untersucht, dessen Kardinalität $\Delta(x) > 1$ ist,
kann man die Liste der Nachbarknoten $n(x)$ von $x$ aufrufen.
Da eine Zusammenhangskomponente selbst vollständig ist\\ (\cref{lem:komponente-complete}),
kann man die Liste der Nachbarknoten $n(y)$ eines beliebigen Nachbarn $y$ von $x$ ($y \in n(x)$) aufrufen.
So kann man jeden Knoten $z \in n(y)$ untersuchen, ob bei $z$ eine 1 in $\bar{W}$ steht.
Falls ja, wird $z$ auch in $\bar{R}$ markiert,
sodass man denselben Vorgang bei einem anderen Knoten in dieser Komponente nicht wiederholen muss.
Falls alle $z$ zu $W$ gehören, wird
die ganze Liste $n(x)$ in $W'$ hinzugefügt. Sonst werden alle Knoten dieser Komponente 
gespeichert, insbesondere diese Obstsorten, die zu $W$ nicht gehören.\\
Man wiederholt diesen Vorgang, bis alle gewünschten Obstsorten mit 1 in $\bar{R}$ markiert werden.

Ausgegeben wird entweder die vollständige Menge $W'$ oder eine Meldung über die jeweilige 
Zusammenhangskomponente, zu der Obstsorten gehören, die nicht gewünscht waren.
Diese werden auch in der Ausgabe aufgezählt.

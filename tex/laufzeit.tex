\subsection{Laufzeit}\label{sec:laufzeit}
\begin{itemize}
  \item $n$ --- die Anzahl der Obstsorten
  \item $m$ --- die Anzahl der Spießkombinationen
  \item $w$ --- die Anzahl der Wünsche (also $|W|$), im worst-case $w = n$, weil $w \leqslant n$
\end{itemize}

Da wir Bitmasken in unserem Programm verwenden, ist noch eine Konstante einzuführen: $\beta$,
die für die $\beta$--Bit--Architektur eines Rechners\footnote{\href{https://en.wikipedia.org/wiki/Word_(computer_architecture)}{https://en.wikipedia.org/wiki/Word\_(computer\_architecture)}}
steht, auf dem das Programm ausgeführt wird. D.h., bei der 64--Bit--Architektur beträgt $\beta = 64$.
Die bitweisen Operationen in C++ auf \ttt{bitset} werden in der Laufzeit von $O(\frac{|k|}{\beta})$
ausgefüht, wobei $|k|$ die Länge eines \ttt{bitset} ist. Eine Operation wird nicht auf einem einzlenen Bit ausgeführt, sondern es handelt sich um eine Operation auf einem integrierten Schaltkreis, deshalb
stellt $\beta$ die Größe des Datenwortes des Rechners dar.
Außerdem muss die Länge eines \ttt{bitset}
konstant sein, d.h., man muss schon im Programm eine feste Länge für alle Eingabegrößen eingeben.
Diese feste Länge nennen wir $N$ und setzen $N =26$, da so viele Obstsorten das größte Beispiel
auf der BWINF--Webseite umfasst.

\begin{itemize}
  \item Einlesen: $O(n (\frac{N}{\beta} + m \log n) + w (\log n + \log w))$ (worst-case)
  \TODO{dodać haszowanie}
  \begin{itemize}
    \item Erstellung der Adjazenzmatrix $M$: $O(n \cdot \frac{N}{\beta})$

    \item Erstellung der Liste \ttt{used} (s. \nameref{sec:umsetzung}): $O(n)$

    \item Einlesen der Menge $W$: $O(w \log w)$\footnote{Praktisch sollte man zu dieser Laufzeit noch den Aufwand vom Vergleichen von zwei Wörtern in der Menge hinzurechnen, aber die Längen der Wörter sind so klein, dass
     ich dies als eine zu vernachlässigende Konstante betrachte.\label{foot:vergleich-aufwand}}\\
     Implementierug von \ttt{set} in C++ als Rot-schwarz-Bäume\footnote{\href{https://en.cppreference.com/w/cpp/container/set}{https://en.cppreference.com/w/cpp/container/set}}.

    \item Einlesen der Spießkombinationen: $O(m \cdot n \log n)$\footref{foot:vergleich-aufwand} (worst-case)\\
    Im schlimmsten Fall enthält jede Spießkombination alle Obstsorten.
    Die logarithmische Laufzeit ist durch Einfügen in eine Menge verursacht.

    \item Zuweisung der internen Indizes zu den Obstsorten (s. \nameref{sec:umsetzung}): $O(n \log n)$\footref{foot:vergleich-aufwand}\\
     Implementierug von \ttt{map} in C++ als Rot-schwarz-Bäume\footnote{\href{https://en.cppreference.com/w/cpp/container/map}{https://en.cppreference.com/w/cpp/container/map}}

    \item Umwandlung der gewünschten Obstsorten von Strings zu Integers: $O(w (\log n + \log w))$\\%worst-case: $O(n \log n)$\\
    Das Suchen in einer \ttt{map} hat logarithmische Laufzeit bezüglich der Anzahl
    der Obstsorten: $O(\log n)$\footref{foot:vergleich-aufwand}.
    Das Einfügen in ein \ttt{set} hat logarithmische Laufzeit bezüglich der Anzahl
    der Wünsche: $O(\log w)$\footref{foot:vergleich-aufwand}. Also gilt für die gesamte Laufzeit: $O(w (\log n + \log w))$. 

    \item Umwandlung der Obstsorten in allen Spießkombinationen von Strings zu Integers:
    $O(m \cdot n \log n)$\\
    In jeder Spießkombination können sich im worst-case alle $n$ Obstsorten befinden.
    Das Suchen in einer \ttt{map} hat logarithmische Laufzeit bezüglich der Anzahl
    der Obstsorten: $O(\log n)$\footref{foot:vergleich-aufwand}.
    Das Einfügen in ein \ttt{set} hat ebenfalls logarithmische Laufzeit bezüglich der Anzahl
    der Obstsorten: $O(\log n)$\footref{foot:vergleich-aufwand}. 
    Deshalb beträgt die Laufzeit für alle Obstsorten in einer Spießkombination höchstens: $O(n \log n)$. 

    \item Die gesamte Laufzeit für diesen Teil (worst-case): $O(n \cdot \frac{N}{\beta}) + O(n) + O(w \log w)
    + O(n \log n) + O(m \cdot n \log n) + O(w (\log n + \log w)) + O(m \cdot n \log n) = 
    O(n \cdot \frac{N}{\beta} + n + w \log w + n \log n + m \cdot n \log n + w (\log n + \log w) +
    m \cdot n \log n) = 
    O(n \cdot \frac{N}{\beta} + m \cdot n \log n + w (\log n + \log w)) = 
    O(n (\frac{N}{\beta} + m \log n) + w (\log n + \log w))$\\
    Man darf nicht vergessen, dass die Laufzeit optimiert werden könnte, wenn die Eingabe nicht
    aus Wörtern bestünde, sondern aus Zahlen. Man könnte beispielsweise die Wörter früher in Zahlen umwandeln lassen.  
  \end{itemize}


  \item Verarbeitung der Spießkombinationen: $O(n\cdot \frac{N}{\beta}(m + n))$ (worst-case)
  \begin{itemize}
    \item Verarbeitung einer Spießkombination $K = (F, Z)$: $O(n \cdot\frac{N}{\beta})$\\
    Man geht davon aus, dass eine Spießkombination im worst-case alle Obstsorten enthält.
    \begin{itemize}
      \item Erstellung der Bitmaske $bf$: $O(n)$

      \item Erstellung der Bitmaske $bn$: $O(|F|)$, worst-case: $O(n)$\\
      Die Operation hat eine lineare Laufzeit bezüglich der Anzahl der Elementen in 
      einer Spießkombination. Eine Spießkombination kann im worst-case alle $n$
      Obstsorten beinhalten.

      \item Erstellung der Bitmaske $br$: $O(\frac{N}{\beta})$

      \item Entfernen der Kanten: $O(n \cdot \frac{N}{\beta})$\\
      Für jede Liste $M_i$ wird geprüft, ob $i$ sich in $F$ befindet.
      Diese Operation kann in $O(1)$ ausgeführt werden, indem wir durch
      die Menge $F$ gleichzeitig iterieren, wie durch die Matrix $M$ (s. \nameref{sec:umsetzung}).
      An jeder Liste $M_i$ wird genau eine bitweise Operation durchgeführt.

      \item Die gesamte Laufzeit für eine Spießkombination beträgt (worst-case):\\
      $O(n) + O(n) + O(\frac{N}{\beta}) + O(n \cdot\frac{N}{\beta}) 
      = O(n + n + \frac{N}{\beta} + n \cdot\frac{N}{\beta}) = O(n \cdot\frac{N}{\beta})$
    \end{itemize}

    \item Verarbeitung aller Spießkombinationen entsprechend: $O(m \cdot n \cdot\frac{N}{\beta})$

    \item Kopieren der Adjazenzmatrix in \ttt{Graph G} (s. \nameref{sec:umsetzung}): average-case: $O(n \cdot \delta \cdot \frac{N}{\beta})$, worst-case: $O(n^2 \cdot \frac{N}{\beta})$ \\
    Für jede Liste $M_i$ werden alle Einsen in \ttt{G} als Kanten vom Knoten $i$ eingefügt.
    Dazu bediene ich mich der eingebauten Funktion \ttt{\_Find\_next()},
    die jeweils das nächste 1--Bit in einem \ttt{bitset} findet.
    Jedoch ihre Laufzeit ist mir nicht bekannt.
    Ich gehe davon aus, dass dieser Vorgang ebenfalls in der Zeit von $O(\frac{N}{\beta})$ abzuschließen ist. 
    (Falls die Laufzeit viel schlechter wäre, könnte man die \href{https://stackoverflow.com/questions/58795338/find-next-array-index-with-true-value-c}{hier}\footnote{\href{https://stackoverflow.com/questions/58795338/find-next-array-index-with-true-value-c}{https://stackoverflow.com/questions/58795338/find-next-array-index-with-true-value-c}} beschriebene Idee anwenden,
    die mit einem logarithmischen Aufwand bezüglich der Länge der Bitmaske läuft.)\\
    Deshalb erfolgt die Iteration über eine Liste $M_i$ in $O(\delta \cdot \frac{N}{\beta})$, wobei
    $\delta$ die Anzahl der Einsen ist. Im schlimmsten Fall, wenn alle Spießkombinationen aus
    $n$ Obstsorten bestehen, gilt: $\delta = n$, also gilt: $O(n \cdot \frac{N}{\beta})$.
    Im allgemeinen Fall gilt: $\delta \ll n$. Den Vorgang muss man für alle $n$ Obstsorten
    ausführen.\\
    Man könnte auch denken, dass das Kopieren unnötig ist. Falls man den Graphen nicht
    in eine Adjazenzliste--Form kopiert,
    muss man sowieso an einer Stelle die entsprechenden Einsen aus der Adjazenzmatrix ablesen.

    \item Die gesamte Laufzeit für diesen Teil beträgt (worst-case):\\
    $O(m \cdot n \cdot\frac{N}{\beta}) + O(n^2 \cdot\frac{N}{\beta})) = 
    O(m \cdot n \cdot\frac{N}{\beta} + n^2 \cdot\frac{N}{\beta})) = O(n\cdot \frac{N}{\beta}(m + n))$ 
    \vspace{9pt}
  \end{itemize}

  \item Prüfung der Korrektheit der Eingabe: $O(n)$
  \begin{itemize}
    \item Für jeden Knoten im Graphen wird geprüft, ob 
    er in \ttt{used} markiert ist ($O(1)$) und ob seine Kardinalität mit der eines seinen
    Nachbarknoten übereinstimmt ($O(1)$). Der Zugriff auf einen der Nachbarknoten erfolgt auch in $O(1)$.\\
    Deshalb gilt für alle Knoten im bipartiten Graphen: $O(n + n) = O(n)$
  \end{itemize}

  \item Prüfung der Existenz einer Lösung: $O(n \log n)$ (worst-case)
  \begin{itemize}
    \item Erstellung von $\bar{W}$: $O(n)$

    \item Erstellung von $\bar{R}$: $O(n)$

    \item Prüfung der Wunschliste: $O(w \log w)$\\
    Es wird geprüft, ob die Kardinalität jedes Knotens 1 beträgt
    \begin{itemize}
    \item Prüfung auf Kardinalität $\Delta(x) = 1$: $O(1)$
    \item Zugriff auf die Liste der Nachbarknoten in $G$: $O(1)$
    \item ggf. Einfügen in $W'$: $O(\log w)$
    \item ggf. Einfügen in \ttt{multip} (s. \nameref{sec:umsetzung}): $O(\log w)$ (worst-case)\\
    Im schlimmsten Fall hat keiner der Knoten in der Wunschliste die Kardinalität von 1.
    \item ggf. Markierung in $\bar{W}$: $O(1)$
    \end{itemize}

    \item Iteration durch \ttt{multip}: $O(n \log w)$ (worst-case)\\
    Die folgenden Operationen werden nur dann ausgeführt, wenn die Komponente,
    zu der der iterierte Knoten gehört, noch nicht bearbeitet wurde, also 
    wenn dieser Knoten in $\bar{R}$ nicht markiert wurde. Wir können feststellen, 
    dass diese Bedingung im schlimmsten Fall nur $\frac{w}{2}$--mal erfüllt werden kann.
    Alle Komponenten mit genau 1 Knoten aus $A$ wurden bereits behandelt und in diesem
    Fall müssten alle Komponenten aus genau 2 Knoten aus $A$ bestehen.
    $\frac{w}{2}$ ist somit die maximale Anzahl an Zusammenhangskomponenten in $G$,
    die mehr als einen Knoten aus $A$ besitzen.
    \begin{itemize}
    \item Prüfung auf Markierung in $\bar{R}$: $O(1)$

    \item ggf. Zugriff auf die Liste der Nachbarn eines iterierten Knotens $x$ ($n(x)$): $O(n)$\\
    Im schlimmsten Fall muss man auf alle Knoten aus $B$ zugreifen.

    \item ggf. Zugriff auf die Liste der Nachbarn $n(y)$ eines Nachbarn $y$ eines iterierten Knotens: $O(n)$\\
    Im schlimmsten Fall muss man auf alle Knoten aus $A$ zugreifen.

    \item ggf. Iteration durch $n(y)$: $O(n)$ (worst-case)\\
    In dieser Schleife wird geprüft, ob der iterierte Index $i$ in $\bar{W}$ markiert ist: $O(1)$,
    und dann wird $i$ in $\bar{R}$ markiert: $O(1)$. Wenn es einen Knoten gibt, der nicht gewünscht ist,
    aber sich auf der Komponente befindet, wird dies mit einer boolschen Variable \ttt{prob} markiert: $O(1)$.\\
    Im worst-case kann die Schleife $n$--mal iteriert werden, wenn es nur eine Zusammenhangskomponente
    in $G$ gibt. Jedoch wird die äußere Schleife nur einmal iteriert, da alle gewünschten 
    Obstsorten auf der Komponente als besucht in $\bar{R}$ markiert werden.
    \item ggf. Kopieren der Knoten aus dieser Komponente zu \ttt{problems} (s. \nameref{sec:umsetzung}): $O(n)$ (worst-case)

    \item ggf. Einfügen der Knoten aus dieser Komponente zu $W'$: $O(n \log w)$ (worst-case)\\
    Das Einfügen in eine Menge von $w$ Elementen hat eine logarithmische Laufzeit bezüglich $w$.

    \item Um die gesamte Laufzeit für diesen Teil zu bestimmen, müssen wir bemerken, dass 
    diese Laufzeit von der Anzahl der Knoten der Menge $A$ auf allen Zusammenhangskomponenten 
    abhängt, auf denen sich mind. eine gewünschte Obstsorte befindet. Durch die Markierung
    in $\bar{R}$ wird jeder Knoten von diesen Zusammenhangskomponenten nur einmal behandelt. 
    Damit ergibt sich im worst-case die Laufzeit von $O(n \log w)$, falls die gewünschten Obstsorten auf 
    allen Zusammenhangskomponenten in $G$ verteilt sind.
    \end{itemize}

    \item Ausgabe für eine Eingabe, für die $W'$ nicht eindeutig bestimmt werden kann: $O(n \log n)$\\
    Es wird durch alle Komponenten iteriert, die mind. eine ungewünschte Obstsorte enthalten,
    und alle Knoten werden mit den Namen der Obstsorten aufgezählt.
    Deshalb muss jedes Mal die Suchfunktion in der \ttt{map} mit den Zuweisungen 
    der internen Indizes der Obstsorten und den Obstsorten aufgerufen werden: $O(\log n)$\footref{foot:vergleich-aufwand}.
    Im schlimmsten Fall muss in der Ausgabe durch alle Knoten in $A$ iteriert werden, falls
    es nur eine Zusammenhangskomponente in $G$ gibt.

    \item Die gesamte Laufzeit für diesen Teil beträgt (worst-case, nach oben abgeschätzt: $n > w$):
    $O(n) + O(n) + O(w \log w) + O(n \log w) + O(n \log n) = O(n + n + w \log w + n \log w + n \log n)
    \subset O(n \log n)$
  \end{itemize}

\end{itemize}

Fassen wir die Laufzeit im worst-case zusammen. Als ein worst-case kann so ein Fall gelten,
in dem alle $m$ Spießkombinationen und die Wunschliste aus allen $n$ Obstsorten bestehen. 
\begin{itemize}
  \item Einlesen: $O(n (\frac{N}{\beta} + m \log n) + w (\log n + \log w))$
  \item Verarbeitung der Spießkombinationen: $O(n\cdot \frac{N}{\beta}(m + n))$
  \item Prüfung der Korrektheit der Eingabe: $O(n)$
  \item Prüfung der Existenz einer Lösung: $O(n \log n)$
\end{itemize}

\TODO{coś z tym zrobić}
\begin{equation*}
\begin{gathered}
O(n (\frac{N}{\beta} + m \log n) + w (\log n + \log w)) +
 O(n\cdot \frac{N}{\beta}(m + n)) + \\
+ O(n) + O(n \log n) =\\
= O(n (\frac{N}{\beta} + m \log n) + w (\log n + \log w) + n\cdot \frac{N}{\beta}(m + n)+ n + n \log n)\\
= O(n (m \log n + m \frac{N}{\beta} + n \frac{N}{\beta})  + w (\log n + \log w) + n \log n)
\end{gathered}
\end{equation*}

Nach der Abschätzung $w \leqslant n$ ergibt sich:
\[
O(n (m \log n + m \frac{N}{\beta} + n \frac{N}{\beta})  + n \log n)) =
O(n (m \log n + m \frac{N}{\beta} + n \frac{N}{\beta}))
\]

\noindent 
Wir können bemerken, dass fast alle modernen Rechner auf einer
mind. 32--Bit-Architektur basieren, das heißt, wir können $\beta = 32$ setzen.
Für die maximale Anzahl an Obstsorten, die in den BWINF--Beispielen auftreten, gilt
dann: $\frac{N}{\beta} = \frac{26}{32} < 1$.
In diesem Fall ist dieser Bruch ein vernachlässiger Faktor. Es ergibt sich im schlimmsten Fall:
\[
O(n (m \log n + m + n))
\]

\subsection{Klasse \ttt{Graph}}
Diese Klasse ist grundsätzlich aus Übersichtlichkeits-- sowie aus Vereinfachungsgründen
entstanden. Theoretisch könnte man die enthaltenen Methoden in der Klasse \ttt{Solver} 
speichern.\\

Der Graph ist als eine Adjazenzliste aus \ttt{vector} von \ttt{vector} von Integers gespeichert.
Der Konstruktor nimmt zwei Parameter: die Größen der beiden Partitionen
im bipartiten Grphen, obwohl sie in der Aufgabe gleich groß sind.\\

Zu den verfügbaren Methoden zählen: \ttt{addEdge()},
die eine ungerichtete Kante zwischen zwei Knoten einfügt; \ttt{deg()}, die die Kardinalität
eines Knotens zurückgibt; \ttt{getNeighbors()}, die den \ttt{vector}, also die Adjazenzliste
eines Knotens zurückgibt. Außerdem gibt es ein paar Methoden, die zum Debugging dienen. 

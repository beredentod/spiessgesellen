\subsection{Bipartiter Graph}
\TODO{use definitions[?]}
Man kann die beiden Mengen $A$ und $B$ zu Knoten eines bipartiten Graphen $G = (A \cup B = V, E)$ umwandeln.
Die Menge der Kanten $E$ wird im Folgenden festgelegt.
Man stellt den Graphen als eine Adjazenzmatrix $M$ der Größe $n \times n$ dar. 
Als $M_i$ bezeichnet wird die Liste der Länge $n$,
die die Beziehungen eines Knotens
$i \in A$ zu jedem Knoten $j \in B$ als 1 (Kante) oder 0 (keine Kante) darstellt.
Als $M_{i, j}$ bezeichnet wird die $j$--te Stelle in der $i$--ten Liste der Matrix.\\
Nach Axiom \ref{ax:obstsorte-index} gehört jeder Obstsorte aus $A$ genau ein Index aus $B$.
Dennoch man kann am Anfang keiner Obsorte einen Index zuweisen.
Deshalb wird zunächst jeder Knoten aus $A$ mit jedem Knoten aus $B$ durch eine Kante verbunden:
\[
E = A\times B = \{(x, y) \,|\,  x \in A \text{ und } y \in B\}.
\]

Am Anfang ist $M$ dementsprechend voll mit 1--en.
Bei der Erstellung der Adjazenzmatrix kann man den Vorteil nutzen, dass die jeweilige 
Liste von Nachbarn des jeden Knotens $x \in A$ nur aus 0--en und 1-en besteht, indem
 diese Liste als Bitmasken dargestellt werden kann (mehr dazu in der \nameref{sec:umsetzung}).

\begin{figure}[H]
\centering
\caption{Beide Abbildungen stellen den Graphen für das Beispiel aus der Aufgabenstellung dar.\\
Die Buchstaben stehen für die entsprechenden Obstsorten aus diesem Beispiel (s. auch \ref{example:0}).}
\begin{subfigure}[b]{.49\textwidth}
\centering
\begin{tikzpicture}
    \node[vertex] (A) {$A$};
    \node[vertex] (B) [below = 0.4cm of A] {$B$};
    \node[vertex] (Br) [below = 0.4cm of B] {$Br$};
    \node[vertex] (E) [below = 0.4cm of Br] {$E$};
    \node[vertex] (P) [below = 0.4cm of E] {$P$};
    \node[vertex] (W) [below = 0.4cm of P] {$W$};
    \node[vertex] (1) [right = 1.5cm of A] {$1$};
    \node[vertex] (2) [right = 1.5cm of B] {$2$};
    \node[vertex] (3) [right = 1.5cm of Br] {$3$};
    \node[vertex] (4) [right = 1.5cm of E] {$4$};
    \node[vertex] (5) [right = 1.5cm of P] {$5$};
    \node[vertex] (6) [right = 1.5cm of W] {$6$};
    \begin{scope}[on background layer]
        \node[draw=blue!20,fill=blue,fill opacity=0.2,fit=(A) (B) (Br) (E) (P) (W)] [label=left:A] {};
        \node[draw=red!20,fill=red,fill opacity=0.2,fit=(1) (2) (3) (4) (5) (6)] [label=right:B] {};
    \end{scope}
\end{tikzpicture}

\caption{Die entsprechenden Mengen des Graphen}
\label{fig:graph-anfang}
\end{subfigure}
\begin{subfigure}[b]{.49\textwidth}
\centering
\begin{tikzpicture}
    \node[vertex] (1) {$A$};
    \node[vertex] (2) [below = 0.4cm of 1] {$B$};
    \node[vertex] (3) [below = 0.4cm of 2] {$Br$};
    \node[vertex] (4) [below = 0.4cm of 3] {$E$};
    \node[vertex] (5) [below = 0.4cm of 4] {$P$};
    \node[vertex] (6) [below = 0.4cm of 5] {$W$};
    \node[vertex] (7) [right = 1.5cm of 1] {$1$};
    \node[vertex] (8) [right = 1.5cm of 2] {$2$};
    \node[vertex] (9) [right = 1.5cm of 3] {$3$};
    \node[vertex] (10) [right = 1.5cm of 4] {$4$};
    \node[vertex] (11) [right = 1.5cm of 5] {$5$};
    \node[vertex] (12) [right = 1.5cm of 6] {$6$};

    \path[draw,thick]
    (1) edge node {} (7)
    (1) edge node {} (8)
    (1) edge node {} (9)
    (1) edge node {} (10)
    (1) edge node {} (11)
    (1) edge node {} (12)
    (2) edge node {} (7)
    (2) edge node {} (8)
    (2) edge node {} (9)
    (2) edge node {} (10)
    (2) edge node {} (11)
    (2) edge node {} (12)
    (3) edge node {} (7)
    (3) edge node {} (8)
    (3) edge node {} (9)
    (3) edge node {} (10)
    (3) edge node {} (11)
    (3) edge node {} (12)
    (4) edge node {} (7)
    (4) edge node {} (8)
    (4) edge node {} (9)
    (4) edge node {} (10)
    (4) edge node {} (11)
    (4) edge node {} (12)
    (5) edge node {} (7)
    (5) edge node {} (8)
    (5) edge node {} (9)
    (5) edge node {} (10)
    (5) edge node {} (11)
    (5) edge node {} (12)
    (6) edge node {} (7)
    (6) edge node {} (8)
    (6) edge node {} (9)
    (6) edge node {} (10)
    (6) edge node {} (11)
    (6) edge node {} (12);
\end{tikzpicture}

\caption{Der Graph am Anfang}
\label{fig:graph-full}
\end{subfigure}
\end{figure}

Jede $i$--te Spießkombination bringt mit sich Informationen über die Obstsorten in $F_i$.
%Als $a \leadsto b$ bezeichnet man, dass $a$ den Index $b$ haben kann. 
Man kann Folgendes festellen. 

\begin{lemma} \label{lem:spiess-numbers}
Sei $K = (F, Z)$ eine Spießkombination. Für jede Obstsorte $o(x, i)$, wobei $x \in F$, gilt:
\begin{enumerate}[label={\upshape(\roman*)}]
  %\item $\forall x \in F\, \forall y \in Z: x \leadsto y$
  %\item $\nexists p \in F\, \forall q \in B \setminus Z: p \leadsto q$.
  \item $i \in Z$, \label{lem:spiess-numbers1}
  \item $i \notin B \setminus Z$. \label{lem:spiess-numbers2}
\end{enumerate}   
\end{lemma}

\begin{proof}
Nach Definition \ref{def:spiesskomb} gilt \ref{lem:spiess-numbers1}. 
Nach Axiom \ref{ax:obstsorte-index} besitzt jede Obstsorte einen einzigartigen Index $i$,
deshalb kann $i$ nicht gleichzeitig zu $Z$ und $B \setminus Z$ gehören \ref{lem:spiess-numbers2}.
\end{proof}

Aus Lemma \ref{lem:spiess-numbers} ergibt sich direkt aus eine andere Bemerkung.

\begin{lemma}\label{lem:komponente-mengen}
Sei $C = (V_c, E_c) \subseteq G$ eine Zusammenhangskomponente.
Sei $K = (F, Z)$ eine Spießkombiantion.
Falls $F \cup Z \subsetneq V_c$ gilt, dann gilt auch:
\begin{enumerate}[label={\upshape(\roman*)}]
  \item $\forall p \in F, o(p, i): i \in Z$,
  \item $\forall q \in A \cap (V_c \cap F), o(q, j): j \in B \cap (V_c \cap Z)$.
\end{enumerate}
Deshalb werden alle Kanten, die aus jedem Knoten $x \in F$ 
zu jedem Knoten $y \in B \cap (V_c \setminus Z)$ führen,\\ aus $E$ entfernt und nur die Kanten gelassen, die
zu allen $z \in Z$ führen.
\end{lemma}

\begin{proof}
\TODO{Beweis}
\end{proof}

 
Da Bitmasken für die Darstellung jeder Liste $M_i$ ($i \in A$) verwendet werden, kann
die Laufzeit bei der Analyse der jeweiligen Spießkombination optimiert werden
(mehr dazu im Teil \ref{sec:laufzeit}),
weil man für die Operation des Entfernens Logikgatter verwenden kann.

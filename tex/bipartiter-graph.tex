\subsection{Bipartiter Graph}
Man kann die beiden Mengen $A$ und $B$ zu Knoten eines bipartiten Graphen $G = (A \cup B = V, E)$ umwandeln.
Die Menge der Kanten $E$ wird im Folgenden festgelegt.
Man stellt den Graphen als eine Adjazenzmatrix $M$ der Größe $n \times n$ dar. 
Als $M_i$ bezeichnet wird die Liste der Länge $n$,
die die Beziehungen des Knotens
$i \in A$ zu jedem Knoten $j \in B$ als 1 (Kante) oder 0 (keine Kante) enthält.
Als $M_{i, j}$ bezeichnet wird die $j$--te Stelle in der $i$--ten Liste der Matrix.

Nach Axiom \ref{ax:obstsorte-index} gehört zu jeder Obstsorte aus $A$ genau ein Index aus $B$.
Dennoch kann man am Anfang keiner Obsorte einen Index zuweisen.
Deshalb wird zunächst jeder Knoten aus $A$ mit jedem Knoten aus $B$ durch eine Kante verbunden:
\[
E = A\times B = \{(x, y) \mid  x \in A \text{ und } y \in B\}.
\]

\begin{figure}[H]
\centering
\caption{Beide Abbildungen stellen den Graphen für das Beispiel aus der Aufgabenstellung dar.\\
Die Buchstaben stehen für die entsprechenden Obstsorten aus diesem Beispiel (s. auch \ref{example:0}).}
\begin{subfigure}[b]{.49\textwidth}
\centering
\begin{tikzpicture}
    \node[vertex] (A) {$A$};
    \node[vertex] (B) [below = 0.4cm of A] {$B$};
    \node[vertex] (Br) [below = 0.4cm of B] {$Br$};
    \node[vertex] (E) [below = 0.4cm of Br] {$E$};
    \node[vertex] (P) [below = 0.4cm of E] {$P$};
    \node[vertex] (W) [below = 0.4cm of P] {$W$};
    \node[vertex] (1) [right = 1.5cm of A] {$1$};
    \node[vertex] (2) [right = 1.5cm of B] {$2$};
    \node[vertex] (3) [right = 1.5cm of Br] {$3$};
    \node[vertex] (4) [right = 1.5cm of E] {$4$};
    \node[vertex] (5) [right = 1.5cm of P] {$5$};
    \node[vertex] (6) [right = 1.5cm of W] {$6$};
    \begin{scope}[on background layer]
        \node[draw=blue!20,fill=blue,fill opacity=0.2,fit=(A) (B) (Br) (E) (P) (W)] [label=left:A] {};
        \node[draw=red!20,fill=red,fill opacity=0.2,fit=(1) (2) (3) (4) (5) (6)] [label=right:B] {};
    \end{scope}
\end{tikzpicture}

\caption{Die entsprechenden Mengen des Graphen}
\label{fig:graph-anfang}
\end{subfigure}
\begin{subfigure}[b]{.49\textwidth}
\centering
\begin{tikzpicture}
    \node[vertex] (1) {$A$};
    \node[vertex] (2) [below = 0.4cm of 1] {$B$};
    \node[vertex] (3) [below = 0.4cm of 2] {$Br$};
    \node[vertex] (4) [below = 0.4cm of 3] {$E$};
    \node[vertex] (5) [below = 0.4cm of 4] {$P$};
    \node[vertex] (6) [below = 0.4cm of 5] {$W$};
    \node[vertex] (7) [right = 1.5cm of 1] {$1$};
    \node[vertex] (8) [right = 1.5cm of 2] {$2$};
    \node[vertex] (9) [right = 1.5cm of 3] {$3$};
    \node[vertex] (10) [right = 1.5cm of 4] {$4$};
    \node[vertex] (11) [right = 1.5cm of 5] {$5$};
    \node[vertex] (12) [right = 1.5cm of 6] {$6$};

    \path[draw,thick]
    (1) edge node {} (7)
    (1) edge node {} (8)
    (1) edge node {} (9)
    (1) edge node {} (10)
    (1) edge node {} (11)
    (1) edge node {} (12)
    (2) edge node {} (7)
    (2) edge node {} (8)
    (2) edge node {} (9)
    (2) edge node {} (10)
    (2) edge node {} (11)
    (2) edge node {} (12)
    (3) edge node {} (7)
    (3) edge node {} (8)
    (3) edge node {} (9)
    (3) edge node {} (10)
    (3) edge node {} (11)
    (3) edge node {} (12)
    (4) edge node {} (7)
    (4) edge node {} (8)
    (4) edge node {} (9)
    (4) edge node {} (10)
    (4) edge node {} (11)
    (4) edge node {} (12)
    (5) edge node {} (7)
    (5) edge node {} (8)
    (5) edge node {} (9)
    (5) edge node {} (10)
    (5) edge node {} (11)
    (5) edge node {} (12)
    (6) edge node {} (7)
    (6) edge node {} (8)
    (6) edge node {} (9)
    (6) edge node {} (10)
    (6) edge node {} (11)
    (6) edge node {} (12);
\end{tikzpicture}

\caption{Der Graph am Anfang}
\label{fig:graph-full}
\end{subfigure}
\end{figure}


Am Anfang ist $M$ dementsprechend voll mit Einsen.
Bei der Erstellung der Adjazenzmatrix kann man den Vorteil nutzen, dass die 
Liste der Nachbarn eines Knotens $x \in A$ nur aus Nullen und Einsen besteht, indem man
diese Liste als eine Bitmaske darstellt (mehr dazu in der \nameref{sec:umsetzung}).

Jede $i$--te Spießkombination $K_i = (F_i, Z_i)$ bringt Informationen über die Obstsorten in $F_i$.
%Als $a \leadsto b$ bezeichnet man, dass $a$ den Index $b$ haben kann. 
Man kann Folgendes festellen. 

\begin{lemma} \label{lem:spiess-numbers}
Sei $K = (F, Z)$ eine Spießkombination. Für jede Obstsorte $o(x, i)$, wobei $x \in F$, gilt:
\begin{enumerate}[label={\upshape(\roman*)}]
  %\item $\forall x \in F\, \forall y \in Z: x \leadsto y$
  %\item $\nexists p \in F\, \forall q \in B \setminus Z: p \leadsto q$.
  \item $i \in Z$, \label{lem:spiess-numbers1}
  \item $i \notin B \setminus Z$. \label{lem:spiess-numbers2}
\end{enumerate}   
\end{lemma}

\begin{proof}
Nach Definition \ref{def:spiesskomb} gilt \ref{lem:spiess-numbers1}. 
Nach Axiom \ref{ax:obstsorte-index} besitzt jede Obstsorte einen einzigartigen Index $i$,
deshalb kann $i$ nicht gleichzeitig zu $Z$ und $B \setminus Z$ gehören \ref{lem:spiess-numbers2}.
\end{proof}

Deshalb darf man alle Kanten, die aus einem Knoten $x \in F$ 
zu einem Knoten $y \in B \setminus Z$, sowie die aus einem Knoten $p \in Z$ zu
einem Knoten $q \in A \setminus F$ führen, aus $E$ entfernen.
Diese Schlussfolgerung gilt, da %der Graph $G$ am Anfang vollständig ist und
jede Kante zwischen zwei beliebigen Knoten $x \in A$ und $y \in B$ die Möglichkeit darstellt,
dass $x$ einen Index $y$ besitzen kann.
Wenn eine Teilmenge von $A$ und $B$ in Form einer Spießkombination ausgegliedert wird,
schrumpft die Anzahl an möglichen Zuweisungen zwischen jedem $x$ und jedem $y$. 


\begin{definition}[Zusammenhangskomponente]\label{def:komponente}
Ein ungerichteter Graph $\mathcal{G} = (\mathcal{V}, \mathcal{E})$ heißt zusammenhängend, wenn es von jedem Knoten $u$ zu jedem anderen Knoten $v$ mindestens einen Pfad gibt.
Ein maximaler zusammenhängender Teilgraph eines ungerichteten Graphen $\mathcal{G}$ heißt \textbf{Zusammenhangskomponente} $C = (V_c \subseteq \mathcal{V}, E_c \subseteq \mathcal{E})$ von $\mathcal{G}$. 
\end{definition}


\noindent Aus \cref{lem:spiess-numbers} ergibt sich direkt auch eine andere Beobachtung.

\begin{korollar}\label{kor:komponente-mengen}
Sei $C = (L_c \cup R_c, E_c)$ eine Zusammenhangskomponente in $G$.
Sei $K = (F, Z)$ eine Spießkombiantion.
%Falls $F \subsetneq L_c$ gilt, dann:
Falls $F \subseteq L_c$ gilt, dann gilt für jede Obstsorte $o(x, i)$, wobei $x \in F$:
\begin{enumerate}[label={\upshape(\roman*)}]
	\item $i \in Z$,
	\item $i \notin R_c \setminus Z$.
  %\item Für jede Obstsorte $o(p, i)$, wobei $p \in F$, gilt: $i \in Z \land i \notin R_c \setminus Z$,\label{lem:komponente-mengen1}
  %\item Für jede Obstsorte $o(q, j)$, wobei $q \in L_c \setminus F$, gilt: 
  %$(j \in R_c \setminus Z) \land j \notin Z$.\label{lem:komponente-mengen2}
\end{enumerate}
Deshalb werden alle Kanten, die aus jedem Knoten $x \in F$ 
zu jedem Knoten $y \in R_c \setminus Z$ führen,\\ aus $E$ entfernt.
\end{korollar}

\begin{comment}
\begin{proof}
\ref{lem:komponente-mengen1} gilt nach Defintion \ref{def:spiesskomb}, Axiom \ref{ax:obstsorte-index}
und Lemma \ref{lem:spiess-numbers}.\\
\ref{lem:komponente-mengen2} gilt aus dem Grund, dass $L_c$ und $R_c$ gleichmächtig sind.
(Sonst, könnte man nicht allen $x \in A$ einen $y \in B$ zuweisen.)
\TODO{Beweis zu Ende}
\end{proof}
\end{comment}


Deshalb darf man alle Kanten, die aus einem Knoten $x \in F$ 
zu einem Knoten $y \in R_c \setminus Z$, sowie die aus einem Knoten $p \in Z$ 
zu einem Knoten $q \in L_c \setminus F$ führen, aus $E$ entfernen.


\begin{figure}[H]
\vspace{-0.7cm}
\caption{Beide Abbildungen stellen die Adjazenzmatrix für das Beispiel aus der Aufgabenstellung dar.
Die Buchstaben in der ersten Spalte stehen für die entsprechenden Obstsorten
und die Zahlen in der ersten Zeile stehen für die Indizes aus demselben Beispiel (s. auch \ref{example:0}).\\
Auf der Abb. \ref{fig:matrix-danach} stehen $bn$ und $br$ für die entsprechenden Bitmasken.}
\begin{subfigure}[b]{.39\textwidth}
\centering
\begin{tabular}{>{\itshape}l|c|c|c|c|c|c|}
 & 6 & 5 & 4 & 3 & 2 & 1 \\ \hline
A & 0 & 1 & 1 &0 & 0 & 1 \\ \hline 
B & 0 & 1 & 1 &0 & 0 & 1 \\ \hline 
Br & 0 & 1 & 1 &0 & 0 & 1 \\ \hline 
E & 1 & 0 & 0 & 1 & 1 & 0 \\ \hline 
P & 1 & 0 & 0 & 1 & 1 & 0 \\ \hline 
W & 1 & 0 & 0 & 1 & 1 & 0 \\ \hline 
\end{tabular}
\caption{$M$ vor der neuen Spießkombination}
\label{fig:matrix-anfang}
\end{subfigure}
\begin{subfigure}[b]{.59\textwidth}
\vspace{0.25cm}
\begin{tabular}{lll}
Spießkombination: & F =& \{Banane, Pflaume, Weintraube\} \\
 				  & Z =& \{3, 5, 6\}
\end{tabular}
\centering

\begin{tabular}{>{\itshape}l|c|c|c|c|c|c|}
 & 6 & 5 & 4 & 3 & 2 & 1 \\ \hline
\cellcolor{lightblue}bn & 1 & 1 & 0 & 1 & 0 & 0 \\ \hline
\cellcolor{lightred}br & 0 & 0 & 1 & 0 & 1 & 1 \\ \hline
\end{tabular}\\
\vspace{0.5cm}
\begin{tabular}{>{\itshape}l|c|c|c|c|c|c|}
 & 6 & 5 & 4 & 3 & 2 & 1 \\ \hline
\cellcolor{lightred}A & 0 & {\color{red} 0} & 1 &0 & 0 & 1 \\ \hline 
\cellcolor{lightblue}B & 0 & 1 & {\color{red} 0} &0 & 0 & {\color{red} 0} \\ \hline 
\cellcolor{lightred}Br & 0 & {\color{red} 0} & 1 &0 & 0 & 1 \\ \hline 
\cellcolor{lightred}E & {\color{red} 0} & 0 & 0 & {\color{red} 0} & 1 & 0 \\ \hline 
\cellcolor{lightblue}P & 1 & 0 & 0 & 1 & {\color{red} 0} & 0 \\ \hline 
\cellcolor{lightblue}W & 1 & 0 & 0 & 1 & {\color{red} 0} & 0 \\ \hline
\end{tabular}
\caption{$M$ nach der Verarbeitung der beschriebenen Spießkombination.}
\label{fig:matrix-danach}
\end{subfigure}
\end{figure}

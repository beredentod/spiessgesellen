\subsection{Formulierung des Problems}
\begin{axiom}\label{ax:obstsorte-index}
Jeder \textbf{Obstsorte} wird genau ein einzigartiger natürlicher Index zugewiesen.\\
Man schreibt: $o(x, i)$ --- eine Obstsorte $x$ besitzt einen Index $i$.
\end{axiom}

Gegeben sind eine Menge von $n$ Obstsorten $A$ und eine Menge von $n$ ganzen Zahlen
$B = \{1, 2, ..., n\}$, zu der die Indizes der Obstsorten aus $A$ gehören.

\begin{definition}[Spießkombination]\label{def:spiesskomb}
Als eine \textbf{Spießkombination} $K = (F, Z)$ bezeichnet man eine Veknüpfung von zwei Mengen 
$F \subseteq A$ und $Z$, wobei $Z = \{i \in B \,|\, \forall x \in F : o(x, i)\}.$
\end{definition}

Gegeben sind auch $m$ Spießkombinationen, wobei jede $i$--te Spießkombination
aus einer Menge von Obstsorten $F_i \subseteq A$ und einer Menge der Zahlen $Z_i \subseteq B$ besteht. 
Nach der Definition \ref{def:spiesskomb} besteht die Menge $Z_i$ nur aus
den in $B$ enthaltenen Indizes, die zu den Obstsorten in $F_i$ gehören, deshalb haben auch die beiden Mengen
$F_i$ und $Z_i$ dieselbe Anzahl an Elementen.\\
Außerdem gegeben ist auch eine \textit{\textbf{Wunschliste}} $W \subseteq A$.\\

Die Aufgabe ist, zu entscheiden,
ob die Menge der Indizes der in $W$ enthaltenen Obstsorten $W' \subseteq B$ anhand der $m$ 
Spießkombinationen eindeutig bestimmt werden kann. Falls ja, soll sie auch ausgegebn werden.\\

In den folgenden Überlegungen wird angenommen, dass das Axiom \ref{ax:obstsorte-index} für alle
Obstsorten in der Eingabe gilt.
Es ist aber möglich, dass die Spießkombination in einer Eingabe diesem Axiom nicht folgen, das heißt,
es an einer Stelle einen Widersprcuh gibt.
Laut der Aufgabenstellung ist ein solcher Fall nicht ausgeschlossen. 
Um diesen Fall zu verhindern, muss man die Korrektheit der Eingabe überprüfen. 
Mehr dazu folgt im Teil \ref{sec:korrektheit-eingabe}.

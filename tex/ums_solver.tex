\subsection{Klasse \ttt{Solver}}
Die Matrix $M$ wird als ein \ttt{vector} von \ttt{bitset} dargestellt.
Dazu muss man erwähnen, dass ein \ttt{bitset} in C++ eine feste Länge besitzen muss.
Dazu wurde die maximale Größe von $n$ eingegeben, also 26.
Das müsste ggf. im Program selbst umgestellt werden,
falls man eine größere Datei einlesen möchte.\\
Die feste Länge ist auch der Grund dafür, dass die Bitmaske $br$ im Teil \ref{sec:logik}
auf folgende Weise definiert wird:
\[
br := \neg(bn) \land bf.
\]
Im Fall, wenn man mit einer Bimaske einer festen Größe operiert, würde 
eine einfache Negation der Bitmaske $bn$ nicht hinreichen.\\

Die Obstsorten werden als Strings eingelesen, aber in der Methode \ttt{readFile()}
wird jeder Obstsorte ein interner Index\footnote{Nummerierung ab 0.} zugewiesen
und in weiteren Operationen im Programm
werden die Obstsorten als einfache Integers behandelt. 
Es werden dazu zwei Maps festgelegt: \ttt{fruit2ID} und \ttt{ID2Fruit}, 
in denen die Obstsorten und die entsprechenden internen Indizes gespeichert sind.\\

Jede Spießkombination wird als \ttt{pair<set<int>, set<int>}, also ein Tupel entsprechend
aus der Menge der internen Indizes der Obstsorten und der Menge der Indizes aus der Aufgabenstellung.
Alle Spießkombinationen werden in einem \ttt{vector} gespiechert, der \ttt{infos} heißt.\\

In \ttt{wishes} werden als ein \ttt{set} von Integers werden die internen Indizes der gewünschten 
Obstsorten, also der Elemente der Menge $W$, gespeichert.
Analog werden in \ttt{result}, also ebenfalls ein \ttt{set} von Integers, die Indizes der gewünschten 
Obstsorten, also die Elemente der Menge $W'$ hinzugefügt.\\ 

Der \ttt{vector}, der \ttt{used} heißt, wird verwendet, um die benutzten internen Indizes der Obstsorten,
wie auch die Indizes der Obstsorten zu markieren, die im Graphen verwendet werden,
da es in einzigen Textdateien ein größeres $n$ gibt als die Mächtigkeit der entsprechenden Menge $A$. 
Diese Markierung zeigt sich bei der Prüfung auf korrekte Eingabe hilfreich.\\

Die Methode \ttt{analyzeInfo()} nimmt als Argument eine Spießkombination. 
In der Methode werden die drei Bitmasken $bn, br, bf$ erstellt.
Um die Laufzeit zu optimieren, wird durch die Menge der internen Indizes der Obstsorten \ttt{fruits}
aus dieser Spießkombination gleichzeitig mit den Obstsorten aus der Matrix iteriert.
Da die Menge \ttt{fruits} vorsortiert ist, müssen wir nicht bei jeder Obstsorte in der Matrix
prüfen, ob sie sich in \ttt{fruits} befindet, um die Entscheidung zu treffen, welche der beiden
Bitmasken anzuwenden.\\
Nachdem alle $m$ Spießkombinationen verarbeitet wurden, werden in der Methode \ttt{analyzeAllInfos()}
alle übrigen 1--Beziehungen in der Adjazenzmatrix
als Kanten in den Graphen \ttt{G} der Klasse \ttt{Graph} kopiert.\\

\TODO{\ttt{checkCoherence()}}

Nachdem die Korrektheit der Eingabe geprüft wurde, kann festgestellt werden, ob $W'$ eindeutig
bestimmt werden kann. Dazu dient die Methode \ttt{checkResult()}.
\TODO{\ttt{checkResult()}}

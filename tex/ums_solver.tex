\subsection{Klasse \ttt{Solver}}
Diese Klasse vererbt die Klasse \ttt{Hashing}.

Die Matrix $M$ wird als ein \ttt{vector} von \ttt{bitset} dargestellt.
Dazu muss man erwähnen, dass ein \ttt{bitset} in C++ eine feste Länge besitzen muss.
Dazu wurde die maximale Größe von $n$ eingegeben, also 26 (s. oben die Größe der Hashtabelle).\\
Die feste Länge ist auch der Grund dafür, dass die Bitmaske $br$ im \cref{sec:logik}
auf folgende Weise definiert wird:
\[
br := \neg(bn) \land bf.
\]
Im Fall, wenn man mit einer Bitmaske einer festen Größe operiert, würde 
eine einfache Negation der Bitmaske $bn$ nicht ausreichen. Wenn $n < 26$, dann enthält 
eine Bitmaske bei einer Negation $26 - n$ zusätzliche Einsen. Aus Gründen der Übersichtlichkeit 
und der möglichen Weiterentwicklung des Programms möchte man sie zu Nullen umwandeln.


%Die Obstsorten werden als Strings eingelesen, aber in der Methode \ttt{readFile()}
%wird jeder Obstsorte ein interner Index\footnote{Nummerierung ab 0.} zugewiesen
%und in weiteren Operationen im Programm
%werden die Obstsorten als einfache Integers behandelt. 
%Es werden dazu zwei Maps festgelegt: \ttt{fruit2ID} und \ttt{ID2Fruit}, 
%in denen die Obstsorten und die entsprechenden internen Indizes gespeichert sind.

%Jede Spießkombination wird als \ttt{pair<set<int>, set<int>} dargestellt, also als ein Tupel entsprechend
%aus der Menge der internen Indizes der Obstsorten und der Menge der Indizes der Obstsorten (aus der Textdatei).
%Alle Spießkombinationen werden in einem \ttt{vector} gespiechert, der \ttt{infos} heißt.

%In \ttt{wishes} werden als ein \ttt{set} von Integers die internen Indizes der gewünschten 
%Obstsorten, also der Elemente der Menge $W$, gespeichert.
%Analog werden in \ttt{result}, ebenfalls in ein \ttt{set} von Integers, die Indizes der gewünschten 
%Obstsorten, also die Elemente der Menge $W'$, hinzugefügt.

Die Methode \ttt{analyzeInfo()} nimmt als Parameter eine Spießkombination. 
In der Methode werden die drei Bitmasken $bn, br, bf$ erstellt.
Um die Laufzeit zu optimieren, wird durch die Menge der internen Indizes der Obstsorten 
dieser Spießkombination \ttt{fruits} gleichzeitig mit den Bitmasken aus der Matrix iteriert.
Da die Menge \ttt{fruits} vorsortiert ist, müssen wir nicht bei jeder Obstsorte in der Matrix
prüfen, ob sie sich in \ttt{fruits} befindet, um die Entscheidung zu treffen, welche der beiden
Bitmasken anzuwenden ist.\\
Nachdem alle $m$ Spießkombinationen verarbeitet wurden, werden in der Methode \ttt{analyzeAllInfos()}
alle übrigen 1--Beziehungen in der Adjazenzmatrix
als Kanten in den Graphen \ttt{G} der Klasse \ttt{Graph} kopiert. 
Die jeweiligen Einsen in \ttt{matrix} werden mithilfe der eingebauten Funktion \ttt{\_Find\_next()}
gefunden.

Die Methode \ttt{checkCoherence()} prüft, ob die Eingabe dem Axiom \ref{ax:obstsorte-index} folgt.
Dennoch gibt es Beispiele, in denen $n$ größer ist als die Anzahl der verwendeten Elementen in den
Spießkombinationen und der Wunschliste.
Bei der Zuweisung der internen Indizes jeder Obstsorte werden die verwendeten Obstsorten und Indizes
in \ttt{used} markiert. Diesen Vorteil können wir nutzen, um die Kardinalität
jedes Knotens in \ttt{G} zu überprüfen: Es wird geprüft, ob zwei Nachbarn dieselbe Kardinalität besitzen.
Wir nehmen dazu die Methode \ttt{getFirstNeighbor()} der Klasse \ttt{Graph} und vergleichen die
Kardinalitäten für jeden Knoten und seinen ersten (beliebigen) Nachbarn. 
Alle Knoten, die in \ttt{used} nicht markiert wurden, werden übersprungen.
Falls bei mindestens einem Knoten die Kardinalitäten nicht übereinstimmen, wird \ttt{false} ausgegeben und
die Eingabe im gegebenen Beispiel ist fehlerhaft.
Ansonsten wird \ttt{true} ausgegeben.

Nachdem die Korrektheit der Eingabe geprüft wurde, kann festgestellt werden, ob $W'$ eindeutig
bestimmt werden kann. Dazu dient die Methode \ttt{checkResult()}.\\
Es werden ein \ttt{vector todo} der Länge $n$, ein \ttt{vector ready} der Länge $n$ und 
ein \ttt{set multip} erstellt.
\ttt{todo} ist die Liste $\bar{W}$. \ttt{ready} ist die Liste $\bar{R}$.\\
Es wird zunächst über die Menge der Wünsche \ttt{wishes} iteriert. 
Falls ein Knoten \ttt{x} (der interne Index einer Obstsorte) in \ttt{G}
die Kardinalität 1 besitzt, wird sein einzelner
Nachbar in \ttt{result}, also die Menge $W'$, hinzugefügt.
Im sonstigen Fall wird \ttt{x} in \ttt{multip} hinzugefügt und die Stelle \ttt{x} in \ttt{todo}
wird mit 1 markiert.\\
Dann wird geprüft, ob die Menge \ttt{multip} überhaupt irgendwelche Elemente enthält.
Falls nicht, gibt die ganze Funktion an dieser Stelle \ttt{true} zurück.\\
Sonst wird eine boolsche Variable \ttt{solv} erstellt, die für die Existenz einer Lösung steht.
Am Anfang nimmt sie \ttt{true} als Wert. 
Dazu wird auch eine Liste von Mengen \ttt{problems} erstellt, die dazu da ist,
die Obstsorten einer Zusammenhangskomponente, die eine nicht gewünschte Obstsorte enthält, zu speichern.\\
Danach wird über die Menge \ttt{multip} iteriert. 
Für jedes Element \ttt{x} aus dieser Menge wird eine boolsche Variable \ttt{prob} erstellt, die 
anzeigt, ob mindestens eine Obstsorte zu der Komponente gehört, die nicht gewünscht ist.
Am Anfang hat sie den Wert \ttt{false}.\\
Es wird zunächst geprüft, ob an der Stelle \ttt{x} in \ttt{ready} eine 0 steht, d.h.,
ob der Knoten zu einer Zusammenhangskomponente gehört, die noch nicht bearbeitet wurde.
Falls ja, dann werden zwei Listen \ttt{setB} und \ttt{setA} erstellt,
die der Liste der Nachbarn von \ttt{x} und der Liste der Nachbarn von einem Nachbarn von \ttt{x}
entsprechen.
Es wird durch die Menge \ttt{setA} iteriert.\\
Falls an der Stelle eines internen Index
einer Obstsorte in \ttt{todo} keine 1 steht, wird \ttt{solv = false} und \ttt{prob = true} gesetzt.
Dies bedeutet, es gibt eine nicht gewünschte Obstsorte in der Komponente.\\
Sonst wird die Stelle des internen Index dieser Obstsorte in \ttt{ready} mit 1 markiert.\\
Danach wird geprüft, ob \ttt{prob == true}. Falls \ttt{prob == false} gilt, wird die Menge \ttt{setB}
in \ttt{result} hinzugefügt. Sonst wird die Menge \ttt{setA} in \ttt{problems} hinzugefügt, um sie 
danach als Nachricht für den Nutzer vorzustellen, dass 
die Menge $W'$ aufgrund von bestimmten Obstsorten nicht eindeutig festgelegt werden kann.\\
Am Ende, nach der Schleife über \ttt{multip}, wird geprüft, ob es eine Lösung zur Aufgabe für eine
Eingabe gibt: Es wird gepüft, ob \ttt{solv == true}. Falls ja, dann wird \ttt{true} zurückgegeben.
Sonst wird eine Meldung ausgegeben, die alle Zusammenhangskomponenten beinhaltet, 
die eine nicht gewünschte Obstsorte enthalten, und die nicht gewünschten Obstsorten werden
ebenfalls angezeigt.

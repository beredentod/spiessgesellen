\newpage
\subsection{Beispiel 9}\label{example:9}
Textdatei: \texttt{spiesse9.txt}\\
Besonderheit: Ein Beispiel für den Fall \ref{probleme2} $\rightarrow$ \ref{fall2}, s. Teil \ref{sec:korrektheit-eingabe}.
\begin{verbatim}
5
Apfel Erdbeere Banane
2
2 3
Banane Clementine
4 5
Banane Dattel
\end{verbatim}

\noindent
Wünsche: \framebox{Apfel, Erdbeere, Banane}
\vspace{0.25cm}

\noindent
\framebox{Error: Es gibt Fehler in der Eingabedatei.}
\vspace{0.25cm}

Die erste Spießkombination legt fest, dass nur Banane und Clementine die Indizes 2 und 3 besitzen dürfen.
Allerdings sollen die Indizes 4 und 5 nach der zweiten Spießkombination den Obstsorten Banane und Dattel 
gehören. Die Obstsorte Banane darf keine zwei unterschiedliche Indizes besitzen.
Deshalb kommt es zu einem Widerspruch --- das Axiom \ref{ax:obstsorte-index} ist verletzt.
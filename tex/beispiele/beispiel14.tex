\subsection{Beispiel 14}\label{example:14}
Textdatei: \texttt{spiesse14.txt}\\
Besonderheit: Ein Beispiel für den Fall \ref{probleme1} $\rightarrow$ \ref{fall2}, s. Teil \ref{sec:korrektheit-eingabe}.
\begin{verbatim}
6
Apfel Clementine Feige
2
1 4 5
Apfel Banane Dattel
4 5 6
Dattel Erdbeere Feige
\end{verbatim}

\noindent
Wünsche: \framebox{Apfel, Clementine, Feige}
\vspace{0.25cm}

\noindent
\framebox{Error: Es gibt Fehler in der Eingabedatei.}
\vspace{0.25cm}

Die erste Spießkombination legt fest, dass die Obstsorten Apfel, Banane, Dattel
jeweils einen der Indizes $\{1, 4, 5\}$ haben können. Dennoch die zweite Spießkombination 
legt im Widerspruch zur ersten fest, dass die Indizes $\{4, 5, 6\}$ jeweils einer der
Obstsorten Dattel, Erdbeere, Feige gehören können. In diesem Fall decken sich 4 und Dattel
in beiden Spießkombinationen, aber 5 kann keiner Obstsorte zugeordnet werden.
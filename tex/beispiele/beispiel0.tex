\subsection{Beispiel 0 (Aufgabenstellung)}\label{example:0}
Textdatei: \texttt{spiesse0.txt}\\
\TODO{nheme Bezug auf die Aufgabenstellung}
\noindent
\framebox{Apfel, Brombeere, Weintraube}\\

\noindent
\framebox{1, 3, 4}\\

Auf Papier kann man das Beispiel auf folgende Weise lösen.
Lassen wir alle 

\begin{figure}[H]
\centering
\begin{adjustbox}{minipage=\linewidth,scale=0.85}
\begin{subfigure}[t]{.24\textwidth}
\centering
\begin{tikzpicture}
    \node[vertex] (1) {$A$};
    \node[vertex] (2) [below = 0.4cm of 1] {$B$};
    \node[vertex] (3) [below = 0.4cm of 2] {$Br$};
    \node[vertex] (4) [below = 0.4cm of 3] {$E$};
    \node[vertex] (5) [below = 0.4cm of 4] {$P$};
    \node[vertex] (6) [below = 0.4cm of 5] {$W$};
    \node[vertex] (7) [right = 1.5cm of 1] {$1$};
    \node[vertex] (8) [right = 1.5cm of 2] {$2$};
    \node[vertex] (9) [right = 1.5cm of 3] {$3$};
    \node[vertex] (10) [right = 1.5cm of 4] {$4$};
    \node[vertex] (11) [right = 1.5cm of 5] {$5$};
    \node[vertex] (12) [right = 1.5cm of 6] {$6$};

    \path[draw,thick]
    (1) edge node {} (7)
    (1) edge node {} (8)
    (1) edge node {} (9)
    (1) edge node {} (10)
    (1) edge node {} (11)
    (1) edge node {} (12)
    (2) edge node {} (7)
    (2) edge node {} (8)
    (2) edge node {} (9)
    (2) edge node {} (10)
    (2) edge node {} (11)
    (2) edge node {} (12)
    (3) edge node {} (7)
    (3) edge node {} (8)
    (3) edge node {} (9)
    (3) edge node {} (10)
    (3) edge node {} (11)
    (3) edge node {} (12)
    (4) edge node {} (7)
    (4) edge node {} (8)
    (4) edge node {} (9)
    (4) edge node {} (10)
    (4) edge node {} (11)
    (4) edge node {} (12)
    (5) edge node {} (7)
    (5) edge node {} (8)
    (5) edge node {} (9)
    (5) edge node {} (10)
    (5) edge node {} (11)
    (5) edge node {} (12)
    (6) edge node {} (7)
    (6) edge node {} (8)
    (6) edge node {} (9)
    (6) edge node {} (10)
    (6) edge node {} (11)
    (6) edge node {} (12);
\end{tikzpicture}

\end{subfigure}\hfill
\begin{subfigure}[t]{.24\textwidth}
\centering
\begin{tikzpicture}
    \node[vertex] (A) {$A$};
    \node[vertex] (B) [below = 0.4cm of A] {$B$};
    \node[vertex] (Br) [below = 0.4cm of B] {$Br$};
    \node[vertex] (E) [below = 0.4cm of Br] {$E$};
    \node[vertex] (P) [below = 0.4cm of E] {$P$};
    \node[vertex] (W) [below = 0.4cm of P] {$W$};
    \node[vertex] (1) [right = 1.5cm of A] {$1$};
    \node[vertex] (2) [right = 1.5cm of B] {$2$};
    \node[vertex] (3) [right = 1.5cm of Br] {$3$};
    \node[vertex] (4) [right = 1.5cm of E] {$4$};
    \node[vertex] (5) [right = 1.5cm of P] {$5$};
    \node[vertex] (6) [right = 1.5cm of W] {$6$};

    \path[draw,thick]
    (A) edge node {} (1)
    (A) edge node {} (4)
    (A) edge node {} (5)
    (B) edge node {} (1)
    (B) edge node {} (4)
    (B) edge node {} (5)
    (Br) edge node {} (1)
    (Br) edge node {} (4)
    (Br) edge node {} (5)
    (E) edge node {} (2)
    (E) edge node {} (3)
    (E) edge node {} (6)
    (P) edge node {} (2)
    (P) edge node {} (3)
    (P) edge node {} (6)
    (W) edge node {} (2)
    (W) edge node {} (3)
    (W) edge node {} (6);
\end{tikzpicture}
\end{subfigure}
\begin{subfigure}[t]{.24\textwidth}
\centering
\input{./tex/tikz/2.spiess.tex}
\end{subfigure}\hfill
\begin{subfigure}[t]{.24\textwidth}
\centering
\begin{tikzpicture}
    \node[vertex,text=white,fill=black!30!green] (A) {$A$};
    \node[vertex] (B) [below = 0.4cm of A] {$B$};
    \node[vertex,text=white,fill=black!30!green] (Br) [below = 0.4cm of B] {$Br$};
    \node[vertex] (E) [below = 0.4cm of Br] {$E$};
    \node[vertex] (P) [below = 0.4cm of E] {$P$};
    \node[vertex,text=white,fill=black!30!green] (W) [below = 0.4cm of P] {$W$};
    \node[vertex,text=white,fill=black!5!blue] (1) [right = 1.5cm of A] {$1$};
    \node[vertex] (2) [right = 1.5cm of B] {$2$};
    \node[vertex,text=white,fill=black!5!blue] (3) [right = 1.5cm of Br] {$3$};
    \node[vertex,text=white,fill=black!5!blue] (4) [right = 1.5cm of E] {$4$};
    \node[vertex] (5) [right = 1.5cm of P] {$5$};
    \node[vertex] (6) [right = 1.5cm of W] {$6$};

    \path[draw,thick]
    (A) edge node {} (1)
    (A) edge node {} (4)
    (B) edge[line width=2pt] node {} (5)
    (Br) edge node {} (1)
    (Br) edge node {} (4)
    (E) edge[line width=2pt] node {} (2)
    (P) edge[line width=2pt] node {} (6)
    (W) edge[line width=2pt] node {} (3);
\end{tikzpicture}
\end{subfigure}
\end{adjustbox}

\end{figure}



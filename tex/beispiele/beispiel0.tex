\subsection{Beispiel 0 (Aufgabenstellung --- Teil a)}\label{example:0}
Textdatei: \texttt{spiesse0.txt}\\
\noindent
\framebox{Apfel, Brombeere, Weintraube}\\

\noindent
\framebox{1, 3, 4}\\

Auf Papier kann man das Beispiel auf folgende Weise lösen.
Am Anfang weiß man nichts über die Obstsorten in $A$ und die Indizes in $B$.
Wir zeichnen deshalb einen vollständigen, bipartiten Graphen $G$ (Abb. \ref{fig:example0-1}).
Jede Kante steht für eine mögliche Zuweisung eines Index einer Obstsorte.\\

Wir analysieren die 1. Spießkombination:
\minibox[frame]{$F_1 = \{\text{Apfel, Banane, Brombeere}\}, Z_1 = \{1, 4, 5\}$}.\\
Wir stellen fest, dass Apfel, Banane und Brombeere jeweils keine Indizes 2, 3, 6 haben können,
weil jeder dieser Obstsorten ein Index aus der Menge $Z_1$ zugewiesen wird.
Gleichzeitig merken wir, dass Erdbeere, Pflaume und Weintraube jeweils keinen der Indizes
1, 4, 5 besitzen können, weil diese ausschließlich den Obstsorten aus der Menge $F_1$ zugewiesen werden.
Somit stellen wir fest, dass die Anzahl der möglichen Zuweisungen schrumpft.
Deshalb dürfen wir alle Kanten in $G$ zwischen allen $x \in F_1$ und allen $y \in B \setminus Z_1$,
sowie zwischen allen $p \in A \setminus F_1$ und allen $q \in Z_1$ entfernen (Abb. \ref{fig:example0-2}).\\

Wir analysieren die 2. Spießkombination:
\minibox[frame]{$F_2 = \{\text{Banane, Pflaume, Weintraube}\}, Z_2 = \{3, 5, 6\}$}.\\
Wir stellen fest, dass Banane, Pflaume, Weintraube jeweils keinen der Indizes 1, 2, 4 haben können,
weil jeder dieser Obstsorten ein Index aus der Menge $Z_2$ zugewiesen wird. 
Wir entfernen alle Kanten in $G$ zwischen allen $x \in F_2$ und allen $y \in B \setminus Z_2$.
Unter anderem wurden die Kanten (Banane, 1) und\\ (Banane, 4) entfernt.
Wir stellen fest, dass die Kardinlität des Knotens „Banane“ 1 beträgt --- er ist nur mit dem Knoten 5 verbunden.
Ebenfalls ist der Knoten 5 nur mit dem Knoten „Banane“ verbunden.
Dies bedeutet, dass es nur eine einzige Möglichkeit gibt, diesen Knoten mit einem Index zu verbinden.
Somit wurde der Index von Banane gefunden. Wir kennen schon eine Obstsorte: $o(\text{Banane}, 5)$.\\
Wir stellen auch fest, dass Apfel, Brombeere und Erdbeere jeweils keinen der Indizes 3, 5, 6 haben können,
da sie nur den Obstsorten aus $F_2$ zugewiesen werden dürfen.
Insbesondere wissen wir schon, dass 5 mit Banane verbunden wird.
Deshalb dürfen wir alle Kanten in $G$ zwischen allen $p \in A \setminus F_2$ und allen $q \in Z_2$ entfernen.
Wir bemerken, dass die Kardinlität des Knotens „Erdbeere“ nun 1 beträgt, der nur mit dem Knoten 2 verbunden ist.
Ebenfalls ist der Knoten 2 mit keinem anderen verbunden. So steht fest:\\ $o$(Erdbeere, 2).
Auf der Abbildung \ref{fig:example0-3} wird der Graph $G$ nach der Verarbeitung der 2. Spießkombination
dargestellt. Die fetten Kanten zeigen an, dass die zwei Endknoten bereits verbunden sind, also, dass 
diesen Obstsorten ihre Indizes zugewiesen wurden.\\

Wir analysieren die 3. Spießkombination:
\minibox[frame]{$F_3 = \{\text{Apfel, Brombeere, Erdbeere}\}, Z_3 = \{1, 2, 4\}$}.\\
An dieser Stelle wurde die Zuweisung für Erdbeere bereits gefunden.
Die Knoten „Apfel“ und „Brombeere“ besitzen keine Kanten mehr als die Kanten, die sie jeweils mit 1 und 4 verbinden. 
Ebenfalls wurde der Index 5 Banane zugewiesen.
Die Knoten „Pflaume“ und „Weintraube“ sind jeweils nur mit 3 und 6 verbunden.
So können wir keine der übrigen Kanten zwischen irgendwelchen zwei Knoten in $G$ entfernen.\\

Wir analysieren die 4. Spießkombination:
\minibox[frame]{$F_4 = \{\text{Erdbeere, Pflaume}\}, Z_4 = \{2, 6\}$}.\\
In der Menge $F_4$ tritt Erdbeere auf, deren Index bereits gefunden wurde.
Somit wissen wir, dass der Index von Pflaume 6 sein muss. 
So entdecken wir eine neue Zuweisung: $o$(Pflaume, 6).
Wir können deshalb alle übrigen Kanten zwischen Pflaume und allen anderen Knoten entfernen.
So bleibt der Knoten „Weintraube“ mit nur einer Kante übrig.
Der einzelne Nachbar von diesem Knoten ist 3. So entdecken wir wieder eine neue Zuweisung: $o$(Weintraube, 3).\\
Wir stellen auch fest, dass keine Kanten mehr entfernt werden können.
Es bleibt immer noch ein Paar von Indizes und ein Paar von Obstsorten ohne eindeutige Zuweisung:
\{Apfel, Brombeere\} und $\{1, 4\}$.\\

An dieser Stelle schauen wir die Wunschliste an: \framebox{Apfel, Brombeere, Weintraube}.\\
Der Index von Weintraube ist erfolgreich gefunden, aber die Indizes der übrigen Obstsorten nicht.
Allerdings soll die Lösung der Aufgabe eine Menge an Indizes der gewünschten Obstsorten sein ---
es müssen keine konkreten Zuweisungen ausgegeben werden. 
Dies wurde erfolgreich gefunden, da die Indizes 1 und 4 nur Apfel oder Brombeere gehören können, weil
keine anderen Kanten aus den Knoten 1 und 4 führen.
Auf der Abbildung \ref{fig:example0-4} wurden alle gefundenen Zuweisungen durch fette Kanten dargestellt
und alle Wünsche mit ihren Indizes wurden entsprechend \colorbox{black!30!green}{\textcolor{white}{grün}} und \colorbox{black!5!blue}{\textcolor{white}{blau}} markiert.

\begin{figure}[H]
\centering
\begin{adjustbox}{minipage=\linewidth,scale=0.85}
\begin{subfigure}[t]{.24\textwidth}
\centering
\begin{tikzpicture}
    \node[vertex] (1) {$A$};
    \node[vertex] (2) [below = 0.4cm of 1] {$B$};
    \node[vertex] (3) [below = 0.4cm of 2] {$Br$};
    \node[vertex] (4) [below = 0.4cm of 3] {$E$};
    \node[vertex] (5) [below = 0.4cm of 4] {$P$};
    \node[vertex] (6) [below = 0.4cm of 5] {$W$};
    \node[vertex] (7) [right = 1.5cm of 1] {$1$};
    \node[vertex] (8) [right = 1.5cm of 2] {$2$};
    \node[vertex] (9) [right = 1.5cm of 3] {$3$};
    \node[vertex] (10) [right = 1.5cm of 4] {$4$};
    \node[vertex] (11) [right = 1.5cm of 5] {$5$};
    \node[vertex] (12) [right = 1.5cm of 6] {$6$};

    \path[draw,thick]
    (1) edge node {} (7)
    (1) edge node {} (8)
    (1) edge node {} (9)
    (1) edge node {} (10)
    (1) edge node {} (11)
    (1) edge node {} (12)
    (2) edge node {} (7)
    (2) edge node {} (8)
    (2) edge node {} (9)
    (2) edge node {} (10)
    (2) edge node {} (11)
    (2) edge node {} (12)
    (3) edge node {} (7)
    (3) edge node {} (8)
    (3) edge node {} (9)
    (3) edge node {} (10)
    (3) edge node {} (11)
    (3) edge node {} (12)
    (4) edge node {} (7)
    (4) edge node {} (8)
    (4) edge node {} (9)
    (4) edge node {} (10)
    (4) edge node {} (11)
    (4) edge node {} (12)
    (5) edge node {} (7)
    (5) edge node {} (8)
    (5) edge node {} (9)
    (5) edge node {} (10)
    (5) edge node {} (11)
    (5) edge node {} (12)
    (6) edge node {} (7)
    (6) edge node {} (8)
    (6) edge node {} (9)
    (6) edge node {} (10)
    (6) edge node {} (11)
    (6) edge node {} (12);
\end{tikzpicture}

\caption{}
\label{fig:example0-1}
\end{subfigure}\hfill
\begin{subfigure}[t]{.24\textwidth}
\centering
\begin{tikzpicture}
    \node[vertex] (A) {$A$};
    \node[vertex] (B) [below = 0.4cm of A] {$B$};
    \node[vertex] (Br) [below = 0.4cm of B] {$Br$};
    \node[vertex] (E) [below = 0.4cm of Br] {$E$};
    \node[vertex] (P) [below = 0.4cm of E] {$P$};
    \node[vertex] (W) [below = 0.4cm of P] {$W$};
    \node[vertex] (1) [right = 1.5cm of A] {$1$};
    \node[vertex] (2) [right = 1.5cm of B] {$2$};
    \node[vertex] (3) [right = 1.5cm of Br] {$3$};
    \node[vertex] (4) [right = 1.5cm of E] {$4$};
    \node[vertex] (5) [right = 1.5cm of P] {$5$};
    \node[vertex] (6) [right = 1.5cm of W] {$6$};

    \path[draw,thick]
    (A) edge node {} (1)
    (A) edge node {} (4)
    (A) edge node {} (5)
    (B) edge node {} (1)
    (B) edge node {} (4)
    (B) edge node {} (5)
    (Br) edge node {} (1)
    (Br) edge node {} (4)
    (Br) edge node {} (5)
    (E) edge node {} (2)
    (E) edge node {} (3)
    (E) edge node {} (6)
    (P) edge node {} (2)
    (P) edge node {} (3)
    (P) edge node {} (6)
    (W) edge node {} (2)
    (W) edge node {} (3)
    (W) edge node {} (6);
\end{tikzpicture}
\caption{}
\label{fig:example0-2}
\end{subfigure}
\begin{subfigure}[t]{.24\textwidth}
\centering
\input{./tex/tikz/2.spiess.tex}
\caption{}
\label{fig:example0-3}
\end{subfigure}\hfill
\begin{subfigure}[t]{.24\textwidth}
\centering
\begin{tikzpicture}
    \node[vertex,text=white,fill=black!30!green] (A) {$A$};
    \node[vertex] (B) [below = 0.4cm of A] {$B$};
    \node[vertex,text=white,fill=black!30!green] (Br) [below = 0.4cm of B] {$Br$};
    \node[vertex] (E) [below = 0.4cm of Br] {$E$};
    \node[vertex] (P) [below = 0.4cm of E] {$P$};
    \node[vertex,text=white,fill=black!30!green] (W) [below = 0.4cm of P] {$W$};
    \node[vertex,text=white,fill=black!5!blue] (1) [right = 1.5cm of A] {$1$};
    \node[vertex] (2) [right = 1.5cm of B] {$2$};
    \node[vertex,text=white,fill=black!5!blue] (3) [right = 1.5cm of Br] {$3$};
    \node[vertex,text=white,fill=black!5!blue] (4) [right = 1.5cm of E] {$4$};
    \node[vertex] (5) [right = 1.5cm of P] {$5$};
    \node[vertex] (6) [right = 1.5cm of W] {$6$};

    \path[draw,thick]
    (A) edge node {} (1)
    (A) edge node {} (4)
    (B) edge[line width=2pt] node {} (5)
    (Br) edge node {} (1)
    (Br) edge node {} (4)
    (E) edge[line width=2pt] node {} (2)
    (P) edge[line width=2pt] node {} (6)
    (W) edge[line width=2pt] node {} (3);
\end{tikzpicture}
\caption{}
\label{fig:example0-4}
\end{subfigure}
\end{adjustbox}
\caption{}
\label{fig:example0}
\end{figure}



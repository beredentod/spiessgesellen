\subsection{Logik}\label{sec:logik}

Betrachten wir eine Spießkombination $s = (F_s, Z_s)$.
%die aus den Mengen $F_s \subseteq A$ und $Z_s \subseteq B$ besteht. 
Wir erstellen 3 Bitmasken $bf, bn$ und $br$ jeweils der Länge $n$.
Die Bitmaske $bf$ besteht aus $n$ 1--en.
In der Maske $bn$ stehen die 1--Bits an allen Stellen, die den Indizes in $Z_s$ entsprechen.
Die Bitmaske $br$ wird auf folgende Weise definiert:
\[
br := \neg(bn) \land bf.
\]
\noindent So können wir auf allen Listen $M_i$, wobei $i \in F_s$, die AND--Operation mit der Maske $bn$ 
durchführen:
\[
M_i := M_i \land bn.
\]
Analog führen wir die AND--Operation mit der Maske $br$ auf allen Listen $M_j$,
wobei $j \in A \setminus F_s$, durch:
\[
M_j := M_j \land br.
\]

\begin{figure}[H]
\vspace{-0.7cm}
\caption{Beide Abbildungen stellen die Adjazenzmatrix für das Beispiel aus der Aufgabenstellung dar.
Die Buchstaben in der ersten Spalte stehen für die entsprechenden Obstsorten
und die Zahlen in der ersten Zeile stehen für die Indizes aus demselben Beispiel (s. auch \ref{example:0}).\\
Auf der Abb. \ref{fig:matrix-danach} stehen $bn$ und $br$ für die entsprechenden Bitmasken.}
\begin{subfigure}[b]{.39\textwidth}
\centering
\begin{tabular}{>{\itshape}l|c|c|c|c|c|c|}
 & 6 & 5 & 4 & 3 & 2 & 1 \\ \hline
A & 0 & 1 & 1 &0 & 0 & 1 \\ \hline 
B & 0 & 1 & 1 &0 & 0 & 1 \\ \hline 
Br & 0 & 1 & 1 &0 & 0 & 1 \\ \hline 
E & 1 & 0 & 0 & 1 & 1 & 0 \\ \hline 
P & 1 & 0 & 0 & 1 & 1 & 0 \\ \hline 
W & 1 & 0 & 0 & 1 & 1 & 0 \\ \hline 
\end{tabular}
\caption{$M$ vor der neuen Spießkombination}
\label{fig:matrix-anfang}
\end{subfigure}
\begin{subfigure}[b]{.59\textwidth}
\vspace{0.25cm}
\begin{tabular}{lll}
Spießkombination: & F =&\{Banane, Pflaume, Weintraube\} \\
 & Z =&\{3, 5, 6\} \\
\end{tabular}\\
\centering

\begin{tabular}{>{\itshape}l|c|c|c|c|c|c|}
 & 6 & 5 & 4 & 3 & 2 & 1 \\ \hline
\cellcolor{lightblue}bn & 1 & 1 & 0 & 1 & 0 & 0 \\ \hline
\cellcolor{lightred}br & 0 & 0 & 1 & 0 & 1 & 1 \\ \hline
\end{tabular}\\
\vspace{0.5cm}
\begin{tabular}{>{\itshape}l|c|c|c|c|c|c|}
 & 6 & 5 & 4 & 3 & 2 & 1 \\ \hline
\cellcolor{lightred}A & 0 & {\color{red} 0} & 1 &0 & 0 & 1 \\ \hline 
\cellcolor{lightblue}B & 0 & 1 & {\color{red} 0} &0 & 0 & {\color{red} 0} \\ \hline 
\cellcolor{lightred}Br & 0 & {\color{red} 0} & 1 &0 & 0 & 1 \\ \hline 
\cellcolor{lightred}E & {\color{red} 0} & 0 & 0 & {\color{red} 0} & 1 & 0 \\ \hline 
\cellcolor{lightblue}P & 1 & 0 & 0 & 1 & {\color{red} 0} & 0 \\ \hline 
\cellcolor{lightblue}W & 1 & 0 & 0 & 1 & {\color{red} 0} & 0 \\ \hline
\end{tabular}
\caption{$M$ nach der Verarbeitung der beschriebenen Spießkombination.}
\label{fig:matrix-danach}
\end{subfigure}
\end{figure}

Auf der obigen Abbildung werden \colorbox{lightblue}{blau} und \colorbox{lightred}{rot} die entsprechenden
Listen gekennzeichnet, auf denen die AND--Operation mit der entsprechenden Bitmaske durchgefüht wurde.
{\color{red} Rot} werden die Bits gekennzeichnet, die sich nach der Verarbeitung der Spießkombination veränderten.\\

Was die beschriebenen Operationen verursachen, wird anhand der folgenden Fallunterscheidung erläutert.
\begin{enumerate}
  \item Falls es sich um einen Knoten $x \in F_s$ handelt, betrachten wir dazu die entsprechende
  Liste $M_x$ und einen Knoten $y \in B$.
  \begin{enumerate}
   \item Falls der Knoten $y$ zu $Z_s$ gehört, aber an der Stelle $M_{x,y}$ 0 steht, bleibt es auch 0.
   %Allerdings ergibt sich laut Lemma \ref {lem:spiess-numbers} ein Widerspruch.
   %Später wird dieser Widerspruch bei der Prüfung der Korrektheit der Eingabe entdeckt.
   \item Falls der Knoten $y$ zu $Z_s$ gehört und an der Stelle $M_{x,y}$ 1 steht, bleibt es auch 1.
   \item Falls der Knoten $y$ zu $Z_s$ nicht gehört und an der Stelle $M_{x,y}$ 0 steht, bleibt es auch 0.
   \item Falls der Knoten $y$ zu $Z_s$ nicht gehört, aber an der Stelle $M_{x,y}$ 1 steht, 
    wird die Stelle $M_{x,y}$ zu 0.
  \end{enumerate}
  \item Falls es sich um einen Knoten $x \in A \setminus F_s$ handelt, betrachten wir dazu die entsprechende
  Liste $M_x$ und einen Knoten $y \in B$.
  \begin{enumerate}
    \item Falls der Knoten $y$ zu $Z_s$ nicht gehört, aber an der Stelle $M_{x,y}$ 0 steht, bleibt es auch 0.
    %Allerdings ergibt sich laut Lemma \ref {lem:spiess-numbers} ein Widerspruch.
    %Später wird dieser Widerspruch bei der Prüfung der Korrektheit der Eingabe entdeckt.
    \item Falls der Knoten $y$ zu nicht $Z_s$ gehört und an der Stelle $M_{x,y}$ 1 steht, bleibt es auch 1.
    \item Falls der Knoten $y$ zu $Z_s$ gehört, aber an der Stelle $M_{x,y}$ 1, 
      wird die Stelle $M_{x,y}$ zu 0.
    \item Falls der Knoten $y$ zu $Z_s$ gehört und an der Stelle $M_{x,y}$ 0 steht, bleibt es auch 0.
  \end{enumerate}
\end{enumerate}



\begin{comment}
\begin{lemma}\label{lem:neue-komponenten}
Sei $C = (V_c, E_c) \subseteq G$ eine Zusammenhangskomponente.
Sei $K = (F, Z)$ eine Spießkombiantion.
Wir betrachten, was mit $G$ nach der Verarbeitung von $K$ passiert.
\begin{enumerate}[label={\upshape(\roman*)}]
  \item Falls gilt: $F \cup Z = V_c$, dann entsteht keine neue Zusammenhangskomponente in $G$. \label{lem:neue-komponenten1}
  \item Falls gilt: $F \cup Z \subsetneq V_c$,%\, \land\, F \cup Z \not\subset V \setminus V_c$, 
  dann entstehen zwei neue Zusammenhangskomponenten in $G$ --- $C$ wird in zwei Komponenten gespalten.\label{lem:neue-komponenten2}
  \item Seien $C_1, C_2, ..., C_k \subset G$ voneinander unterschiedliche Zusammenhangskomponenten.\\
  Falls $F \cup Z$ aus mehreren Teilmengen aus $C_1, C_2, ..., C_k$ besteht, gelten für jede Komponente $C_i$ ebenfalls \ref{lem:neue-komponenten1} und \ref{lem:neue-komponenten2}.
\end{enumerate}

\end{lemma}

\begin{proof}
\TODO{was machen wir mit dem Beweis?}
Die Beweise für die entsprechenden Punkte:
\begin{enumerate}[label={\upshape(\roman*)}]
  \item Nach der Definition einer Zusammenhangskomponente gilt: $\nexists x \in V_c: x \in V \setminus V_c$.
  Bei Bearbeiteung von $K$ werden nach Lemma \ref{lem:spiess-numbers} alle Kanten zwischen 
  $x \in V \setminus V_c$ und $y \in V_c$ aus $E$ entfernt.
  Deshalb werden bei so einer Spießkombination $K$ keine Kanten entfernt.
  \item blablabla
  %Laut dieser Bedingung gilt: $\exists x \in A \cap V_c: x \notin F$ und
  %$\exists y \in B \cap V_c: y \notin Z$.\\
\end{enumerate}
\end{proof}
\end{comment}

\begin{lemma}\label{lem:komponente-complete}
Sei $C = (V_c, E_c)$ eine beliebige Zusammenhangskomponente in $G$. 
Dann bildet $C$ nach Verarbeitung jeder $k$--ten Spießkombination
selbt einen vollständigen, bipartiten Graphen.
\end{lemma}

\begin{proof}
Diese Aussage kann durch die vollständige Induktion für jedes $k \in \mathbb{N}$ bewiesen werden.\\
\noindent
\tbf{Induktionsanfang}: 
Die beiden Mengen $A$ und $B$ sind gleichmächtig und ganz am Anfang ist $G$ vollständig.
Sei die erste Spießkombination $K_1 = (F_1, Z_1)$, wobei $F_1 \neq A$. (Falls $F_1 = A$,
dann gilt die Aussage sofort für $k = 1$.)
Nach der Verarbeitung von $K_1$ 
entstehen zwei Zusammenhangskomponenten: $C_1 \cup C_2 = G$, wobei o.B.d.A $C_1 = F \cup Z$.
Dann sind $C_1 \cap A$ und $C_1 \cap B$ nach Definition \ref{def:spiesskomb} auch gleichmächtig. 
Ebenfalls sind dann $C_2 \cap A$ und $C_2 \cap B$ gleichmächtig.
Nach \ref{lem:spiess-numbers} gilt, dass alle Kanten zwischen $C_1$ und $C_2$
aus $E$ entfernt wurden, aber alle innerhalb von $C_1$ und innerhalb von $C_2$
beibehalten wurden.
Dies bedeutet, dass die Komponenten $C_1$ und $C_2$ selbst vollständige, bipartite Graphen sind.\\
Damit ist die Aussage für $k = 1$ bewiesen und der Induktionsanfang erledigt.\\

\noindent
\tbf{Induktionsschritt}: Es gelte die Aussage, also die \tbf{Induktionsannahme},
für $k \in \mathbb{N}$, d.h., es gelte,
dass jede Zusammenhangskomponente in $G$ nach Verarbeitung von $k$ Spießkombinationen selbst
einen vollständigen, bipartiten Graphen bildet.\\

Zu zeigen ist die Aussage für $k + 1$, also, dass jede Zusammenhangskomponente in $G$ nach Verarbeitung
von $k + 1$ Spießkombinationen selbst einen vollständigen, bipartiten Graphen bildet.\\

Sei $K_i = (F_i, Z_i)$ die $k+1$--te Spießkombination.
Zu untersuchen ist die folgende Fallunterscheidung:
\begin{enumerate}[label={\upshape(\roman*)}]
  \item Sei $D = (V_D, E_D)$ eine Zusammenhangskomponente in $G$. Sei $F_i \cup Z_i = V_D$.
  Da alle Knoten der Spießkombination sich mit allen Knoten von $D$ decken,
  können, nach Lemma \ref{lem:komponente-mengen}, keine Kanten aus $E$ entfernt werden, deshalb
  entstet keine neue Zusammenhangskomponente, also ist jede Zusammenhangskomponente
  nach der Induktionsannahme ein vollständiger, bipartiter Graph.\label{lem:komponente-complete1} 
  %Damit ist der Induktionsschritt für diesen Fall vollzogen und die Behauptung gilt für jedes
  %$k \in \mathbb{N}$.
  \item Sei $D = (V_D, E_D)$ eine Zusammenhangskomponente in $G$. Sei $F_i \cup Z_i \subsetneq V_D
  \land (F_i \cup Z_i) \not\subset (A \cup B)$, also $F_i \cup Z_i$ gehört nur zu einer
  Zusammenhangskomponente in $G$.
  $D$ ist laut Induktionsannahme selbt ein vollständiger, bipartiter Graph.
  Nach Lemma \ref{lem:komponente-mengen} werden alle Kanten zwischen 
  allen $x \in F_i$ und allen $y \in V_D \cap Z_i$ entfernt.
  So entstehen zwei neue Zusammenhangskomponenten:
  $C_1 = F_i \cup Z_i$ und $C_2 = V_D \setminus (F_i \cup Z_i)$, die ebenfalls selbt 
  vollständige, bipartite Grpahen sind.
  Jede andere Zusammenhangskomponente in $G$ ist
  nach der Induktionsannahme ein vollständiger, bipartiter Graph.\label{lem:komponente-complete2} 
  \item Sei $1 \leqslant p \leqslant n$ beliebig, aber fest.
  Seien $C_1, C_2, ..., C_p$ untereinander unterschiedliche Zusammnhangskomponenten in $G$.
  Gehöre $(F_i \cup Z_i)$ zu mehreren Komponenten $C_p, ..., C_q$.
  Dann gilt für jede Zusammenhangskomponente $C_i$ entweder \ref{lem:komponente-complete1} 
  oder \ref{lem:komponente-complete2}, abhängig davon, ob $C_i$ vollständig zu $F_i \cup Z_i$
  gehört oder nur zum Teil. Das bedeutet, entweder entsteht keine neue Zusammenhangskomponente 
  \ref{lem:komponente-complete1} oder $C_i$ wird in zwei neue Zusammenhangskomponenten gespalten
  \ref{lem:komponente-complete2}.
\end{enumerate}

Da alle mögliche untersucht wurden, ist der Induktionsschritt vollzogen und die Behauptung gilt für jedes
$k \in \mathbb{N}$.
\end{proof}

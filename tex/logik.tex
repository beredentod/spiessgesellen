\subsection{Logik}\label{sec:logik}
Da Bitmasken für die Darstellung der Listen $M_i$ ($i \in A$) verwendet werden, kann
die Laufzeit bei der Verarbeitung der jeweiligen Spießkombination optimiert werden
(mehr dazu im Teil \nameref{sec:laufzeit}),
weil man für die Operation des Entfernens Logikgatter verwenden kann.\\

Betrachten wir eine Spießkombination $s = (F_s, Z_s)$.
%die aus den Mengen $F_s \subseteq A$ und $Z_s \subseteq B$ besteht. 
Wir erstellen 3 Bitmasken $bf, bn$ und $br$ jeweils der Länge $n$.
Die Bitmaske $bf$ besteht aus $n$ Einsen.
In der Maske $bn$ stehen die 1--Bits an allen Stellen, die den Indizes in $Z_s$ entsprechen.
Die Bitmaske $br$ wird auf folgende Weise definiert (mehr dazu in der \nameref{sec:umsetzung}):
\[
br := \neg(bn) \land bf.
\]
\noindent So können wir auf allen Listen $M_i$, wobei $i \in F_s$, die AND--Operation mit der Maske $bn$ 
durchführen:
\[
M_i := M_i \land bn.
\]
Analog führen wir die AND--Operation mit der Maske $br$ auf allen Listen $M_j$,
wobei $j \in A \setminus F_s$, durch:
\[
M_j := M_j \land br.
\]

\begin{figure}[H]
\vspace{-0.7cm}
\caption{Beide Abbildungen stellen die Adjazenzmatrix für das Beispiel aus der Aufgabenstellung dar.
Die Buchstaben in der ersten Spalte stehen für die entsprechenden Obstsorten
und die Zahlen in der ersten Zeile stehen für die Indizes aus demselben Beispiel (s. auch \ref{example:0}).\\
Auf der Abb. \ref{fig:matrix-danach} stehen $bn$ und $br$ für die entsprechenden Bitmasken.}
\begin{subfigure}[b]{.39\textwidth}
\centering
\begin{tabular}{>{\itshape}l|c|c|c|c|c|c|}
 & 6 & 5 & 4 & 3 & 2 & 1 \\ \hline
A & 0 & 1 & 1 &0 & 0 & 1 \\ \hline 
B & 0 & 1 & 1 &0 & 0 & 1 \\ \hline 
Br & 0 & 1 & 1 &0 & 0 & 1 \\ \hline 
E & 1 & 0 & 0 & 1 & 1 & 0 \\ \hline 
P & 1 & 0 & 0 & 1 & 1 & 0 \\ \hline 
W & 1 & 0 & 0 & 1 & 1 & 0 \\ \hline 
\end{tabular}
\caption{$M$ vor der neuen Spießkombination}
\label{fig:matrix-anfang}
\end{subfigure}
\begin{subfigure}[b]{.59\textwidth}
\vspace{0.25cm}
\begin{tabular}{lll}
Spießkombination: & F =&\{Banane, Pflaume, Weintraube\} \\
 & Z =&\{3, 5, 6\} \\
\end{tabular}\\
\centering

\begin{tabular}{>{\itshape}l|c|c|c|c|c|c|}
 & 6 & 5 & 4 & 3 & 2 & 1 \\ \hline
\cellcolor{lightblue}bn & 1 & 1 & 0 & 1 & 0 & 0 \\ \hline
\cellcolor{lightred}br & 0 & 0 & 1 & 0 & 1 & 1 \\ \hline
\end{tabular}\\
\vspace{0.5cm}
\begin{tabular}{>{\itshape}l|c|c|c|c|c|c|}
 & 6 & 5 & 4 & 3 & 2 & 1 \\ \hline
\cellcolor{lightred}A & 0 & {\color{red} 0} & 1 &0 & 0 & 1 \\ \hline 
\cellcolor{lightblue}B & 0 & 1 & {\color{red} 0} &0 & 0 & {\color{red} 0} \\ \hline 
\cellcolor{lightred}Br & 0 & {\color{red} 0} & 1 &0 & 0 & 1 \\ \hline 
\cellcolor{lightred}E & {\color{red} 0} & 0 & 0 & {\color{red} 0} & 1 & 0 \\ \hline 
\cellcolor{lightblue}P & 1 & 0 & 0 & 1 & {\color{red} 0} & 0 \\ \hline 
\cellcolor{lightblue}W & 1 & 0 & 0 & 1 & {\color{red} 0} & 0 \\ \hline
\end{tabular}
\caption{$M$ nach der Verarbeitung der beschriebenen Spießkombination.}
\label{fig:matrix-danach}
\end{subfigure}
\end{figure}


Auf der obigen Abbildung werden \colorbox{lightblue}{blau} und \colorbox{lightred}{rot} diejenigen
Listen gekennzeichnet, auf denen die AND--Operation mit der entsprechenden Bitmaske durchgefüht wurde.
{\color{red} Rot} werden die Bits gekennzeichnet, die sich nach der Verarbeitung der Spießkombination veränderten.\\

Was die beschriebenen Operationen verursachen, wird anhand der folgenden Fallunterscheidung erläutert.
\begin{enumerate}
  \item Falls es sich um einen Knoten $x \in F_s$ handelt, betrachten wir dazu die entsprechende
  Liste $M_x$ und einen Knoten $y \in B$.
  \begin{enumerate}
   \item Falls der Knoten $y$ zu $Z_s$ gehört, aber an der Stelle $M_{x,y}$ 0 steht, bleibt es auch 0.
   %Allerdings ergibt sich laut Lemma \ref {lem:spiess-numbers} ein Widerspruch.
   %Später wird dieser Widerspruch bei der Prüfung der Korrektheit der Eingabe entdeckt.
   \item Falls der Knoten $y$ zu $Z_s$ gehört und an der Stelle $M_{x,y}$ 1 steht, bleibt es auch 1.
   \item Falls der Knoten $y$ nicht zu $Z_s$ gehört und an der Stelle $M_{x,y}$ 0 steht, bleibt es auch 0.
   \item Falls der Knoten $y$ nicht zu $Z_s$  gehört, aber an der Stelle $M_{x,y}$ 1 steht, 
    wird die Stelle $M_{x,y}$ zu 0.
  \end{enumerate}
  \item Falls es sich um einen Knoten $x \in A \setminus F_s$ handelt, betrachten wir dazu die entsprechende
  Liste $M_x$ und einen Knoten $y \in B$.
  \begin{enumerate}
    \item Falls der Knoten $y$ nicht zu $Z_s$ gehört, aber an der Stelle $M_{x,y}$ 0 steht, bleibt es auch 0.
    %Allerdings ergibt sich laut Lemma \ref {lem:spiess-numbers} ein Widerspruch.
    %Später wird dieser Widerspruch bei der Prüfung der Korrektheit der Eingabe entdeckt.
    \item Falls der Knoten $y$ nicht zu $Z_s$ gehört und an der Stelle $M_{x,y}$ 1 steht, bleibt es auch 1.
    \item Falls der Knoten $y$ zu $Z_s$ gehört, aber an der Stelle $M_{x,y}$ 1, 
      wird die Stelle $M_{x,y}$ zu 0.
    \item Falls der Knoten $y$ zu $Z_s$ gehört und an der Stelle $M_{x,y}$ 0 steht, bleibt es auch 0.
  \end{enumerate}
\end{enumerate}




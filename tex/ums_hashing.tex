\subsection{Klasse \ttt{Hashing}}
Alle auf der BWINF--Webseite verfügbaren Beispiele beinhalten Obstsorten, die mit paarweise verschiedenen 
Buchstaben anfangen.
(Die Ausnahme ist das Beispiel aus der Aufgabenstellung, in dem zwei Obstsorten mit demselben Buchstaben beginnen.)
Man kann diese Tatsache bei der Wahl einer Datenstruktur für die Namen der Obstsorten nutzen. 
Im Programm operiert man zwar nicht auf Strings, sondern auf internen Indizes, die Integers sind. 
Allerdings muss man die Verbindungen zwischen dem Namen der Obstsorte und ihrer internen Index speichern. 
Die Anwendung vom Hashing in der Lösung zu dieser Aufgabe für die gegebenen Beispiele
erweist sich besonders vorteilhaft im Vergleich zur Verwendung von einfachen Maps, indem
man auf die Elemente in der Hashtabelle in $O(1)$ zugreifen kann (bis auf \nameref{example:16}).

Dazu verwenden wir \textit{open address hashing} mit \textit{linear probing}.
Diese Art von Hashing löst eine Kollision auf, indem eine neue Stelle für ein Element durch Probieren 
gefunden wird.\\
Am Anfang erstellen wir eine Hashtabelle der Größe 26
(im Programm ist dies eine Konstante \ttt{MAXN}, s. auch \ttt{bitset} unten),
weil diese Zahl der Größe des lateinischen Alphabets entpricht und gleichzeitg
die maximale Größe von $n$ darstellt. 
Diese Tabelle besteht aus Tupeln von \ttt{string} und \ttt{int}.
Am Anfang wird diese Tabelle mit Tupeln aus einem leeren String und dem Wert $-1$ gefüllt.

Wir kontruieren dazu eine Hashfunktion, die als Parameter einen String nimmt. 
Diese Funktion ermittelt die Stelle in der Hashtabelle für den gegebenen String.
Sei $\ell$ eine Zahl zwischen 0--25, die dem ersten Buchstaben im String
entspricht (0 -- A, 25 -- Z).
So wird mithilfe der Methode \ttt{hash()} die Stelle $p$ für einen gegebenen String 
auf folgende Weise ermittelt:
\[
p \equiv \ell\; (\textrm{mod}\; \ttt{MAXN}).
\]
Wenn an dieser Stelle in der Hashtabelle ein leerer String steht,
wird der gegebene String mithilfe der Methode \ttt{saveHash()} dort gespeichert.
Falls es zu einer Kollision kommt, wird die Stelle $p$ auf folgende Weise inkrementiert:
\[
p :\equiv (p + 1)\; (\textrm{mod}\; \ttt{MAXN}). 
\]
$p$ wird solange inkrementiert, bis eine Stelle gefunden wird, an der sich ein leerer String befindet.
Es ist gesichert, dass jeder String in der Eingabe in der Hashtabelle gespeichert wird, da
die Größe der Tabelle dem größten $n$ entspricht.
Falls man ein größeres Beispiel lösen möchte, soll man die Konstante \ttt{MAXN} im Programm ändern.

Mithilfe der Methode \ttt{getHash()} wird die Stelle eines Strings in der Hashtabelle zurückgegeben, 
falls dieser in der Tabelle enthalten ist. Sonst wird der Wert $-1$ zurückgegeben.
Diese Funktion geht analog vor, wie die Funktion \ttt{saveHash()}.
Man ermittelt zunächst die Stelle in der Hashtabelle mithilfe der Hashfunktion,
iteriert von dort und prüft, ob das iterierte Element dem gegebenen String entspricht.
Wenn der Iterator auf eine Stelle stoßt, die einen leeren String enthält,
wird sofort $-1$ zurückgegeben, da der String sich in der Hashtabelle nicht befinden kann,
weil dieser String nach dem oben beschriebenen Prinzip an der ersten leeren Stelle 
gespeichert werden sollte.

Die Methode \ttt{readFile()} lest die Daten aus der Textdatei ein. 
Nach dem Einlesen von $n$ und $m$, also der Anzahl der Obstsorten und der Anzahl der Spießkombinationen,
werden die Wünsche eingelesen.
Die Menge $W$ wird zunächst als ein \ttt{vector} von Strings mit dem Namen \ttt{wishes\_words} gespeichert.
Außerdem werden die Strings zu einem anderen \ttt{set} von Strings \ttt{all\_fruits} kopiert.
In dieser Liste werden alle Obstsorten gespeichert, die in der Eingabe auftreten. 
Danach folgt das Einlesen der Spießkombinationen.
Jede Menge $Z_i$ wird als \ttt{set} von Integers gespeichert.
Gleichzeitig werden auch die Indizes, die in $Z_i$ auftreten, in der Liste \ttt{used} markiert.

Der \ttt{vector}, der \ttt{used} heißt, wird verwendet, um die benutzten internen Indizes der Obstsorten,
wie auch die Indizes der Obstsorten zu markieren, die im Graphen verwendet werden,
da es in einigen Textdateien ein größeres $n$ gibt als die Anzahl der in den Spießkombinationen 
und der Wunschliste verwendeten Obstsorten und Indizes. 
Diese Markierung erweist sich bei der Prüfung auf korrekte Eingabe als hilfreich.

Danach wird die Menge $F_i$ als ein \ttt{vector} von Strings gespeichert und 
alle Obstsorten, die in der Eingabe auftreten, werden ebenfalls zu \ttt{all\_fruits} kopiert.
Die Menge $Z_i$ und der \ttt{vector} $F_i$ werden temporär als Tupel in den \ttt{vector}
\ttt{tempInfos} hinzugefügt.

Danach erfolgt das Hashing. \ttt{all\_fruits} wird iteriert und jedes Mal wird
die Funktion \ttt{getHash} aufgerufen, um zu prüfen, ob der iterierte String sich bereits
in der Hashtabelle befindet. Wenn nicht, wird er mithilfe der Funktion \ttt{saveHash()} 
in dieser Tabelle gespeichert.\\
Im Weiteren werden die Obstsorten nicht als String behandelt, sondern wird jeder Obstsorte
ein interner Index zugewiesen. Theoretish könnte man einfach die Stellen der Obstsorten in der Hashtabelle 
nehmen.
Da in vielen Fällen gilt: $n < 26$, können wir, anstatt diese Stellen zu nehmen, die unabhängig von $n$
zwischen 0 und 26 liegen, die internen Indizes der Obstsorten zu Zahlen zwischen 0 und $n-1$
komprimieren. Wir erstellen eine neue Variable \ttt{it} und setzen sie zu 0. Wir iterieren durch die Hashtabelle. 
Wenn an einer Stelle sich ein nichtleerer String befindet, wird dort der Wert des Integers
\ttt{it}
im Tupel gemeinsam mit diesem String gespeichert (die Hashtabelle besteht aus Tupeln von Strings und Integers). 
Folglich wird \ttt{it} inkrementiert.
Gleichzeitig wird der an dieser Stelle stehende String in einen \ttt{vector} \ttt{ID2Fruit} hinzugefügt und
der interne Index dieser Obstsorte wird in \ttt{used} markiert.
Man beachte, dass die $i$--te Stelle in \ttt{ID2Fruit} dem $i$--ten internen Index einer Obstsorte entspricht. 

Am Ende wird die Liste \ttt{wishes\_words} iteriert und an jedem iterierten String wird
die Methode \ttt{getIndex} aufgerufen, die den internen Index dieses Strings in der Hashtabelle zurückgibt.
Auf eine analoge Weise werden dann die Strings in \ttt{tempInfos} zu ihren entsprechenden
internen Indizes umgewandelt.





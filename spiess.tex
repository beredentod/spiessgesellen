\documentclass[a4paper,10pt,ngerman]{scrartcl}
\usepackage{babel}
\usepackage[T1]{fontenc}
\usepackage[utf8]{inputenc}
\usepackage{textcomp}
\usepackage[a4paper,margin=2.5cm,footskip=0.5cm]{geometry}

% Die nächsten drei Felder bitte anpassen:
\newcommand{\Aufgabe}{Aufgabe 2: Spießgesellen} % Aufgabennummer und Aufgabennamen angeben
\newcommand{\TeilnahmeId}{55628}       % Teilnahme-Id angeben
\newcommand{\Namen}{Michal Boron} % Namen der Bearbeiter/-innen dieser Aufgabe angeben
 
% Kopf- und Fußzeilen
\usepackage{scrlayer-scrpage, lastpage}
\setkomafont{pageheadfoot}{\large\textrm}
\lohead{\Aufgabe}
\rohead{Teilnahme-Id: \TeilnahmeId}
\cfoot*{\thepage{}/\pageref{LastPage}}

% Position des Titels
\usepackage{titling}
\setlength{\droptitle}{-1.0cm}
\usepackage{seqsplit}
\usepackage{verbatim}

% Für mathematische Befehle und Symbole
\usepackage{amsmath}
\usepackage{amssymb}
%\usepackage{cite}

\usepackage[backend=bibtex]{biblatex}
\addbibresource{stromrallye.bib}

\usepackage{hyperref}
\hypersetup{
    colorlinks=false,
    linkcolor=blue,
    filecolor=magenta,      
    urlcolor=cyan,
}
% Für Bilder
\usepackage{graphicx}
\usepackage[all]{xy}
\usepackage{svg}
\graphicspath{ {./images/} }

% Für Algorithmen
\usepackage{algpseudocode}
\usepackage{algorithm}
\usepackage{gensymb}
\usepackage{tikz}

% Für Quelltext
\usepackage{listings}
\usepackage{color}
\definecolor{mygreen}{rgb}{0,0.6,0}
\definecolor{mygray}{rgb}{0.5,0.5,0.5}
\definecolor{mymauve}{rgb}{0.58,0,0.82}
\lstset{
  keywordstyle=\color{blue},commentstyle=\color{mygreen},
  stringstyle=\color{mymauve},rulecolor=\color{black},
  basicstyle=\footnotesize\ttfamily,numberstyle=\tiny\color{mygray},
  captionpos=b, % sets the caption-position to bottom
  keepspaces=true, % keeps spaces in text
  numbers=left, numbersep=5pt, showspaces=false,showstringspaces=true,
  showtabs=false, stepnumber=2, tabsize=2, title=\lstname
}

% Diese beiden Pakete müssen zuletzt geladen werden
%\usepackage{hyperref} % Anklickbare Links im Dokument
\usepackage{cleveref}
\newtheorem{lemma}{Lemma}
\usepackage[backgroundcolor=lightgray]{todonotes}
\newcommand{\TODO}[1]{\todo[inline]{#1}}
\renewcommand{\bf}[1]{\textbf{#1}}

\usepackage{minibox}

% Daten für die Titelseite
\title{\textbf{\Huge\Aufgabe}}
\author{\LARGE Teilnahme-Id: \LARGE \TeilnahmeId \\\\
	    \LARGE Bearbeiter dieser Aufgabe: \\ 
	    \LARGE \Namen\\\\}
\date{\LARGE April 2021}

\begin{document}

\maketitle
\tableofcontents

\vspace{0.5cm}

\section{Lösungsidee}

\section{Umsetzung}

\section{Beispiele}


\subsection{Beispiel 0 (Aufgabenstellung)}\label{example:0}
Textdatei: \texttt{spiesse0.txt}\\

\noindent
\framebox{Apfel, Brombeere, Weintraube}\\

\noindent
\framebox{1, 3, 4}


\subsection{Beispiel 1 (BWINF)}\label{example:1}
Textdatei: \texttt{spiesse1.txt}\\

\noindent
\framebox{Clementine, Erdbeere, Grapefruit, Himbeere, Johannisbeere}\\

\noindent
\framebox{1, 2, 4, 5, 7}

\subsection{Beispiel 2 (BWINF)}\label{example:2}
Textdatei: \texttt{spiesse2.txt}\\

\noindent
\framebox{Apfel, Banane, Clementine, Himbeere, Kiwi, Litschi}\\

\noindent
\framebox{1, 5, 6, 7, 10, 11}


\subsection{Beispiel 3 (BWINF)}\label{example:3}
Textdatei: \texttt{spiesse3.txt}\\

\noindent
\framebox{Clementine, Erdbeere, Feige, Himbeere, Ingwer, Kiwi, Litschi}\\

unlösbar: Litschi ist in der Komponente mit Grapefruit. Dabei ist Grapefruit kein Wunsch.

\subsection{Beispiel 4 (BWINF)}\label{example:4}
Textdatei: \texttt{spiesse4.txt}\\

\noindent
\framebox{Apfel, Feige, Grapefruit, Ingwer, Kiwi, Nektarine, Orange, Pflaume}\\

\noindent
\framebox{2, 6, 7, 8, 9, 12, 13, 14}


\subsection{Beispiel 5 (BWINF)}\label{example:5}
Textdatei: \texttt{spiesse5.txt}\\

\noindent
\minibox[frame]{Apfel, Banane, Clementine, Dattel, Grapefruit, Himbeere, Mango, Nektarine, Orange, Pflaume,\\ Quitte, Sauerkirsche, Tamarinde}\\

\noindent
\framebox{1, 2, 3, 4, 5, 6, 9, 10, 12, 14, 16, 19, 20}


\subsection{Beispiel 6 (BWINF)}\label{example:6}
Textdatei: \texttt{spiesse6.txt}\\

\noindent
\framebox{Clementine, Erdbeere, Himbeere, Orange, Quitte, Rosine, Ugli, Vogelbeere}\\

\noindent
\framebox{4, 6, 7, 10, 11, 15, 18, 20}


\subsection{Beispiel 7 (BWINF)}\label{example:7}
Textdatei: \texttt{spiesse7.txt}\\

\noindent
\minibox[frame]{Apfel, Clementine, Dattel, Grapefruit, Mango, Sauerkirsche, Tamarinde, Ugli, Vogelbeere, Xenia,\\ Yuzu, Zitrone}\\

unlösbar: Litschi ist in der Komponente mit Apfel. Dabei ist Litschi kein Wunsch.


\section{Quellcode}
\lstinputlisting[language=C++]{spiesse.m}

\end{document}
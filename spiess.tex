\documentclass[a4paper,10pt,ngerman]{scrartcl}
\usepackage{babel}
\usepackage[T1]{fontenc}
\usepackage[utf8]{inputenc}
\usepackage{textcomp}
\usepackage[a4paper,margin=2.5cm,footskip=0.5cm]{geometry}

% Die nächsten drei Felder bitte anpassen:
\newcommand{\Aufgabe}{Aufgabe 2: Spießgesellen} % Aufgabennummer und Aufgabennamen angeben
\newcommand{\TeilnahmeId}{55628}       % Teilnahme-Id angeben
\newcommand{\Namen}{Michal Boron} % Namen der Bearbeiter/-innen dieser Aufgabe angeben
 
% Kopf- und Fußzeilen
\usepackage{scrlayer-scrpage, lastpage}
\setkomafont{pageheadfoot}{\large\textrm}
\lohead{\Aufgabe}
\rohead{Teilnahme-Id: \TeilnahmeId}
\cfoot*{\thepage{}/\pageref{LastPage}}

% Position des Titels
\usepackage{titling}
\setlength{\droptitle}{-1.0cm}
\usepackage{seqsplit}
\usepackage{verbatim}

% Für mathematische Befehle und Symbole
\usepackage{amsmath}
\usepackage{amssymb}
%\usepackage{cite}

\usepackage[backend=bibtex]{biblatex}
\addbibresource{stromrallye.bib}

\usepackage{hyperref}
\hypersetup{
    colorlinks=false,
    linkcolor=blue,
    filecolor=magenta,      
    urlcolor=cyan,
}
% Für Bilder
\usepackage{graphicx}
\usepackage[all]{xy}
\usepackage{svg}
\graphicspath{ {./images/} }

% Für Algorithmen
\usepackage{algpseudocode}
\usepackage{algorithm}
\usepackage{gensymb}
\usepackage{tikz}
\usepackage{caption}
\usepackage{subcaption}

\usepackage[backgroundcolor=lightgray]{todonotes}
\usepackage{minibox}

\usepackage{amsthm}

% Für Quelltext
\usepackage{listings}
\usepackage{color}
\definecolor{mygreen}{rgb}{0,0.6,0}
\definecolor{mygray}{rgb}{0.5,0.5,0.5}
\definecolor{mymauve}{rgb}{0.58,0,0.82}
\lstset{
  keywordstyle=\color{blue},commentstyle=\color{mygreen},
  stringstyle=\color{mymauve},rulecolor=\color{black},
  basicstyle=\footnotesize\ttfamily,numberstyle=\tiny\color{mygray},
  captionpos=b, % sets the caption-position to bottom
  keepspaces=true, % keeps spaces in text
  numbers=left, numbersep=5pt, showspaces=false,showstringspaces=true,
  showtabs=false, stepnumber=2, tabsize=2, title=\lstname
}

% Diese beiden Pakete müssen zuletzt geladen werden
%\usepackage{hyperref} % Anklickbare Links im Dokument
\usepackage{cleveref}
%\newtheorem{lemma}{Lemma}
%\newenvironment{proof}{\paragraph{Beweis:}}{\hfill$\square$}
\newtheorem{lemma}{Lemma}
%\renewcommand*{\proofname}{Solution}

\newcommand{\TODO}[1]{\todo[inline]{TODO: #1}}
\newcommand{\mb}[1]{{\color{red}[MB: #1]}}
\newcommand{\tbf}[1]{\textbf{#1}}
\newcommand{\ttt}[1]{\texttt{#1}}

\usetikzlibrary{fit,backgrounds,positioning}
\tikzset{vertex/.style={circle,draw,minimum size=0.8cm,inner sep=1pt,fill=white}}

% Daten für die Titelseite
\title{\textbf{\Huge\Aufgabe}}
\author{\LARGE Teilnahme-Id: \LARGE \TeilnahmeId \\\\
	    \LARGE Bearbeiter dieser Aufgabe: \\ 
	    \LARGE \Namen\\\\}
\date{\LARGE April 2021}

\begin{document}

\maketitle
\tableofcontents

\vspace{0.5cm}

\section{Lösungsidee}
\subsection{Aufgabenstellung}
Gegeben sind eine Menge von $n$ \textit{Obstsorten} $A$ und eine Menge von $n$ ganzen Zahlen $B = \{1, 2, ..., n\}$,
die für die Indizes der Obstsorten stehen.
Gegeben sind auch $m$ \textit{Spießkombinationen}, wobei jede $i$--te Spießkombination
aus einer Menge von Obstsorten $F_i \subseteq A$ und einer Menge der Zahlen $Z_i \subseteq B$ besteht. 
Für jedes $i$ besteht die Menge $Z_i$ nur aus
den in $B$ enthaltenen Indizes, die den Obstsorten in $F_i$ entsprechen, deshalb haben auch die beiden Mengen
$F_i$ und $Z_i$ dieselbe Anzahl an Elementen.
Außerdem gegeben ist auch eine \textit{Wunschliste} $W \subseteq A$.\\
Die Aufgabe ist, zu entscheiden,
ob die Menge der Indizes der in $W$ gegebenen Obstsorten $W' \subseteq B$ anhand der $m$ 
Spießkombinationen eindeutig bestimmt werden kann. Falls ja, soll sie auch ausgegebn werden.

\subsection{Graph}
\TODO{\\Name ändern\\Bild}

Man kann die beiden Mengen $A$ und $B$ zu Knoten eines bipartiten Graphen $G = (A \cup B, E)$ umwandeln.
Die Menge der Kanten $E$ wird im Folgenden festgelegt.
Man stellt den Graphen als eine Adjazenzmatrix $M$ der Größe $n \times n$ dar. 
Als $M_i$ bezeichne ich die Liste der Länge $n$, die die Beziehungen des Knotens $i \in A$ zu jedem Knoten 
$j \in B$ als 1 (Kante) 0 (keine Kante) darstellt.\\
Nach der Aufgabenstellung gehört jeder Obstsorte aus $A$ genau ein Index aus $B$.
Dennoch man kann am Anfang keiner Obsorte einen Index zuweisen.
Deshalb verbinden wir zunächst jeden Knoten aus $A$ mit jedem Knoten aus $B$ durch eine Kante.
Am Anfang ist $M$ dementsprechend voll mit 1--en.
Bei der Erstellung der Adjazenzmatrix können wir den Vorteil nutzen, dass die jeweilige 
Liste von Nachbarn des jeden Knotens $x \in A$ nur aus 0--en und 1-en besteht, indem
wir diese Liste als Bitmasken darstellen (mehr dazu in der \nameref{sec:umsetzung}).

\begin{figure}[H]
\centering
\caption{Beide Abbildungen stellen den Graphen für das Beispiel aus der Aufgabenstellung dar\\ 
  (s. auch \ref{example:0}).}
\begin{subfigure}[c]{.49\textwidth}
\centering
\begin{tikzpicture}
    \node[vertex] (A) {$A$};
    \node[vertex] (B) [below = 0.4cm of A] {$B$};
    \node[vertex] (Br) [below = 0.4cm of B] {$Br$};
    \node[vertex] (E) [below = 0.4cm of Br] {$E$};
    \node[vertex] (P) [below = 0.4cm of E] {$P$};
    \node[vertex] (W) [below = 0.4cm of P] {$W$};
    \node[vertex] (1) [right = 1.5cm of A] {$1$};
    \node[vertex] (2) [right = 1.5cm of B] {$2$};
    \node[vertex] (3) [right = 1.5cm of Br] {$3$};
    \node[vertex] (4) [right = 1.5cm of E] {$4$};
    \node[vertex] (5) [right = 1.5cm of P] {$5$};
    \node[vertex] (6) [right = 1.5cm of W] {$6$};
    \begin{scope}[on background layer]
        \node[draw=blue!20,fill=blue,fill opacity=0.2,fit=(A) (B) (Br) (E) (P) (W)] [label=left:A] {};
        \node[draw=red!20,fill=red,fill opacity=0.2,fit=(1) (2) (3) (4) (5) (6)] [label=right:B] {};
    \end{scope}
\end{tikzpicture}

\caption{Die entsprechenden Mengen des Graphen}
\label{fig:graph-anfang}
\end{subfigure}
\begin{subfigure}[c]{.49\textwidth}
\centering
\begin{tikzpicture}
    \node[vertex] (1) {$A$};
    \node[vertex] (2) [below = 0.4cm of 1] {$B$};
    \node[vertex] (3) [below = 0.4cm of 2] {$Br$};
    \node[vertex] (4) [below = 0.4cm of 3] {$E$};
    \node[vertex] (5) [below = 0.4cm of 4] {$P$};
    \node[vertex] (6) [below = 0.4cm of 5] {$W$};
    \node[vertex] (7) [right = 1.5cm of 1] {$1$};
    \node[vertex] (8) [right = 1.5cm of 2] {$2$};
    \node[vertex] (9) [right = 1.5cm of 3] {$3$};
    \node[vertex] (10) [right = 1.5cm of 4] {$4$};
    \node[vertex] (11) [right = 1.5cm of 5] {$5$};
    \node[vertex] (12) [right = 1.5cm of 6] {$6$};

    \path[draw,thick]
    (1) edge node {} (7)
    (1) edge node {} (8)
    (1) edge node {} (9)
    (1) edge node {} (10)
    (1) edge node {} (11)
    (1) edge node {} (12)
    (2) edge node {} (7)
    (2) edge node {} (8)
    (2) edge node {} (9)
    (2) edge node {} (10)
    (2) edge node {} (11)
    (2) edge node {} (12)
    (3) edge node {} (7)
    (3) edge node {} (8)
    (3) edge node {} (9)
    (3) edge node {} (10)
    (3) edge node {} (11)
    (3) edge node {} (12)
    (4) edge node {} (7)
    (4) edge node {} (8)
    (4) edge node {} (9)
    (4) edge node {} (10)
    (4) edge node {} (11)
    (4) edge node {} (12)
    (5) edge node {} (7)
    (5) edge node {} (8)
    (5) edge node {} (9)
    (5) edge node {} (10)
    (5) edge node {} (11)
    (5) edge node {} (12)
    (6) edge node {} (7)
    (6) edge node {} (8)
    (6) edge node {} (9)
    (6) edge node {} (10)
    (6) edge node {} (11)
    (6) edge node {} (12);
\end{tikzpicture}

\caption{Der Graph am Anfang}
\label{fig:graph-full}
\end{subfigure}
\end{figure}

Jede $i$--te \textit{Spießkombinationen} bringt uns Informationen über die Obstsorten in $F_i$.
Als $a \leadsto b$ bezeichnen wir, dass $a$ den Index $b$ haben kann. Wir können Folgendes festellen. 

\begin{lemma} \label{lem:spiess-numbers}
Für jede $i$--te Spießkombination mit $F_i \subseteq A$ und $Z_i \subseteq B$ gilt, dass
$\forall x \in F_i, \forall y \in Z_i: x \leadsto y$.\\
Es gilt gleichzeitig, dass $\nexists p \in F_i, \forall q \in B \setminus Z_i: p \leadsto q$  
\end{lemma}

\TODO{Beweis}
\begin{proof}
\end{proof}

Nach Lemma \ref{lem:spiess-numbers} dürfen wir alle Kanten, die aus jedem Knoten $x \in F_i$ 
zu einem Knoten $y \in B \setminus Z_i$ führen, aus $E$ entfernen und nur die Kanten lassen, die
zu allen $z \in Z_i$ führen.
Da wir Bitmasken für die Darstellung jeder Liste $M_i$ ($i \in A$) verwenden, können wir 
die Laufzeit bei der Analyse der jeweiligen Spießkombination optimieren,
weil ich für die Operation des Entfernens Logikgatter verwende.

\subsection{Logik}
\TODO{Veranschauung}

Betrachten wir eine Spießkombination $s$, die aus den Mengen $F_s \subseteq A$ und $Z_s \subseteq B$ besteht.\\
Wir erstellen 3 Bitmasken $bf, bn$ und $br$ jeweils der Länge $n$. Die Bitmaske $bf$ besteht aus $n$ 1--en.
In der Maske $bn$ stehen die 1--Bits an allen Stellen, die den Indizes in $Z_s$ entsprechen.
Die Bitmaske $br$ wird auf folgende Weise definiert:
\[
br := \neg(bn) \land bf.
\]
\noindent So können wir auf allen Listen $M_i$, wobei $i \in F_s$, die AND--Operation mit der Maske $bn$ 
durchführen:
\[
M_i := M_i \land bn.
\]
Analog führen wir die AND--Operation mit der Maske $br$ auf allen Listen $M_j$,
wobei $j \in A \setminus F_s$, durch:
\[
M_j := M_j \land br.
\]
Was die beschriebenen Operationen verursachen, erläutere ich anhand der folgenden Fallunterscheidung.
\begin{enumerate}
  \item Falls es sich um einen Knoten $x \in F_s$ handelt.
  \begin{enumerate}
   \item Falls ein Knoten $y$ zu $Z_s$ gehört, aber an der Stelle $y$ in $M_x$ 0 steht, ergibt sich laut Lemma 
    \ref {lem:spiess-numbers} ein Widerspruch. \mb{No i co z tego?? Dopisać} 
   \item Falls ein Knoten $y$ zu $Z_s$ gehört und an der Stelle $y$ in $M_x$ 1 steht, bleibt es auch 1.
   \item Falls ein Knoten $y$ nicht zu $Z_s$ gehört und an der Stelle $y$ in $M_x$ 0 steht, bleibt es auch 0.
   \item Falls ein Knoten $y$ nicht zu $Z_s$ gehört, aber an der Stelle $y$ in $M_x$ 1 steht, 
    wird die Stelle $y$ in $M_x$ zu 0.
  \end{enumerate}
  \item Falls es sich um einen Knoten $x \in A \setminus F_s$ handelt.
  \begin{enumerate}
    \item Falls ein Knoten $y$ nicht zu $Z_s$ gehört und an der Stelle $y$ in $M_x$ 0 steht, ergibt sich 
      laut Lemma \ref {lem:spiess-numbers} ein Widerspruch.
    \item Falls ein Knoten $y$ nicht zu $Z_s$ gehört, aber an der Stelle $y$ in $M_x$ 1 steht, wird die Stelle 
      $y$ in $M_x$ zu 0.
    \item Falls ein Knoten $y$ zu $Z_s$ gehört, aber an der Stelle $y$ in $M_x$ 1
    \item Falls ein Knoten $y$ zu $Z_s$ gehört, aber an der Stelle $y$ in $M_x$ 0
  \end{enumerate}
\end{enumerate}

\subsection{Komponenten}

\begin{figure}[h]
\centering
\begin{subfigure}[c]{.49\textwidth}
\centering
\begin{tikzpicture}
    \node[vertex] (A) {$A$};
    \node[vertex] (B) [below = 0.4cm of A] {$B$};
    \node[vertex] (Br) [below = 0.4cm of B] {$Br$};
    \node[vertex] (E) [below = 0.4cm of Br] {$E$};
    \node[vertex] (P) [below = 0.4cm of E] {$P$};
    \node[vertex] (W) [below = 0.4cm of P] {$W$};
    \node[vertex] (1) [right = 1.5cm of A] {$1$};
    \node[vertex] (2) [right = 1.5cm of B] {$2$};
    \node[vertex] (3) [right = 1.5cm of Br] {$3$};
    \node[vertex] (4) [right = 1.5cm of E] {$4$};
    \node[vertex] (5) [right = 1.5cm of P] {$5$};
    \node[vertex] (6) [right = 1.5cm of W] {$6$};

    \path[draw,thick]
    (A) edge node {} (1)
    (A) edge node {} (4)
    (B) edge node {} (5)
    (Br) edge node {} (1)
    (Br) edge node {} (4)
    (E) edge node {} (2)
    (P) edge node {} (6)
    (W) edge node {} (3);
\end{tikzpicture}

\caption{Der Graph nach der Analyse der allen Spießkombinationen}
\label{fig:graph-after-infos}
\end{subfigure}
\begin{subfigure}[c]{.49\textwidth}
\centering
\begin{tikzpicture}
    \node[vertex] (A) {$A$};
    \node[vertex] (Br) [below = 0.4cm of A] {$Br$};
    \node[vertex] (1) [right = 1.5cm of A] {$1$};
    \node[vertex] (4) [right = 1.5cm of Br] {$4$};

    \path[draw,thick]
    (A) edge node {} (1)
    (A) edge node {} (4)
    (Br) edge node {} (1)
    (Br) edge node {} (4);
\end{tikzpicture}

\caption{Die ürbige Zusammenhangskomponente}
\label{fig:component-left}
\end{subfigure}
\end{figure}

\subsection{Laufzeit}
\subsection{Prüfung auf Korrektheit der Eingabe}

\section{Umsetzung}\label{sec:umsetzung}

\section{Beispiele}

\subsection{Beispiel 0 (Aufgabenstellung)}\label{example:0}
Textdatei: \texttt{spiesse0.txt}\\

\noindent
\framebox{Apfel, Brombeere, Weintraube}\\

\noindent
\framebox{1, 3, 4}


\subsection{Beispiel 1 (BWINF)}\label{example:1}
Textdatei: \texttt{spiesse1.txt}\\

\noindent
\framebox{Clementine, Erdbeere, Grapefruit, Himbeere, Johannisbeere}\\

\noindent
\framebox{1, 2, 4, 5, 7}

\subsection{Beispiel 2 (BWINF)}\label{example:2}
Textdatei: \texttt{spiesse2.txt}\\

\noindent
\framebox{Apfel, Banane, Clementine, Himbeere, Kiwi, Litschi}\\

\noindent
\framebox{1, 5, 6, 7, 10, 11}

\subsection{Beispiel 3 (BWINF)}\label{example:3}
Textdatei: \texttt{spiesse3.txt}\\

\noindent
Wünsche: \framebox{Clementine, Erdbeere, Feige, Himbeere, Ingwer, Kiwi, Litschi}\\

\noindent
\framebox{Dieses Beispiel ist unlösbar.}
\begin{verbatim}
Für die folgenden Obstsorten konnte keine eindeutige Zuweisung gefunden werden.
Komponente: Grapefruit Litschi 
	--> Nicht auf der Wunschliste: Grapefruit 
\end{verbatim}

\subsection{Beispiel 4 (BWINF)}\label{example:4}
Textdatei: \texttt{spiesse4.txt}\\

\noindent
\framebox{Apfel, Feige, Grapefruit, Ingwer, Kiwi, Nektarine, Orange, Pflaume}\\

\noindent
\framebox{2, 6, 7, 8, 9, 12, 13, 14}

\newpage
\subsection{Beispiel 5 (BWINF)}\label{example:5}
Textdatei: \texttt{spiesse5.txt}
\vspace{0.25cm}

\noindent
Wünsche: \minibox[frame]{Apfel, Banane, Clementine, Dattel, Grapefruit, Himbeere, Mango,\\ Nektarine, Orange, Pflaume, Quitte, Sauerkirsche, Tamarinde}
\vspace{0.25cm}

\noindent
\framebox{1, 2, 3, 4, 5, 6, 9, 10, 12, 14, 16, 19, 20}

\subsection{Beispiel 6 (BWINF)}\label{example:6}
Textdatei: \texttt{spiesse6.txt}\\

\noindent
\framebox{Clementine, Erdbeere, Himbeere, Orange, Quitte, Rosine, Ugli, Vogelbeere}\\

\noindent
\framebox{4, 6, 7, 10, 11, 15, 18, 20}

\subsection{Beispiel 7 (BWINF)}\label{example:7}
Textdatei: \texttt{spiesse7.txt}\\

\noindent
\minibox[frame]{Apfel, Clementine, Dattel, Grapefruit, Mango, Sauerkirsche, Tamarinde, Ugli, Vogelbeere, Xenia,\\ Yuzu, Zitrone}\\

unlösbar: Apfel, Grapefruit und Xenia gehören zur Komponente mit Litschi. Dabei ist Litschi kein Wunsch.
Ugli gehört zur Komponente mit Banane. Dabei ist Banane kein Wunsch.


\section{Quellcode}
\lstinputlisting[language=C++]{./tex/spiesse.m}

\end{document}
\documentclass[a4paper,10pt,ngerman]{scrartcl}
\usepackage{babel}
\usepackage[T1]{fontenc}
\usepackage[utf8]{inputenc}
\usepackage{textcomp}
\usepackage[a4paper,margin=2.5cm,footskip=0.5cm]{geometry}

% Die nächsten drei Felder bitte anpassen:
\newcommand{\Aufgabe}{Aufgabe 2: Spießgesellen} % Aufgabennummer und Aufgabennamen angeben
\newcommand{\TeilnahmeId}{55628}       % Teilnahme-Id angeben
\newcommand{\Namen}{Michal Boron} % Namen der Bearbeiter/-innen dieser Aufgabe angeben
 
% Kopf- und Fußzeilen
\usepackage{scrlayer-scrpage, lastpage}
\setkomafont{pageheadfoot}{\large\textrm}
\lohead{\Aufgabe}
\rohead{Teilnahme-Id: \TeilnahmeId}
\cfoot*{\thepage{}/\pageref{LastPage}}

% Position des Titels
\usepackage{titling}
\setlength{\droptitle}{-1.0cm}
\usepackage{seqsplit}
\usepackage{verbatim}

% Für mathematische Befehle und Symbole
\usepackage{amsmath}
\usepackage{amssymb}
%\usepackage{cite}

\usepackage[backend=bibtex]{biblatex}
\addbibresource{stromrallye.bib}

\usepackage{hyperref}
\hypersetup{
    colorlinks=false,
    linkcolor=blue,
    filecolor=magenta,      
    urlcolor=cyan,
}
% Für Bilder
\usepackage{graphicx}
\usepackage[all]{xy}
\usepackage{svg}
\graphicspath{ {./images/} }

% Für Algorithmen
\usepackage{algpseudocode}
\usepackage{algorithm}
\usepackage{gensymb}
\usepackage{tikz}
\usepackage{caption}
\usepackage{subcaption}
\usepackage{array}
\usepackage{makecell}

\usepackage[backgroundcolor=lightgray]{todonotes}
\usepackage{minibox}

\usepackage{amsthm}
\usepackage{enumitem}

\usepackage[backend=bibtex]{biblatex}
\addbibresource{spiess.bib}

% Für Quelltext
\usepackage{listings}
\usepackage{color, colortbl}
\definecolor{mygreen}{rgb}{0,0.6,0}
\definecolor{mygray}{rgb}{0.5,0.5,0.5}
\definecolor{mymauve}{rgb}{0.58,0,0.82}
\definecolor{lightblue}{HTML}{cce6ff}
\definecolor{lightred}{HTML}{ffb3b3}

\lstset{
  keywordstyle=\color{blue},commentstyle=\color{mygreen},
  stringstyle=\color{mymauve},rulecolor=\color{black},
  basicstyle=\footnotesize\ttfamily,numberstyle=\tiny\color{mygray},
  captionpos=b, % sets the caption-position to bottom
  keepspaces=true, % keeps spaces in text
  numbers=left, numbersep=5pt, showspaces=false,showstringspaces=true,
  showtabs=false, stepnumber=2, tabsize=2, title=\lstname
}

% Diese beiden Pakete müssen zuletzt geladen werden
%\usepackage{hyperref} % Anklickbare Links im Dokument
\usepackage{cleveref}
%\newtheorem{lemma}{Lemma}
%\newenvironment{proof}{\paragraph{Beweis:}}{\hfill$\square$}
\newtheorem{lemma}{Lemma}
\newtheorem{definition}{Definition}
\newtheorem{satz}{Satz}
\newtheorem{axiom}{Axiom}
\newtheorem{korollar}{Korollar}
%\renewcommand*{\proofname}{Solution}

\usepackage[export]{adjustbox}

\newcommand{\TODO}[1]{\todo[inline]{TODO: #1}}
\newcommand{\mb}[1]{{\color{red}[MB: #1]}}
\newcommand{\tbf}[1]{\textbf{#1}}
\newcommand{\ttt}[1]{\texttt{#1}}

\usetikzlibrary{fit,backgrounds,positioning}
\tikzset{vertex/.style={circle,draw,minimum size=0.8cm,inner sep=1pt,fill=white}}

% Daten für die Titelseite
\title{\textbf{\Huge\Aufgabe}}
\author{\LARGE Teilnahme-Id: \LARGE \TeilnahmeId \\\\
	    \LARGE Bearbeiter dieser Aufgabe: \\ 
	    \LARGE \Namen\\\\}
\date{\LARGE April 2021}

\begin{document}

\maketitle
\tableofcontents

\section{Lösungsidee}
\subsection{Formulierung des Problems}\label{sec:formulierung}
\begin{axiom}\label{ax:obstsorte-index}
Jeder \textbf{Obstsorte} wird genau ein einzigartiger natürlicher Index zugewiesen.

Man schreibt: $o(x, i)$ --- eine Obstsorte $x$ besitzt einen Index $i$.
\end{axiom}


Gegeben sind eine Menge von $n$ Obstsorten $A$ und eine Menge von $n$ ganzen Zahlen
$B = \{1, 2, ..., n\}$, zu der die Indizes der Obstsorten aus $A$ gehören.

\begin{definition}[Spießkombination]\label{def:spiesskomb}
Als eine \textbf{Spießkombination} $K = (F, Z)$ bezeichnet man eine Veknüpfung von zwei Mengen 
$F \subseteq A$ und $Z$, wobei $Z = \{i \in B \,|\, \forall x \in F : o(x, i)\}.$
\end{definition}


Gegeben sind auch $m$ Spießkombinationen, wobei jede $i$--te Spießkombination
aus einer Menge von Obstsorten $F_i \subseteq A$ und einer Menge der Indizes $Z_i \subseteq B$ besteht. 
Nach der Definition \ref{def:spiesskomb} besteht die Menge $Z_i$ nur aus
den in $B$ enthaltenen Indizes, die zu den Obstsorten in $F_i$ gehören, deshalb sind die beiden Mengen
$F_i$ und $Z_i$ auch gleichmächtig.\\
\indent Außerdem gegeben ist auch eine \textit{\textbf{Wunschliste}} $W \subseteq A$.\\

Die Aufgabe ist ein Entscheidungsproblem. Es soll entschieden werden,
ob die Menge der Indizes der in $W$ enthaltenen Obstsorten $W' \subseteq B$ anhand der $m$ 
Spießkombinationen eindeutig bestimmt werden kann. Falls ja, soll sie auch ausgegeben werden.\\

In den folgenden Überlegungen wird angenommen, dass das Axiom \ref{ax:obstsorte-index} für alle
Obstsorten in der Eingabe gilt.
Es ist aber möglich, dass die Spießkombination in einer Eingabe diesem Axiom nicht folgen, das heißt,
es an einer Stelle einen Widerspruch gibt.
Laut der Aufgabenstellung ist ein solcher Fall nicht ausgeschlossen. 
Um diesen Fall zu verhindern, muss man die Korrektheit der Eingabe überprüfen. 
Mehr dazu folgt im Teil \ref{sec:korrektheit-eingabe}.


\subsection{Bipartiter Graph}
Man kann die beiden Mengen $A$ und $B$ zu Knoten eines bipartiten Graphen $G = (A \cup B = V, E)$ umwandeln.
Die Menge der Kanten $E$ wird im Folgenden festgelegt.
Man stellt den Graphen als eine Adjazenzmatrix $M$ der Größe $n \times n$ dar. 
Als $M_i$ bezeichnet wird die Liste der Länge $n$,
die die Beziehungen des Knotens
$i \in A$ zu jedem Knoten $j \in B$ als 1 (Kante) oder 0 (keine Kante) enthält.
Als $M_{i, j}$ bezeichnet wird die $j$--te Stelle in der $i$--ten Liste der Matrix.

Nach Axiom \ref{ax:obstsorte-index} gehört zu jeder Obstsorte aus $A$ genau ein Index aus $B$.
Dennoch kann man am Anfang keiner Obsorte einen Index zuweisen.
Deshalb wird zunächst jeder Knoten aus $A$ mit jedem Knoten aus $B$ durch eine Kante verbunden:
\[
E = A\times B = \{(x, y) \mid  x \in A \text{ und } y \in B\}.
\]

\begin{figure}[H]
\centering
\begin{subfigure}[b]{.49\textwidth}
\centering
\begin{tikzpicture}
    \node[vertex] (A) {$A$};
    \node[vertex] (B) [below = 0.4cm of A] {$B$};
    \node[vertex] (Br) [below = 0.4cm of B] {$Br$};
    \node[vertex] (E) [below = 0.4cm of Br] {$E$};
    \node[vertex] (P) [below = 0.4cm of E] {$P$};
    \node[vertex] (W) [below = 0.4cm of P] {$W$};
    \node[vertex] (1) [right = 1.5cm of A] {$1$};
    \node[vertex] (2) [right = 1.5cm of B] {$2$};
    \node[vertex] (3) [right = 1.5cm of Br] {$3$};
    \node[vertex] (4) [right = 1.5cm of E] {$4$};
    \node[vertex] (5) [right = 1.5cm of P] {$5$};
    \node[vertex] (6) [right = 1.5cm of W] {$6$};
    \begin{scope}[on background layer]
        \node[draw=blue!20,fill=blue,fill opacity=0.2,fit=(A) (B) (Br) (E) (P) (W)] [label=left:A] {};
        \node[draw=red!20,fill=red,fill opacity=0.2,fit=(1) (2) (3) (4) (5) (6)] [label=right:B] {};
    \end{scope}
\end{tikzpicture}

\caption{Die entsprechenden Mengen des Graphen}
\label{fig:graph-anfang}
\end{subfigure}
\begin{subfigure}[b]{.49\textwidth}
\centering
\begin{tikzpicture}
    \node[vertex] (1) {$A$};
    \node[vertex] (2) [below = 0.4cm of 1] {$B$};
    \node[vertex] (3) [below = 0.4cm of 2] {$Br$};
    \node[vertex] (4) [below = 0.4cm of 3] {$E$};
    \node[vertex] (5) [below = 0.4cm of 4] {$P$};
    \node[vertex] (6) [below = 0.4cm of 5] {$W$};
    \node[vertex] (7) [right = 1.5cm of 1] {$1$};
    \node[vertex] (8) [right = 1.5cm of 2] {$2$};
    \node[vertex] (9) [right = 1.5cm of 3] {$3$};
    \node[vertex] (10) [right = 1.5cm of 4] {$4$};
    \node[vertex] (11) [right = 1.5cm of 5] {$5$};
    \node[vertex] (12) [right = 1.5cm of 6] {$6$};

    \path[draw,thick]
    (1) edge node {} (7)
    (1) edge node {} (8)
    (1) edge node {} (9)
    (1) edge node {} (10)
    (1) edge node {} (11)
    (1) edge node {} (12)
    (2) edge node {} (7)
    (2) edge node {} (8)
    (2) edge node {} (9)
    (2) edge node {} (10)
    (2) edge node {} (11)
    (2) edge node {} (12)
    (3) edge node {} (7)
    (3) edge node {} (8)
    (3) edge node {} (9)
    (3) edge node {} (10)
    (3) edge node {} (11)
    (3) edge node {} (12)
    (4) edge node {} (7)
    (4) edge node {} (8)
    (4) edge node {} (9)
    (4) edge node {} (10)
    (4) edge node {} (11)
    (4) edge node {} (12)
    (5) edge node {} (7)
    (5) edge node {} (8)
    (5) edge node {} (9)
    (5) edge node {} (10)
    (5) edge node {} (11)
    (5) edge node {} (12)
    (6) edge node {} (7)
    (6) edge node {} (8)
    (6) edge node {} (9)
    (6) edge node {} (10)
    (6) edge node {} (11)
    (6) edge node {} (12);
\end{tikzpicture}

\caption{Der Graph am Anfang}
\label{fig:graph-full}
\end{subfigure}
\caption{Beide Abbildungen stellen den Graphen für das Beispiel aus der Aufgabenstellung dar.\\
Die Buchstaben stehen für die entsprechenden Obstsorten aus diesem Beispiel (s. auch \ref{example:0}).}
\end{figure}


Am Anfang ist $M$ dementsprechend voll mit Einsen.
Bei der Erstellung der Adjazenzmatrix kann man den Vorteil nutzen, dass die 
Liste der Nachbarn eines Knotens $x \in A$ nur aus Nullen und Einsen besteht, indem man
diese Liste als eine Bitmaske darstellt (mehr dazu in der \nameref{sec:umsetzung}).

Jede $i$--te Spießkombination $K_i = (F_i, Z_i)$ bringt Informationen über die Obstsorten in $F_i$.
%Als $a \leadsto b$ bezeichnet man, dass $a$ den Index $b$ haben kann. 
Man kann Folgendes festellen. 

\begin{lemma} \label{lem:spiess-numbers}
Sei $K = (F, Z)$ eine Spießkombination. Für jede Obstsorte $o(x, i)$, wobei $x \in F$, gilt:
\begin{enumerate}[label={\upshape(\roman*)}]
  %\item $\forall x \in F\, \forall y \in Z: x \leadsto y$
  %\item $\nexists p \in F\, \forall q \in B \setminus Z: p \leadsto q$.
  \item $i \in Z$, \label{lem:spiess-numbers1}
  \item $i \notin B \setminus Z$. \label{lem:spiess-numbers2}
\end{enumerate}   
\end{lemma}

\begin{proof}
Nach Definition \ref{def:spiesskomb} gilt \ref{lem:spiess-numbers1}. 
Nach Axiom \ref{ax:obstsorte-index} besitzt jede Obstsorte einen einzigartigen Index $i$,
deshalb kann $i$ nicht gleichzeitig zu $Z$ und $B \setminus Z$ gehören \ref{lem:spiess-numbers2}.
\end{proof}


\begin{definition}[Zusammenhangskomponente]\label{def:komponente}
Ein ungerichteter Graph $\mathcal{G} = (\mathcal{V}, \mathcal{E})$ heißt zusammenhängend, wenn es von jedem Knoten $u$ zu jedem anderen Knoten $v$ mindestens einen Pfad gibt.
Ein maximaler zusammenhängender Teilgraph eines ungerichteten Graphen $\mathcal{G}$ heißt \textbf{Zusammenhangskomponente} $C = (V_c \subseteq \mathcal{V}, E_c \subseteq \mathcal{E})$ von $\mathcal{G}$. 
\end{definition}


\noindent Aus Lemma \ref{lem:spiess-numbers} ergibt sich direkt auch eine andere Beobachtung.

\begin{korollar}\label{kor:komponente-mengen}
Sei $C = (L_c \cup R_c, E_c)$ eine Zusammenhangskomponente in $G$.
Sei $K = (F, Z)$ eine Spießkombiantion.
%Falls $F \subsetneq L_c$ gilt, dann:
Falls $F \subseteq L_c$ gilt, dann gilt für jede Obstsorte $o(x, i)$, wobei $x \in F$:
\begin{enumerate}[label={\upshape(\roman*)}]
	\item $i \in Z$,
	\item $i \notin R_c \setminus Z$.
  %\item Für jede Obstsorte $o(p, i)$, wobei $p \in F$, gilt: $i \in Z \land i \notin R_c \setminus Z$,\label{lem:komponente-mengen1}
  %\item Für jede Obstsorte $o(q, j)$, wobei $q \in L_c \setminus F$, gilt: 
  %$(j \in R_c \setminus Z) \land j \notin Z$.\label{lem:komponente-mengen2}
\end{enumerate}
Deshalb werden alle Kanten, die aus jedem Knoten $x \in F$ 
zu jedem Knoten $y \in R_c \setminus Z$ führen,\\ aus $E$ entfernt.
\end{korollar}

\begin{comment}
\begin{proof}
\ref{lem:komponente-mengen1} gilt nach Defintion \ref{def:spiesskomb}, Axiom \ref{ax:obstsorte-index}
und Lemma \ref{lem:spiess-numbers}.\\
\ref{lem:komponente-mengen2} gilt aus dem Grund, dass $L_c$ und $R_c$ gleichmächtig sind.
(Sonst, könnte man nicht allen $x \in A$ einen $y \in B$ zuweisen.)
\TODO{Beweis zu Ende}
\end{proof}
\end{comment}





\subsection{Logik}\label{sec:logik}

Betrachten wir eine Spießkombination $s = (F_s, Z_s)$.
%die aus den Mengen $F_s \subseteq A$ und $Z_s \subseteq B$ besteht. 
Wir erstellen 3 Bitmasken $bf, bn$ und $br$ jeweils der Länge $n$.
Die Bitmaske $bf$ besteht aus $n$ 1--en.
In der Maske $bn$ stehen die 1--Bits an allen Stellen, die den Indizes in $Z_s$ entsprechen.
Die Bitmaske $br$ wird auf folgende Weise definiert:
\[
br := \neg(bn) \land bf.
\]
\noindent So können wir auf allen Listen $M_i$, wobei $i \in F_s$, die AND--Operation mit der Maske $bn$ 
durchführen:
\[
M_i := M_i \land bn.
\]
Analog führen wir die AND--Operation mit der Maske $br$ auf allen Listen $M_j$,
wobei $j \in A \setminus F_s$, durch:
\[
M_j := M_j \land br.
\]

\begin{figure}[H]
\vspace{-0.7cm}
\caption{Beide Abbildungen stellen die Adjazenzmatrix für das Beispiel aus der Aufgabenstellung dar.
Die Buchstaben in der ersten Spalte stehen für die entsprechenden Obstsorten
und die Zahlen in der ersten Zeile stehen für die Indizes aus demselben Beispiel (s. auch \ref{example:0}).\\
Auf der Abb. \ref{fig:matrix-danach} stehen $bn$ und $br$ für die entsprechenden Bitmasken.}
\begin{subfigure}[b]{.39\textwidth}
\centering
\begin{tabular}{>{\itshape}l|c|c|c|c|c|c|}
 & 6 & 5 & 4 & 3 & 2 & 1 \\ \hline
A & 0 & 1 & 1 &0 & 0 & 1 \\ \hline 
B & 0 & 1 & 1 &0 & 0 & 1 \\ \hline 
Br & 0 & 1 & 1 &0 & 0 & 1 \\ \hline 
E & 1 & 0 & 0 & 1 & 1 & 0 \\ \hline 
P & 1 & 0 & 0 & 1 & 1 & 0 \\ \hline 
W & 1 & 0 & 0 & 1 & 1 & 0 \\ \hline 
\end{tabular}
\caption{$M$ vor der neuen Spießkombination}
\label{fig:matrix-anfang}
\end{subfigure}
\begin{subfigure}[b]{.59\textwidth}
\vspace{0.25cm}
\begin{tabular}{lll}
Spießkombination: & F =&\{Banane, Pflaume, Weintraube\} \\
 & Z =&\{3, 5, 6\} \\
\end{tabular}\\
\centering

\begin{tabular}{>{\itshape}l|c|c|c|c|c|c|}
 & 6 & 5 & 4 & 3 & 2 & 1 \\ \hline
\cellcolor{lightblue}bn & 1 & 1 & 0 & 1 & 0 & 0 \\ \hline
\cellcolor{lightred}br & 0 & 0 & 1 & 0 & 1 & 1 \\ \hline
\end{tabular}\\
\vspace{0.5cm}
\begin{tabular}{>{\itshape}l|c|c|c|c|c|c|}
 & 6 & 5 & 4 & 3 & 2 & 1 \\ \hline
\cellcolor{lightred}A & 0 & {\color{red} 0} & 1 &0 & 0 & 1 \\ \hline 
\cellcolor{lightblue}B & 0 & 1 & {\color{red} 0} &0 & 0 & {\color{red} 0} \\ \hline 
\cellcolor{lightred}Br & 0 & {\color{red} 0} & 1 &0 & 0 & 1 \\ \hline 
\cellcolor{lightred}E & {\color{red} 0} & 0 & 0 & {\color{red} 0} & 1 & 0 \\ \hline 
\cellcolor{lightblue}P & 1 & 0 & 0 & 1 & {\color{red} 0} & 0 \\ \hline 
\cellcolor{lightblue}W & 1 & 0 & 0 & 1 & {\color{red} 0} & 0 \\ \hline
\end{tabular}
\caption{$M$ nach der Verarbeitung der beschriebenen Spießkombination.}
\label{fig:matrix-danach}
\end{subfigure}
\end{figure}

Auf der obigen Abbildung werden \colorbox{lightblue}{blau} und \colorbox{lightred}{rot} die entsprechenden
Listen gekennzeichnet, auf denen die AND--Operation mit der entsprechenden Bitmaske durchgefüht wurde.
{\color{red} Rot} werden die Bits gekennzeichnet, die sich nach der Verarbeitung der Spießkombination veränderten.\\

Was die beschriebenen Operationen verursachen, wird anhand der folgenden Fallunterscheidung erläutert.
\begin{enumerate}
  \item Falls es sich um einen Knoten $x \in F_s$ handelt, betrachten wir dazu die entsprechende
  Liste $M_x$ und einen Knoten $y \in B$.
  \begin{enumerate}
   \item Falls der Knoten $y$ zu $Z_s$ gehört, aber an der Stelle $M_{x,y}$ 0 steht, bleibt es auch 0.
   %Allerdings ergibt sich laut Lemma \ref {lem:spiess-numbers} ein Widerspruch.
   %Später wird dieser Widerspruch bei der Prüfung der Korrektheit der Eingabe entdeckt.
   \item Falls der Knoten $y$ zu $Z_s$ gehört und an der Stelle $M_{x,y}$ 1 steht, bleibt es auch 1.
   \item Falls der Knoten $y$ zu $Z_s$ nicht gehört und an der Stelle $M_{x,y}$ 0 steht, bleibt es auch 0.
   \item Falls der Knoten $y$ zu $Z_s$ nicht gehört, aber an der Stelle $M_{x,y}$ 1 steht, 
    wird die Stelle $M_{x,y}$ zu 0.
  \end{enumerate}
  \item Falls es sich um einen Knoten $x \in A \setminus F_s$ handelt, betrachten wir dazu die entsprechende
  Liste $M_x$ und einen Knoten $y \in B$.
  \begin{enumerate}
    \item Falls der Knoten $y$ zu $Z_s$ nicht gehört, aber an der Stelle $M_{x,y}$ 0 steht, bleibt es auch 0.
    %Allerdings ergibt sich laut Lemma \ref {lem:spiess-numbers} ein Widerspruch.
    %Später wird dieser Widerspruch bei der Prüfung der Korrektheit der Eingabe entdeckt.
    \item Falls der Knoten $y$ zu nicht $Z_s$ gehört und an der Stelle $M_{x,y}$ 1 steht, bleibt es auch 1.
    \item Falls der Knoten $y$ zu $Z_s$ gehört, aber an der Stelle $M_{x,y}$ 1, 
      wird die Stelle $M_{x,y}$ zu 0.
    \item Falls der Knoten $y$ zu $Z_s$ gehört und an der Stelle $M_{x,y}$ 0 steht, bleibt es auch 0.
  \end{enumerate}
\end{enumerate}



\begin{comment}
\begin{lemma}\label{lem:neue-komponenten}
Sei $C = (V_c, E_c) \subseteq G$ eine Zusammenhangskomponente.
Sei $K = (F, Z)$ eine Spießkombiantion.
Wir betrachten, was mit $G$ nach der Verarbeitung von $K$ passiert.
\begin{enumerate}[label={\upshape(\roman*)}]
  \item Falls gilt: $F \cup Z = V_c$, dann entsteht keine neue Zusammenhangskomponente in $G$. \label{lem:neue-komponenten1}
  \item Falls gilt: $F \cup Z \subsetneq V_c$,%\, \land\, F \cup Z \not\subset V \setminus V_c$, 
  dann entstehen zwei neue Zusammenhangskomponenten in $G$ --- $C$ wird in zwei Komponenten gespalten.\label{lem:neue-komponenten2}
  \item Seien $C_1, C_2, ..., C_k \subset G$ voneinander unterschiedliche Zusammenhangskomponenten.\\
  Falls $F \cup Z$ aus mehreren Teilmengen aus $C_1, C_2, ..., C_k$ besteht, gelten für jede Komponente $C_i$ ebenfalls \ref{lem:neue-komponenten1} und \ref{lem:neue-komponenten2}.
\end{enumerate}

\end{lemma}

\begin{proof}
\TODO{was machen wir mit dem Beweis?}
Die Beweise für die entsprechenden Punkte:
\begin{enumerate}[label={\upshape(\roman*)}]
  \item Nach der Definition einer Zusammenhangskomponente gilt: $\nexists x \in V_c: x \in V \setminus V_c$.
  Bei Bearbeiteung von $K$ werden nach Lemma \ref{lem:spiess-numbers} alle Kanten zwischen 
  $x \in V \setminus V_c$ und $y \in V_c$ aus $E$ entfernt.
  Deshalb werden bei so einer Spießkombination $K$ keine Kanten entfernt.
  \item blablabla
  %Laut dieser Bedingung gilt: $\exists x \in A \cap V_c: x \notin F$ und
  %$\exists y \in B \cap V_c: y \notin Z$.\\
\end{enumerate}
\end{proof}
\end{comment}

\begin{lemma}\label{lem:komponente-complete}
Sei $C = (V_c, E_c)$ eine beliebige Zusammenhangskomponente in $G$. 
Dann bildet $C$ nach Verarbeitung jeder $k$--ten Spießkombination
selbt einen vollständigen, bipartiten Graphen.
\end{lemma}

\begin{proof}
Diese Aussage kann durch die vollständige Induktion für jedes $k \in \mathbb{N}$ bewiesen werden.\\
\noindent
\tbf{Induktionsanfang}: 
Die beiden Mengen $A$ und $B$ sind gleichmächtig und ganz am Anfang ist $G$ vollständig.
Sei die erste Spießkombination $K_1 = (F_1, Z_1)$, wobei $F_1 \neq A$. (Falls $F_1 = A$,
dann gilt die Aussage sofort für $k = 1$.)
Nach der Verarbeitung von $K_1$ 
entstehen zwei Zusammenhangskomponenten: $C_1 \cup C_2 = G$, wobei o.B.d.A $C_1 = F \cup Z$.
Dann sind $C_1 \cap A$ und $C_1 \cap B$ nach Definition \ref{def:spiesskomb} auch gleichmächtig. 
Ebenfalls sind dann $C_2 \cap A$ und $C_2 \cap B$ gleichmächtig.
Nach \ref{lem:spiess-numbers} gilt, dass alle Kanten zwischen $C_1$ und $C_2$
aus $E$ entfernt wurden, aber alle innerhalb von $C_1$ und innerhalb von $C_2$
beibehalten wurden.
Dies bedeutet, dass die Komponenten $C_1$ und $C_2$ selbst vollständige, bipartite Graphen sind.\\
Damit ist die Aussage für $k = 1$ bewiesen und der Induktionsanfang erledigt.\\

\noindent
\tbf{Induktionsschritt}: Es gelte die Aussage, also die \tbf{Induktionsannahme},
für $k \in \mathbb{N}$, d.h., es gelte,
dass jede Zusammenhangskomponente in $G$ nach Verarbeitung von $k$ Spießkombinationen selbst
einen vollständigen, bipartiten Graphen bildet.\\

Zu zeigen ist die Aussage für $k + 1$, also, dass jede Zusammenhangskomponente in $G$ nach Verarbeitung
von $k + 1$ Spießkombinationen selbst einen vollständigen, bipartiten Graphen bildet.\\

Sei $K_i = (F_i, Z_i)$ die $k+1$--te Spießkombination.
Zu untersuchen ist die folgende Fallunterscheidung:
\begin{enumerate}[label={\upshape(\roman*)}]
  \item Sei $D = (V_D, E_D)$ eine Zusammenhangskomponente in $G$. Sei $F_i \cup Z_i = V_D$.
  Da alle Knoten der Spießkombination sich mit allen Knoten von $D$ decken,
  können, nach Lemma \ref{lem:komponente-mengen}, keine Kanten aus $E$ entfernt werden, deshalb
  entstet keine neue Zusammenhangskomponente, also ist jede Zusammenhangskomponente
  nach der Induktionsannahme ein vollständiger, bipartiter Graph.\label{lem:komponente-complete1} 
  %Damit ist der Induktionsschritt für diesen Fall vollzogen und die Behauptung gilt für jedes
  %$k \in \mathbb{N}$.
  \item Sei $D = (V_D, E_D)$ eine Zusammenhangskomponente in $G$. Sei $F_i \cup Z_i \subsetneq V_D
  \land (F_i \cup Z_i) \not\subset (A \cup B)$, also $F_i \cup Z_i$ gehört nur zu einer
  Zusammenhangskomponente in $G$.
  $D$ ist laut Induktionsannahme selbt ein vollständiger, bipartiter Graph.
  Nach Lemma \ref{lem:komponente-mengen} werden alle Kanten zwischen 
  allen $x \in F_i$ und allen $y \in V_D \cap Z_i$ entfernt.
  So entstehen zwei neue Zusammenhangskomponenten:
  $C_1 = F_i \cup Z_i$ und $C_2 = V_D \setminus (F_i \cup Z_i)$, die ebenfalls selbt 
  vollständige, bipartite Grpahen sind.
  Jede andere Zusammenhangskomponente in $G$ ist
  nach der Induktionsannahme ein vollständiger, bipartiter Graph.\label{lem:komponente-complete2} 
  \item Sei $1 \leqslant p \leqslant n$ beliebig, aber fest.
  Seien $C_1, C_2, ..., C_p$ untereinander unterschiedliche Zusammnhangskomponenten in $G$.
  Gehöre $(F_i \cup Z_i)$ zu mehreren Komponenten $C_p, ..., C_q$.
  Dann gilt für jede Zusammenhangskomponente $C_i$ entweder \ref{lem:komponente-complete1} 
  oder \ref{lem:komponente-complete2}, abhängig davon, ob $C_i$ vollständig zu $F_i \cup Z_i$
  gehört oder nur zum Teil. Das bedeutet, entweder entsteht keine neue Zusammenhangskomponente 
  \ref{lem:komponente-complete1} oder $C_i$ wird in zwei neue Zusammenhangskomponenten gespalten
  \ref{lem:komponente-complete2}.
\end{enumerate}

Da alle mögliche untersucht wurden, ist der Induktionsschritt vollzogen und die Behauptung gilt für jedes
$k \in \mathbb{N}$.
\end{proof}


\subsection{Zusammenhangskomponenten}
Nach der Verarbeitung aller $m$ Spießkombinationen verfügen wir über den Graphen $G$,
in dem viele Kanten in $E$ entfernt wurden.
Auf diese Weise können wir schon anfangen, die Indizes der Obstsorten aus $W$ festzulegen.
Definieren wir zunächst, was generell ein \tbf{Matching} ist.

\begin{definition}[Matching]\label{def:matching}
Sei $\mathcal{G} = (\mathcal{V}, \mathcal{E})$ ein ungerichteter Graph.
Als ein \textbf{Matching} bezeichnen wir eine Teilmenge $\mathcal{S} \subseteq \mathcal{E}$,
sodass für alle $v \in \mathcal{V}$ gilt, dass höchstens eine Kante 
aus $\mathcal{S}$ inzident zu $v$ ist.\\
Wir bezeichnen einen Knoten $v \in \mathcal{V}$ als in $\mathcal{S}$ \textbf{gematcht},
wenn eine Kante aus $\mathcal{S}$ inzident zu $v$ ist.\\\textnormal{\cite[S.~732]{cormen:matchings}}.
\end{definition}


Zwischen verschiedenen Typen des Matchings unterscheidet man auch das \tbf{perfekte Matching}.

\begin{definition}[Perfektes Matching]\label{def:perfect-matching}
Sei $\mathcal{G} = (\mathcal{V}, \mathcal{E})$ ein ungerichteter Graph. Ein \textbf{perfektes Matching} 
 ist so ein Matching, in dem alle Knoten aus $\mathcal{V}$ gematcht sind.
\end{definition}


Um die Aufgabe in der Form zu lösen, eignet sich gut der \tbf{Satz von Hall},
der als ein Ausgangspunkt der ganzen Matching--Theorie gilt. 
Um sich dieses Satzes zu bedienen, muss man noch den Begriff der \tbf{Nachbarschaft} einführen.

\begin{definition}[Nachbarschaft]\label{def:nachbarschaft}
Sei $\mathcal{G} = (\mathcal{V}, \mathcal{E})$ ein ungerichteter Graph.
Für alle $X \subseteq \mathcal{V}$ definieren wir die \textbf{Nachbarschaft}
von $X$ als $N(X) = \{y \in \mathcal{V} \,|\, \forall x \in X : (x, y) \in \mathcal{E}\}$.
\end{definition}


\begin{satz}[Satz von Hall]
Sei $\mathcal{G} = (\mathcal{L} \cup \mathcal{R}, \mathcal{E})$ ein bipartiter, ungerichteter Graph.
Es existiert ein perfektes Matching genau dann,
wenn für alle Teilmengen $\mathcal{K} \subseteq \mathcal{L}$ gilt: $|\mathcal{K}| \leqslant |N(\mathcal{K})|$.\textnormal{\cite[S.~736, Übung]{cormen:matchings}}
\end{satz}

\begin{proof}
Auf den
\href{https://homes.cs.washington.edu/~anuprao/pubs/CSE599sExtremal/lecture6.pdf}{Beweis}
\footnote{s. etwa: Anup Rao. Lecture 6 Hall’s Theorem. October 17, 2011. University of Washington. [Zugang 21.01.2021]\\
\url{https://homes.cs.washington.edu/~anuprao/pubs/CSE599sExtremal/lecture6.pdf}}
verzichten wir.
\end{proof}


An dieser Stelle stellen wir Folgendes fest.

\begin{lemma}\label{lem:komponente-complete}
Sei $C = (V_c, E_c)$ eine beliebige Zusammenhangskomponente in $G$. 
Dann bildet $C$ nach Verarbeitung jeder $k$--ten Spießkombination
selbst einen vollständigen, bipartiten Graphen.
\end{lemma}

\begin{proof}
Diese Aussage kann durch vollständige Induktion über $k \in \mathbb{N}$ bewiesen werden.

\tbf{Induktionsanfang}: 
Die beiden Mengen $A$ und $B$ sind gleichmächtig und ganz am Anfang ist $G$ vollständig.
Sei die erste Spießkombination $K_1 = (F_1, Z_1)$, wobei $F_1 \neq A$. (Falls $F_1 = A$,
dann gilt sofort die Aussage für $k = 1$.)
Nach der Schlussfolgerung nach Lemma \ref{lem:spiess-numbers} werden alle Kanten zwischen 
allen\break $x \in F_1$ und allen $y \in B \setminus Z_1$,
sowie zwischen allen $p \in Z_1$ und allen $q \in A \setminus F_1$ entfernt.
Nach der Verarbeitung von $K_1$ 
entstehen so zwei Zusammenhangskomponenten:
$C_1 = (L_1 \cup R_1, E_1)$ und\break $C_2 = (L_2 \cup R_2, E_2)$,
$C_1 \cup C_2 = G$, wobei o.B.d.A $L_1 \cup R_1 = F_1 \cup Z_1$.
Dann sind $L_1$ und $R_1$
nach Definition \ref{def:spiesskomb} auch gleichmächtig. 
Ebenfalls sind dann $L_2$ und $R_2$ gleichmächtig.
Nach der gennanten Schlussfolgerung gilt, dass alle Kanten zwischen $C_1$ und $C_2$
aus $E$ entfernt werden, aber alle Kanten innerhalb von $C_1$ und innerhalb von $C_2$
beibehalten werden.
Dies bedeutet, dass die Komponenten $C_1$ und $C_2$ selbst vollständige, bipartite Graphen sind.
Damit ist die Aussage für $k = 1$ bewiesen und der Induktionsanfang erledigt.

\tbf{Induktionsschritt}: Es gelte die Aussage, also die \tbf{Induktionsannahme},
für ein beliebiges, aber festes $k \in \mathbb{N}$, d.h., es gelte,
dass jede Zusammenhangskomponente in $G$ nach Verarbeitung von $k$ Spießkombinationen selbst
einen vollständigen, bipartiten Graphen bildet.

Zu zeigen ist die Aussage für $k + 1$, also, dass jede Zusammenhangskomponente in $G$ nach Verarbeitung
von $k + 1$ Spießkombinationen selbst einen vollständigen, bipartiten Graphen bildet.

Sei $K_i = (F_i, Z_i)$ die $k+1$--te Spießkombination.
Zu untersuchen ist die folgende Fallunterscheidung:
\begin{enumerate}[label={\upshape(\roman*)}]
  \item Sei $D = (V_D, E_D)$ eine Zusammenhangskomponente in $G$. Sei $F_i \cup Z_i = V_D$.
  Da alle Knoten der Spießkombination sich mit allen Knoten von $D$ decken,
  können keine Kanten nach der Folgerung aus Korollar \ref{kor:komponente-mengen} aus $E$ entfernt werden, deshalb
  entsteht keine neue Zusammenhangskomponente, also ist jede Zusammenhangskomponente
  nach der Induktionsannahme ein vollständiger, bipartiter Graph.\label{lem:komponente-complete1} 
  %Damit ist der Induktionsschritt für diesen Fall vollzogen und die Behauptung gilt für jedes
  %$k \in \mathbb{N}$.
  \item Sei $D = (L_D \cup R_D = V_D,\, E_D)$ eine Zusammenhangskomponente in $G$.
  Sei $F_i \cup Z_i \subsetneq V_D$, also gehört $F_i \cup Z_i$ nur zu einer
  Zusammenhangskomponente in $G$, aber deckt sich nicht mit allen Knoten.
  $D$ ist laut Induktionsannahme selbst ein vollständiger, bipartiter Graph.
  Nach der Schlussfolgerung nach Korollar \ref{kor:komponente-mengen} werden alle Kanten zwischen 
  allen $x \in F_i$ und allen $y \in R_D \setminus Z_i$,
  sowie zwischen allen $p \in Z_i$ und allen $q \in L_D \setminus F_i$ entfernt.
  So entstehen zwei neue Zusammenhangskomponenten: $C_1 = (L_1 \cup R_1, E_1)$ und $C_2 = (L_2 \cup R_2, E_2)$,
  o.B.d.A. $L_1 \cup R_1 = F_i \cup Z_i$ und $L_2 \cup R_2 = V_D \setminus (F_i \cup Z_i)$, die ebenfalls selbst 
  vollständige, bipartite Graphen sind.
  Jede andere Zusammenhangskomponente in $G$ ist
  nach der Induktionsannahme ein vollständiger, bipartiter Graph.\label{lem:komponente-complete2}

  \item Sei $1 \leqslant t \leqslant n$ beliebig ($n$ ist die Anzahl der Obstsorten), aber fest.
  Seien $C_1, C_2, ..., C_t$ paarweise verschiedene Zusammenhangskomponenten in $G$.
  Gehöre $F_i \cup Z_i$ zu mehreren Komponenten $C_p, ..., C_q$.
  Dann gilt für jede Zusammenhangskomponente $C_i$ entweder \ref{lem:komponente-complete1} 
  oder \ref{lem:komponente-complete2}, abhängig davon, ob $C_i$ vollständig zu $F_i \cup Z_i$
  gehört oder nur zum Teil. Das bedeutet, entweder entsteht keine neue Zusammenhangskomponente 
  \ref{lem:komponente-complete1} oder $C_i$ wird in zwei neue Zusammenhangskomponenten gespalten
  \ref{lem:komponente-complete2}.
\end{enumerate}

Da alle möglichen Fälle untersucht wurden, ist der Induktionsschritt vollzogen und die Behauptung gilt für jedes
$k \in \mathbb{N}$.
\end{proof}


\begin{figure}[ht]
\caption{Abbgebildet ist das Beispiel aus der Aufgabenstellung nach
der Verarbeitung von allen $m$ Spießkombinationen.}
\label{fig:graph-after-analysis}
\centering
\begin{subfigure}[b]{.49\textwidth}
\centering
\begin{tikzpicture}
    \node[vertex] (A) {$A$};
    \node[vertex] (B) [below = 0.4cm of A] {$B$};
    \node[vertex] (Br) [below = 0.4cm of B] {$Br$};
    \node[vertex] (E) [below = 0.4cm of Br] {$E$};
    \node[vertex] (P) [below = 0.4cm of E] {$P$};
    \node[vertex] (W) [below = 0.4cm of P] {$W$};
    \node[vertex] (1) [right = 1.5cm of A] {$1$};
    \node[vertex] (2) [right = 1.5cm of B] {$2$};
    \node[vertex] (3) [right = 1.5cm of Br] {$3$};
    \node[vertex] (4) [right = 1.5cm of E] {$4$};
    \node[vertex] (5) [right = 1.5cm of P] {$5$};
    \node[vertex] (6) [right = 1.5cm of W] {$6$};

    \path[draw,thick]
    (A) edge node {} (1)
    (A) edge node {} (4)
    (B) edge node {} (5)
    (Br) edge node {} (1)
    (Br) edge node {} (4)
    (E) edge node {} (2)
    (P) edge node {} (6)
    (W) edge node {} (3);
\end{tikzpicture}

\caption{Der Graph nach der Verarbeitung aller Spießkombinationen}
\label{fig:graph-after-infos}
\end{subfigure}
\begin{subfigure}[b]{.49\textwidth}
\centering
\begin{tikzpicture}
    \node[vertex] (A) {$A$};
    \node[vertex] (Br) [below = 0.4cm of A] {$Br$};
    \node[vertex] (1) [right = 1.5cm of A] {$1$};
    \node[vertex] (4) [right = 1.5cm of Br] {$4$};

    \path[draw,thick]
    (A) edge node {} (1)
    (A) edge node {} (4)
    (Br) edge node {} (1)
    (Br) edge node {} (4);
\end{tikzpicture}

\caption{Die übrige Zusammenhangskomponente mit mehr als 2 Knoten}
\label{fig:component-left}
\end{subfigure}
\end{figure}



\begin{lemma}\label{lem:komponente-matching}
Sei $C = (L_c \cup R_c, E_c)$ eine beliebige
Zusammenhangskomponente in $G$. Dann existiert immer ein perfektes Matching zu $C$.
\end{lemma}
\begin{proof} 
Nach Lemma \ref{lem:komponente-complete} ist jede Zusammenhangskomponente in $G$ ein vollständiger,
bipartiter Graph. Nach Satz von Hall existiert ein perfektes Matching, wenn
für alle Teilmengen $K \subseteq L_c$ gilt: $|K| \leqslant |N(K)|$.
Die obere Behauptung kann für beliebig große Mächtigkeiten $|K| = k \in \mathbb{N}$
durch die vollständige Induktion bewiesen werden.\\

\noindent
\tbf{Induktionsanfang:} Für $k = 1$ hat der einzelne Knoten $x \in K \subseteq L_c$
die Kardinalität $\Delta(x) = 1$.
Deshalb gilt: $|K| = 1 \leqslant |N(x)| = 1$. Damit stimmt die Behauptung für $k = 1$ und der Induktionsanfang ist erledigt.\\

\noindent
\tbf{Induktionsschritt:} Es gelte die Aussage für ein beliebiges $k \in \mathbb{N}$, also für eine Teilemenge
$K \subseteq L_c$, die aus $k$ Knoten besteht und in der jeder Knoten $x \in K$ die Kardinalität
$\Delta(x) = k$ hat.\\ Es gelte also: $|K| \leqslant |N(K)|$.\\
Zu zeigen ist die Aussage für $k + 1$, also für eine Teilmenge $K' \subseteq L_c$ der Mächtigkeit 
$|K'| = k+1$:
\[
|K'| \leqslant |N(K')|.
\] 
Wir verifizieren:

Jeder Knoten in $C$ hat den Grad $k+1$, also: $|K'| = k + 1 \leqslant |N(K')| = (k+1)^2 = k^2 + 2k + 1$.\\
Folglich stimmt die Behauptung für $k+1$.\\

Der Induktionsschritt ist damit vollzogen und es wurde bewiesen, dass die Behauptung für beliebige
Mächtigkeit von $K$ gilt.
Dadurch wurde auch bewiesen, dass es in einer Zusammenhangskomponente in $G$
 immer ein perfektes Matching gibt.
\end{proof}


Nach der Verarbeitung aller Spießkombinationen entsteht ein Graph mit vielen Zusammenhangskomponenten
(s. \cref{fig:graph-after-analysis}).
An dieser Stelle muss man noch die Wunschliste $W$ untersuchen, um die entsprechende Menge $W'$ zu bestimmen.
Dazu muss man die folgenden zwei Beobachtungen betrachten.

\begin{lemma}\label{lem:komponente-all-wunschliste}
Sei $C = (L_c \cup R_c, E_c)$ eine Zusammenhangskomponente in $G$.
Wenn gilt: $\forall x \in L_c : x \in W$, dann werden alle $y \in R_c$ in $W'$ hinzugefügt.
\end{lemma}
\begin{proof} 
Nach Axiom \ref{ax:obstsorte-index} besitzt jede Obstsorte genau einen einzigartigen Index.
Die Zusammenhangskomponente $C$ beschreibt nach Lemmata \ref{lem:komponente-complete} und \ref{lem:komponente-matching},
dass jede Obstsorte $p \in L_c$ jeden Index $q \in R_c$ haben kann, weil $C$ ein vollständiger, bipartiter Graph ist und ein perfektes Matching stets existiert.

Dadurch, dass $\forall x \in L_c : x \in W$ gilt,
ist ohne Bedeutung, welchen Index die jeweilige Obstsorte besitzt, da
 die Lösung des Problems eine Menge $W'$ mit den Indizes der Obstsorten aus $W$ sein soll.\\
Dadurch, dass $L_c \subseteq W$ gilt, gilt auch: $R_c \subseteq W'$. 
\end{proof}


\begin{lemma}\label{lem:komponente-one-not-wunschliste}
Sei $C = (L_c \cup R_c, E_c)$ eine Zusammenhangskomponente in $G$.
Wenn gilt: $\exists x \in L_c : x \notin W$ und $\exists y \in L_c : y \in W$,
dann kann die Menge $W'$ nicht eindeutig
bestimmt werden.
\end{lemma}
\begin{proof}
Nach Axiom \ref{ax:obstsorte-index} besitzt jede Obstsorte genau einen einzigartigen Index.
Die Zusammenhangskomponente $C$ beschreibt nach Lemmata \ref{lem:komponente-complete} und
\ref{lem:komponente-matching},
dass jede Obstsorte $p \in L_c$ jeden Index $q \in R_c$ haben kann, weil $C$ ein vollständiger, bipartiter Graph ist und ein perfektes Matching stets existiert.

Angenommen, $\exists r \in L_c : r \notin W$. Dann ist es unmöglich, festzustellen,
welcher Index aus $R_c$ der Obstsorte $r$ gehört.
Also ist es auch unmöglich, festzustellen, welche Indizes in $W'$ hinzugefügt werden sollen.
Deshalb
ist es unmöglich (unabhängig von allen anderen Zusammenhangskomponenten des Graphen $G$),
eine eindeutige Menge der Indizes der gewünschten Obstsorten festzulegen.
Dadurch gibt es keine eindeutige Lösung zu diesem Problem für diese Eingabe.
\end{proof}


Direkt aus \cref{lem:komponente-all-wunschliste} ergibt sich das folgende Korollar.
Man bedient sich dessen und des Lemmas \ref{lem:komponente-one-not-wunschliste}, um das ganze Problem zu lösen, also: Ob die Menge $W'$ eindeutig bestimmt werden kann.

\begin{korollar}
Seien $C_1 = (L_1 \cup R_1, E_1), ..., C_k = (L_k \cup R_k, E_k)$ alle Zusammenhangskomponenten in $G$,
für jede $i$--te von denen gilt: $\exists x \in L_i : x \in W$.
Falls für jede $i$--te von diesen Komponenten gilt: $L_i \subseteq W$, dann
kann $W'$ eindeutig und vollständig bestimmt werden.
\end{korollar}


Man stellt fest, dass man die Menge $W$ untersuchen kann und wenn ein $x \in W$ in $G$ die
Kardinalität $\Delta(x) = 1$ besitzt, kann der einzelne Nachbar von $x$ in $W'$ hinzugefügt werden
(Lemma \ref{lem:komponente-all-wunschliste}).
Im sonstigen Fall, also wenn $\Delta(x) > 1$, muss die ganze Zusammenhangskomponente
$C_x = (L_x \cup R_c, E_x)$, zu der $x$ gehört, untersucht werden, ob gilt: $\forall p \in L_x: p \in W$
(Lemmata \ref{lem:komponente-all-wunschliste} und \ref{lem:komponente-one-not-wunschliste}).

Man erstellt eine Liste $\bar{W}$ der Länge $n$,
in der die Zugehörigkeit einer Obstsorte zur Wunschliste $W$ durch 1 oder 0 gekennzeichnet wird
(s. \nameref{sec:umsetzung}). Außerdem erstellt wird eine Liste $\bar{R}$ der Länge $n$,
in der jede gewünschte Obstsorte $x$ als 1 gekennzeichnet wird, falls der Knoten $x$ in $G$ bereits
besucht wurde (s. \nameref{sec:umsetzung}).

Wenn man einen Knoten $x \in W$ untersucht, dessen Kardinalität $\Delta(x) > 1$ ist,
kann man die Liste der Nachbarknoten $n(x)$ von $x$ aufrufen.
Da eine Zusammenhangskomponente selbst vollständig ist\\ (\cref{lem:komponente-complete}),
kann man die Liste der Nachbarknoten $n(y)$ eines beliebigen Nachbarn $y$ von $x$ ($y \in n(x)$) aufrufen.
So kann man jeden Knoten $z \in n(y)$ untersuchen, ob bei $z$ eine 1 in $\bar{W}$ steht.
Falls ja, wird $z$ auch in $\bar{R}$ markiert,
sodass man denselben Vorgang bei einem anderen Knoten in dieser Komponente nicht wiederholen muss.
Falls alle $z$ zu $W$ gehören, wird
die ganze Liste $n(x)$ in $W'$ hinzugefügt. Sonst werden alle Knoten dieser Komponente 
gespeichert, insbesondere diese Obstsorten, die zu $W$ nicht gehören.\\
Man wiederholt diesen Vorgang, bis alle gewünschten Obstsorten mit 1 in $\bar{R}$ markiert werden.

Ausgegeben wird entweder die vollständige Menge $W'$ oder eine Meldung über die jeweilige 
Zusammenhangskomponente, zu der Obstsorten gehören, die nicht gewünscht waren.
Diese werden auch in der Ausgabe aufgezählt.


\subsection{Prüfung auf Korrektheit der Eingabe}\label{sec:korrektheit-eingabe}
Am Ende des Teils \ref{sec:formulierung} wurde bemerkt, dass die Korrektheit und Vollständigkeit
der Lösung davon abhängt, ob alle Obstsorten in einer Eingabe Axiom \ref{ax:obstsorte-index} folgen.

Indentifizieren wir zuerst die Probleme, die auftreten können.
In den folgenden Überlegungen nehmen wir an, dass jede Spießkombination $K = (F_i, Z_i)$ so gebildet wird,
dass gilt: $|F_i| = |Z_i|$.
(Falls man dies nicht angenommen hätte, wäre eine Eingabe schon an dieser Stelle
falsch, da eine Obstsorte zwei verschiedene Indizes oder zwei Obstsorten denselben Index haben müssten.
Außerdem ist dieser Fehler leicht herauszufinden, indem man beim Einlesen prüft,
ob die beiden Mengen gleichmächtig sind.)
%\TODO{dopisac ten przypadek}
Im Allgemeinen kommt es zu einem Widerspruch, wenn die Eingabe
dem Axiom \ref{ax:obstsorte-index} nicht folgt. 
Das heißt, es können die folgenden 
Möglichkeiten auftreten:
\begin{enumerate}[label={(P\arabic*)}]
  \item In einer Eingabe existieren zwei Obstsorten: $o(x, i)$ und $o(y, i)$, wobei $x \neq y$,\label{probleme1}
  \item In einer Eingabe existieren zwei Obstsorten: $o(x, i)$ und $o(x, j)$, wobei $i \neq j$.\label{probleme2}
\end{enumerate}

Untersuchen wir die Situation, in der die folgenden 
zwei Obstsorten existieren: $o(x, i)$ und $o(y, j)$.
Nehmen wir an dieser Stelle an, dass $i=j$.
Betrachten wir dazu zwei Spießkombinationen: $K_1 = (F_1, Z_1)$ und $K_2 = (F_2, Z_2)$.
Es gelte: $x \in F_1$ und entsprechend $i \in Z_1$.
%\TODO{Bild vielleicht?}
\begin{enumerate}[label={\upshape(F\arabic*)}]
  \item Falls $y \in F_1$ und $i = j$, dann ist $i$ bereits in $Z_1$. Damit $|F_1| = |Z_1|$ gilt,
  muss gelten: $\exists o(z, k) : z \notin F_1 \land k \in Z_1$.
  Dann muss zwar kein Widerpsurch erfolgen, aber wir haben der Obstsorte einen Index zugewiesen,
  also kann an dieser Stelle die Beziehung zwischen $z$ und $k$ gar nicht festgestellt werden.
  Falls alle anderen Spießkombinationen widerspruchsfrei sind,
  wird $z$ ein Index $\ell \in B \setminus F_1$ zugewiesen.\label{fall1}

  \item Falls $y \notin F_1 \land y \in F_2 \land x \notin F_2 \land i =j$, dann ist $i$ bereits in $Z_1$.
  Dann muss für $i$ auch gelten: $i \in Z_2$. Am Anfang ist der bipartite Graph $G$ vollständig.
  Nach der Verarbeitung der Spießkombination $K_1$ werden alle Kanten zwischen allen
  $p \in Z_1$ und allen $q \in A \setminus F_1$, sowie alle Kanten zwischen allen
  $p \in F_1$ und allen $q \in B \setminus Z_1$ entfernt, darunter auch die Kante zwischen
  $y$ und $i$. Nach der Verarbeitung von $K_2$ wird auch die Kante zwischen $x$ und $i$
  entfernt, da $x \notin F_2$. Der Knoten $x$ hat dann eine Kardinalität um 1 kleiner
  als der Rest der Knoten auf dieser Komponente. Insbesondere: Wenn die Mengen
  $F_1$ und $F_2$ jeweils eine Mächtigkeit von 2 haben, hat $x$ dann den Grad 0.\label{fall2}
\end{enumerate}

Bei der Untersuchung der Situation für \ref{probleme2} geht man durch eine analoge Fallunterscheidung wie
in \ref{fall1} und \ref{fall2} vor.

Um zu prüfen, ob die Eingabe Axiom \ref{ax:obstsorte-index} widerspricht, muss man deshalb
nur untersuchen, ob die Kardinalität jedes Knotens mit der Kardinalität eines seiner 
Nachbarn nicht übereinstimmt.\\
Dazu muss man beachten, dass die Zahl $n$ in einigen Beispieldateien größer ist
als die Anzahl der in Spießkombinationen und in der Wunschliste verwendeteten Obstsorten
und Indizes. In diesem Fall muss man die nicht genutzten Obstsorten und Indizes beim Einesen
entsprechend markieren und sie beim Prüfen auf Korrektheit der Eingabe überspringen. Mehr dazu
in der \nameref{sec:umsetzung}.

Sehen Sie dazu die folgenden Beispiele: \nameref{example:8}, \nameref{example:9}, \nameref{example:14}.

%Untersuchen wir noch die Situation, in der die Mengen $F$ und $Z$ einer Spießkombination $K$ nicht
%gleichmächtig seien. Es gelte: $|F| = |Z| + h, h \in \mathbb{N}$.


\subsection{Laufzeit}\label{sec:laufzeit}
$n$ --- die Anzahl der Obstsorten\\
$m$ --- die Anzahl der Spießkombinationen\\
$w$ --- die Anzahl der Wünsche (also $|W|$), im worst-case $w = n$\\

Da wir Bitmasken in unserem Programm verwenden, ist nocht eine Konstante einzuführen: $\beta$,
die für die $\beta$--Bit--Architektur eines Rechners\footnote{\href{https://en.wikipedia.org/wiki/Word_(computer_architecture)}{https://en.wikipedia.org/wiki/Word\_(computer\_architecture)}}
steht, auf dem das Programm ausgeführt wird. D.h., bei der 64--Bit--Architektur beträgt $\beta = 64$.
Die bitweisen Operationen in C++ auf \ttt{bitset} werden in der Laufzeit von $O(\frac{|k|}{\beta})$
ausgefüht, wobei $|k|$ die Länge eines \ttt{bitset} ist. Außerdem muss die Länge eines \ttt{bitset}
kontant sein, d.h., man muss schon im Programm eine feste Länge für alle Eingabegrößen eingeben.
Diese feste Länge nennen wir $N$ und setzen $N =26$, da so viele Obstsorten das größte Beispiel
auf der BWINF--Webseite umfasst.
\mb{coś na temat słowa maszynowego}\\

\begin{itemize}
  \item Einlesen: $O(n (\frac{N}{\beta} + m \log n) + w (\log n + \log w))$ (worst-case) \mb{sprawdzić bitsety}
  \begin{itemize}
    \item Erstellung der Adjazenzmatrix $M$: $O(n \cdot \frac{N}{\beta})$

    \item Erstellung der Liste \ttt{used} (s. \nameref{sec:umsetzung}): $O(n)$

    \item Einlesen der Menge $W$: $O(w \log w)$\\%, worst-case: $O(n \log n)$\\
     Implementierug von \ttt{set} in C++ als Rot-schwarz-Bäume\footnote{\href{https://en.cppreference.com/w/cpp/container/set}{https://en.cppreference.com/w/cpp/container/set}}

    \item Einlesen der Spießkombinationen: $O(m \cdot n \log n)$ (worst-case)\\
    Im schlimmsten Fall enthält jede Spießkombination alle Obstsorten.
    Die logarithmische Laufzeit ist durch Einfügen in eine Menge verursacht.

    \item Zuweisung der internen Indizes der Obstsorten (s. \nameref{sec:umsetzung}): $O(n \log n)$\\
     Implementierug von \ttt{map} in C++ als Rot-schwarz-Bäume\footnote{\href{https://en.cppreference.com/w/cpp/container/map}{https://en.cppreference.com/w/cpp/container/map}}

    \item Umwandlung der gewünschten Obstsorten von Strings zu Integers: $O(w (\log n + \log w))$\\%worst-case: $O(n \log n)$\\
    Das Suchen in einer \ttt{map} hat logarithmische Laufzeit bezüglich der Anzahl
    der Obstsorten: $O(\log n)$.
    Das Einfügen in eine \ttt{set} hat logarithmische Laufzeit bezüglich der Anzahl
    der Wünsche: $O(\log w)$. Also gilt für die gesamte Laufzeit: $O(w (\log n + \log w))$. 

    \item Umwandlung der Obstsorten in allen Spießkombinationen von Strings zu Integers:
    $O(m \cdot n \log n)$\\
    In jeder Spießkombination können sich im worst-case alle $n$ Obstsorten befinden.
    Das Suchen in einer \ttt{map} hat logarithmische Laufzeit bezüglich der Anzahl
    der Obstsorten: $O(\log n)$.
    Das Einfügen in eine \ttt{set} hat ebenfalls logarithmische Laufzeit bezüglich der Anzahl
    der Obstsorten: $O(\log n)$. 
    Deshalb beträgt die Laufzeit für alle Obstsorten in einer Spießkombination höchstens: $O(n \log n)$. 

    \item Die gesamte Laufzeit für diesen Teil (worst-case): $O(n \cdot \frac{N}{\beta}) + O(n) + O(w \log w)
    + O(m \cdot n \log n) + O(w (\log n + \log w)) + O(m \cdot n \log n) = 
    O(n \cdot \frac{N}{\beta} + n + w \log w + m \cdot n \log n + w (\log n + \log w) + m \cdot n \log n) \in 
    O(n \cdot \frac{N}{\beta} + m \cdot n \log n + w (\log n + \log w)) \in 
    O(n (\frac{N}{\beta} + m \log n) + w (\log n + \log w))$ 
  \end{itemize}


  \item Verarbeitung der Spießkombinationen: $O(m \cdot n \cdot\frac{N}{\beta} + n^2 \log N)$ (worst-case)
  \begin{itemize}
    \item Verarbeitung einer Spießkombination $K = (F, Z)$: $O(n \cdot\frac{N}{\beta})$\\
    Man geht davon aus, dass eine Spießkombination im worst-case alle Obstsorten enthält.
    \begin{itemize}
      \item Erstellung der Bitmaske $bf$: $O(n)$

      \item Erstellung der Bitmaske $bn$: $O(|F|)$, worst-case: $O(n)$\\
      Die Operation hat eine lineare Laufzeit bezüglich der Anzahl der Elementen in 
      einer Spießkombination. Eine Spießkombination kann im worst-case alle $n$
      Obstsorten beinhalten.

      \item Erstellung der Bitmaske $br$: $O(\frac{N}{\beta})$

      \item Entfernen der Kanten: $O(n \cdot \frac{N}{\beta})$\\
      Für jede Liste $M_i$ wird geprüft, ob $i$ sich in $F$ befindet.
      Diese Operation kann in $O(1)$ ausgeführt werden, indem wir durch
      die Menge $F$ gleichzeitig iterieren, wie durch die Matrix $M$ (s. \nameref{sec:umsetzung}).
      An jeder Liste $M_i$ wird genau eine bitweise Operation durchgeführt.

      \item Die gesamte Laufzeit für eine Spießkombination beträgt (worst-case):\\
      $O(n) + O(n) + O(\frac{N}{\beta}) + O(n \cdot\frac{N}{\beta}) 
      = O(n + n + \frac{N}{\beta} + n \cdot\frac{N}{\beta}) \in O(n \cdot\frac{N}{\beta})$ \mb{zapis?}
    \end{itemize}

    \item Verarbeitung der allen Spießkombination entsprechend: $O(m \cdot n \cdot\frac{N}{\beta})$

    \item Kopieren der Adjazenzmatrix in \ttt{Graph G} (s. \nameref{sec:umsetzung}): average-case: $O(n \cdot \delta \cdot \log N)$, worst-case: $O(n^2 \log N)$ \\
    Für jede Liste $M_i$ werden alle 1--en in \ttt{G} als Kanten vom Knoten $i$ eingefügt.
    Dazu bediene ich mich der eingebauten Funktion \ttt{\_Find\_next()},
    die jeweils das nächste 1--Bit in einem \ttt{bitset} findet.
    Jedoch ihre Laufzeit ist mir nicht bekannt.
    Ich gehe davon aus, dass dieser Vorgang in logarithmischer Laufzeit abzuschließen ist, 
    wie es \href{https://stackoverflow.com/questions/58795338/find-next-array-index-with-true-value-c}{hier}\footnote{\href{https://stackoverflow.com/questions/58795338/find-next-array-index-with-true-value-c}{https://stackoverflow.com/questions/58795338/find-next-array-index-with-true-value-c}} beschrieben ist.\\
    Deshalb erfolgt die Iteration über eine Liste $M_i$ in $O(\delta \log N)$, wobei
    $\delta$ die Anzahl der 1--en ist. Im schlimmsten Fall, wenn alle Spießkombinationen aus
    $n$ Obstsorten bestehen, gilt: $\delta = n$, also gilt es: $O(n\log N)$.
    Dennoch im allgemeinen Fall gilt: $\delta \ll n$. Den Vorgang muss man für alle $n$ Obstsorten
    ausführen.\\
    Man könnte auch denken, dass das Kopieren unnötig ist. Falls man den Graphen nicht
    in eine Adjazenzliste--Form kopiert,
    muss man sowieso an einer Stelle die entsprechenden 1--en aus der Adjazenzmatrix ablesen.

    \item Die gesamte Laufzeit für diesen Teil beträgt (worst-case):\\
    $O(m \cdot n \cdot\frac{N}{\beta}) + O(n^2 \log N) = O(m \cdot n \cdot\frac{N}{\beta} + n^2 \log N)$ 

  \end{itemize}

  \item Prüfung der Korrektheit der Eingabe: $O(n)$
  \begin{itemize}
    \item Für jeden Knoten im Graphen wird geprüft, ob 
    er in \ttt{used} markiert ist: $O(1)$ und, bo die Kardinaliät
    dieses Knotens nicht 0 beträgt: $O(1)$.\\
    Deshalb für alle Knoten im bipartiten Graphen gilt: $O(n + n) \in O(n)$ \mb{zapis?}
  \end{itemize}


  \item Prüfung der Existenz einer Lösung: $O(n \log n)$ (worst-case)
  \begin{itemize}
    \item Erstellung von $\bar{W}$: $O(n)$

    \item Erstellung von $\bar{R}$: $O(n)$

    \item Prüfung der Wunschliste: $O(w \log w)$\\
    Es wird geprüft, ob die Kardinalität des jeden Knotens 1 beträgt
    \begin{itemize}
    \item Prüfung auf Kardinalität $\Delta(x) = 1$: $O(1)$
    \item Zugriff auf die Liste der Nachbarknoten in $G$: $O(1)$
    \item ggf. Einfügen in $W'$: $O(\log w)$
    \item ggf. Einfügen in \ttt{multip} (s. \nameref{sec:umsetzung}): $O(\log w)$ (worst-case)\\
    Im schlimmsten Fall hat keiner der Knoten in der Wunschliste die Kardinalität von 1.
    \item ggf. Markierung in $\bar{W}$: $O(1)$
    \end{itemize}

    \item Iteration durch \ttt{multip}: $O(n \log w)$ (worst-case)\\
    Die folgenden Operation werden nur dann ausgeführt, wenn die Komponente,
    zu der der iterierte Knoten gehört, noch nicht bearbeitet wurde, also 
    ob dieser Knoten in $\bar{R}$ markiert wurde. Wir können feststellen, 
    dass diese Bedingung im schlimmsten Fall nur $\frac{w}{2}$--mal erfüllt werden kann.
    Alle Komponenten mit genau 1 Knoten aus $A$ wurden bereits behandelt und in diesem
    Fall müssten alle Komponenten aus genau 2 Knoten aus $A$ bestehen.
    $\frac{w}{2}$ ist somit die maximale Anzahl an Zusammenhangskomponenten in $G$,
    die mehr als einen Knoten aus $A$ besitzen.
    \begin{itemize}
    \item Prüfung auf Markierung in $\bar{R}$: $O(1)$

    \item ggf. Zugriff auf die Liste der Nachbarn eines iterierten Knotens $x$ ($n(x)$): $O(1)$

    \item ggf. Zugriff auf die Liste der Nachbarn eines Nachbarn $y$ eines iterierten Knotens ($n(y)$): $O(1)$

    \item ggf. Iteration durch $n(y)$: $O(n)$ (worst-case)\\
    In dieser Schleife wird geprüft, ob der iterierte Index $i$ in $\bar{W}$ markiert ist: $O(1)$,
    und dann wird $i$ in $\bar{R}$ markiert: $O(1)$. Wenn es einen Knoten gibt, der nicht gwünscht ist,
    aber sich auf der Komponente befindet, wird dies mit einer boolschen Variable \ttt{prob} markiert: $O(1)$.\\
    Im worst-case kann die Schleife $n$--mal iteriert werden, wenn es nur eine Zusammenhangskomponente
    in $G$ gibt. Jedoch wird die äußere Schleife nur einmal iteriert, da alle gewünschten 
    Obstsorten auf der Komponente als besucht in $\bar{R}$ markiert werden.
    \item ggf. Kopieren der Knoten aus dieser Komponente zu \ttt{problems} (s. \nameref{sec:umsetzung}): $O(n)$ (worst-case)

    \item ggf. Einfügen der Knoten aus dieser Komponente zu $W'$: $O(n \log w)$ (worst-case)\\
    Das Einfügen in eine Menge hat eine logarithmische Laufzeit bezüglich $w$.

    \item Um die gesamte Laufzeit für diesen Teil zu bestimmen, müssen wir bemerken, dass 
    diese Laufzeit von der Anzahl der Knoten der Menge $A$ auf allen Zusammenhangskomponenten 
    abhängt, auf deren sich mind. eine gewünschte Obstsorte befindet. Durch die Markierung
    in $\bar{R}$ wird jeder Knoten auf jeder von diesen Zusammenhangskomponenten nur einmal behandelt. 
    Damit ergibt sich im worst-case die Laufzeit von $O(n \log w)$, indem die gewünschten Obstsorten auf 
    allen Zusammenhangskomponenten in $G$ verteilt sind.
    \end{itemize}

    \item Ausgabe für eine Eingabe, für die $W'$ nicht eindeutig bestimmt werden kann: $O(n \log n)$\\
    Es wird durch alle Komponenten iteriert, die mind. eine ungewünschte Obstsorte enthalten,
    und alle Knoten werden mit den Namen der Obstsorte aufgezählt.
    Deshalb muss jedes Mal die Suchfunktion in der \ttt{map} mit den Zuweisungen 
    der internen Indizes der Obstsorten und den Obstsorten aufgerufen werden: $O(\log n)$.
    Im schlimmsten Fall muss in der Ausgabe durch alle Knoten in $A$ iteriert werden, falls
    es nur eine Zusammenhangskomponente in $G$ gibt.

    \item Die gesamte Laufzeit für diesen Teil beträgt (worst-case, nach oben geschätzt: $n > w$):
    $O(n) + O(n) + O(w \log w) + O(n \log w) + O(n \log n) = O(n + n + w \log w + n \log w + n \log n)
    \in O(n \log n)$ \mb{zapis? zgadza się?}
  \end{itemize}

\end{itemize}

Fassen wir die Laufzeit im worst-case zusammen. Also ein worst-case kann so ein Fall gelten,
in dem alle $m$ Spießkombinationen und die Wunschliste aus allen $n$ Obstsorten bestehen. 
\begin{itemize}
  \item Einlesen: $O(n (\frac{N}{\beta} + m \log n) + w (\log n + \log w))$
  \item Verarbeitung der Spießkombinationen: $O(m \cdot n \cdot\frac{N}{\beta} + n^2 \log N)$
  \item Prüfung der Korrektheit der Eingabe: $O(n)$
  \item Prüfung der Existenz einer Lösung: $O(n \log n)$
\end{itemize}

\TODO{sprawdzić}
$O(n (\frac{N}{\beta} + m \log n) + w (\log n + \log w)) + 
O(m \cdot n \cdot\frac{N}{\beta} + n^2 \log N)+
O(n)+
O(n \log n) =\\=
O(n (\frac{N}{\beta} + m \log n) + w (\log n + \log w) + 
m \cdot n \cdot\frac{N}{\beta} + n^2 \log N + n + n \log n)\\ \in
O(n (\frac{N}{\beta} + m (\log n + \frac{N}{\beta}) + n \log N) + w (\log n + \log w))$\\

\noindent Nach der Abschätzung $w = n$ ergibt sich:\\
$O(n (\frac{N}{\beta} + m (\log n + \frac{N}{\beta}) + n \log N))$\\

\noindent Für die maximale Anzahl an Obstsorten, die in den BWINF--Beispielen auftreten, 
können wir bemerken, dass fast alle modernen Rechner auf einer
mind. 32--Bit-Architektur basieren, das heißt, wir können $\beta = 32$ setzen. 
In diesem Zusammenhang gilt: $\frac{N}{\beta} = \frac{26}{32} < 0$.
In diesem Fall ist dieser Bruch ein vernachlässiger Faktor. Es ergibt sich im schlimmsten Fall:\\
\[
O(n (m \log n + n \log N))
\]


%\subsection{Speicherplatz}\label{sec:Speicherplatz}


\printbibliography

\newpage
\section{Umsetzung}\label{sec:umsetzung}
\subsection{Klasse \ttt{Solver}}
Die Matrix $M$ wird als ein \ttt{vector} von \ttt{bitset} dargestellt.
Dazu muss man erwähnen, dass ein \ttt{bitset} in C++ eine feste Länge besitzen muss.
Dazu wurde die maximale Größe von $n$ eingegeben, also 26.
Das müsste ggf. im Program selbst umgestellt werden,
falls man eine größere Datei einlesen möchte.\\
Die feste Länge ist auch der Grund dafür, dass die Bitmaske $br$ im Teil \ref{sec:logik}
auf folgende Weise definiert wird:
\[
br := \neg(bn) \land bf.
\]
Im Fall, wenn man mit einer Bimaske einer festen Größe operiert, würde 
eine einfache Negation der Bitmaske $bn$ nicht hinreichen.\\

Die Obstsorten werden als Strings eingelesen, aber in der Methode \ttt{readFile()}
wird jeder Obstsorte ein interner Index\footnote{Nummerierung ab 0.} zugewiesen
und in weiteren Operationen im Programm
werden die Obstsorten als einfache Integers behandelt. 
Es werden dazu zwei Maps festgelegt: \ttt{fruit2ID} und \ttt{ID2Fruit}, 
in denen die Obstsorten und die entsprechenden internen Indizes gespeichert sind.\\

Jede Spießkombination wird als \ttt{pair<set<int>, set<int>}, also ein Tupel entsprechend
aus der Menge der internen Indizes der Obstsorten und der Menge der Indizes aus der Aufgabenstellung.
Alle Spießkombinationen werden in einem \ttt{vector} gespiechert, der \ttt{infos} heißt.\\

In \ttt{wishes} werden als ein \ttt{set} von Integers werden die internen Indizes der gewünschten 
Obstsorten, also der Elemente der Menge $W$, gespeichert.
Analog werden in \ttt{result}, also ebenfalls ein \ttt{set} von Integers, die Indizes der gewünschten 
Obstsorten, also die Elemente der Menge $W'$ hinzugefügt.\\ 

Der \ttt{vector}, der \ttt{used} heißt, wird verwendet, um die benutzten internen Indizes der Obstsorten,
wie auch die Indizes der Obstsorten zu markieren, die im Graphen verwendet werden,
da es in einzigen Textdateien ein größeres $n$ gibt als die Anzahl der in den Spießkombinationen 
un der Wunschliste verwendeten Obstsorten und Indizes. 
Diese Markierung zeigt sich bei der Prüfung auf korrekte Eingabe hilfreich.\\

Die Methode \ttt{analyzeInfo()} nimmt als Argument eine Spießkombination. 
In der Methode werden die drei Bitmasken $bn, br, bf$ erstellt.
Um die Laufzeit zu optimieren, wird durch die Menge der internen Indizes der Obstsorten \ttt{fruits}
aus dieser Spießkombination gleichzeitig mit den Obstsorten aus der Matrix iteriert.
Da die Menge \ttt{fruits} vorsortiert ist, müssen wir nicht bei jeder Obstsorte in der Matrix
prüfen, ob sie sich in \ttt{fruits} befindet, um die Entscheidung zu treffen, welche der beiden
Bitmasken anzuwenden.\\
Nachdem alle $m$ Spießkombinationen verarbeitet wurden, werden in der Methode \ttt{analyzeAllInfos()}
alle übrigen 1--Beziehungen in der Adjazenzmatrix
als Kanten in den Graphen \ttt{G} der Klasse \ttt{Graph} kopiert. 
Die einzelnen 1--en in \ttt{matrix} werden mithilfe der eingebauten Funktion \ttt{\_Find\_next()}
gefunden.\\

Die Methode \ttt{checkCoherence()} prüft, ob die Eingabe dem Axiom \ref{ax:obstsorte-index} folgt.
Dennoch es gibt Beispiele, in denen $n$ größer ist als die Anzahl der verwendeten Elementen in den
Spießkombinationen und der Wunschliste.
Bei der Zuweisung der internen Indizes jeder Obstsorte werden die verwendeten Obstsorten und Indizes
in \ttt{used} markiert. Diesen Vorteil können wir nutzen, um die Prüfung der Kardinalität des
jeden Knotens in \ttt{G} zu überprüfen: Es wird geprüft, ob die Kardinalität 0 beträgt. 
Alle Knoten, die in \ttt{used} nicht markiert wurden, werden übersprungen.
Falls bei mindestens einem Knoten die Kardinalität 0 ist, wird \ttt{false} ausgegeben und
die Eingabe im gegebenen Beispiel ist fehlerhaft.\\
Sonst, wenn alle verwendeten Knoten eine Kardinalität von mindestens 1 besitzen, wird \ttt{true}
ausgegeben.\\ 

Nachdem die Korrektheit der Eingabe geprüft wurde, kann festgestellt werden, ob $W'$ eindeutig
bestimmt werden kann. Dazu dient die Methode \ttt{checkResult()}.\\
Es werden ein \ttt{vector todo} der Länge $n$, ein \ttt{vector ready} der Länge $n$ und 
ein \ttt{set multip} erstellt.
\ttt{todo} ist die Liste $\bar{W}$. \ttt{ready} ist die Liste $\bar{R}$.\\
Es wird zunächst über die Menge der Wünsche \ttt{wishes} iteriert. 
Falls ein Knoten \ttt{x} (der interne Index einer Obstsorte) in \ttt{G}
die Kardinalität 1 besitzt, wird sein einzelner
Nachbar in \ttt{result} hinzugefügt.
Im sonstigen Fall wird \ttt{x} in \ttt{multip} hinzugefügt und die Stelle \ttt{x} in \ttt{todo}
wird mit 1 markiert.\\
Dann wird geprüft, ob die Menge \ttt{multip} überhaupt irgendwelche Elemente enthält.
Falls nicht, gibt die ganze Funktion an dieser Stelle \ttt{true} zurück.\\
Sont wird eine boolsche Variable \ttt{solv} erstellt, die für die Existenz einer Lösung steht.
Am Anfang nimmt sie \ttt{true} als Wert. 
Dazu wird auch eine Liste von Mengen \ttt{problems} erstellt, die dazu da ist, um
die Obstsorten einer Zusammenhangskomponente zu speichern, die eine nicht gewünschte Obstsorte enthält.\\
Danach wird über die Menge \ttt{multip} iteriert. 
Für jedes Element \ttt{x} aus dieser Menge wird eine boolsche Variable \ttt{prob} erstellt, die 
anzeigt, ob mindestens eine Obstsorte zu der Komponente gehört, die nicht gewünscht ist.
Am Anfang hat sie den Wert \ttt{false}.\\
Es wird zunächst geprüft, ob an der Stelle \ttt{x} in \ttt{ready} 0 steht, d.h.,
der Knoten gehört zu einer Zusammenhangskomponente, die noch nicht bearbeitet wurde.
Falls ja, dann werden zwei Listen \ttt{setB} und \ttt{setA} erstellt,
die der Liste der Nachbarn von \ttt{x} und der Liste der Nachbarn von einem Nachbarn von \ttt{x}
entsprechen.
Es wird durch die Menge \ttt{setA} iteriert.\\
Falls an der Stelle eines internen Index
einer Obstsorte in \ttt{todo} keine 1 steht, wird \ttt{solv = false} und \ttt{prob = true} gesetzt.
Dies bedeutet, es gibt eine nicht gewünschte Obstsorte in der Komponente.\\
Sonst wird die Stelle des internen Index dieser Obstsorte in \ttt{ready} mit 1 markiert.\\
Danach wird geprüft, ob \ttt{prob == true}. Falls \ttt{prob == false} gilt, wird die Menge \ttt{setB}
in \ttt{result} hinzugefügt. Sonst, wird die Menge \ttt{setA} in \ttt{problems} hinzugefügt, um sie 
danach als Nachricht für den Nutzer vorzustellen, dass aufgrund von einzigen Obstsorten 
die Menge $W'$ nicht eindeutig bestimmt werden kann.\\
Am Ende, nach der Schleife über \ttt{multip}, wird geprüft, ob es eine Lösung zur Aufgabe für eine
Eingabe gibt: Es wird gepüft, ob \ttt{solv == true}. Falls ja, dann wir \ttt{true} zurückgegeben.
Sonst wird eine Meldung ausgegeben, die alle Zusammenhangskomponenten beinhaltet, 
die eine nicht gewünschte Obstsorte enthalten, und die nicht gewünschten Obstsorten werden
ebenfalls angezeigt.


\subsection{Klasse \ttt{Graph}}
Diese Klasse ist grundsätzlich aus Übersichtlichkeits--  sowie aus Vereinfachungsgründen
entstanden. Theoretisch könnte man die enthaltenen Methoden in der Klasse \ttt{Solver} 
speichern.\\

Der Graph ist als eine Adjazenzliste aus \ttt{vector} von \ttt{vector} von Integers gespeichert.
Der Konstruktor nimmt zwei Parameter: die Größen der beiden Partitionen
im bipartiten Graphen, obwohl sie in der Aufgabe gleich groß sind.\\

Zu den verfügbaren Methoden zählen: \ttt{addEdge()},
die eine ungerichtete Kante zwischen zwei Knoten einfügt; \ttt{getFirstNeighbor()}, die den ersten Nachbarn eines Knotens zurückgibt; \ttt{deg()}, die die Kardinalität
eines Knotens zurückgibt; \ttt{getNeighbors()}, die den \ttt{vector}, also die Adjazenzliste
eines Knotens zurückgibt. Außerdem gibt es ein paar Methoden, die zum Debugging dienen. 


\newpage
\section{Beispiele}
\TODO{dodać dodatkowe przykłady}
\subsection{Beispiel 0 (Aufgabenstellung --- Teil a)}\label{example:0}
Textdatei: \texttt{spiesse0.txt}\\
\noindent
\framebox{Apfel, Brombeere, Weintraube}\\

\noindent
\framebox{1, 3, 4}\\

Auf Papier kann man das Beispiel auf folgende Weise lösen.
Am Anfang weiß man nichts über die Obstsorten in $A$ und die Indizes in $B$.
Wir zeichnen deshalb einen vollständigen, bipartiten Graphen $G$ (Abb. \ref{fig:example0-1}).
Jede Kante steht für eine mögliche Zuweisung eines Index einer Obstsorte.\\

Wir analysieren die 1. Spießkombination:
\minibox[frame]{$F_1 = \{\text{Apfel, Banane, Brombeere}\}, Z_1 = \{1, 4, 5\}$}.\\
Wir stellen fest, dass Apfel, Banane und Brombeere jeweils keine Indizes 2, 3, 6 haben können,
weil jeder dieser Obstsorten ein Index aus der Menge $Z_1$ zugewiesen wird.
Gleichzeitig merken wir, dass Erdbeere, Pflaume und Weintraube jeweils keinen der Indizes
1, 4, 5 besitzen können, weil diese ausschließlich den Obstsorten aus der Menge $F_1$ zugewiesen werden.
Somit stellen wir fest, dass die Anzahl der möglichen Zuweisungen schrumpft.
Deshalb dürfen wir alle Kanten in $G$ zwischen allen $x \in F_1$ und allen $y \in B \setminus Z_1$,
sowie zwischen allen $p \in A \setminus F_1$ und allen $q \in Z_1$ entfernen (Abb. \ref{fig:example0-2}).\\

Wir analysieren die 2. Spießkombination:
\minibox[frame]{$F_2 = \{\text{Banane, Pflaume, Weintraube}\}, Z_2 = \{3, 5, 6\}$}.\\
Wir stellen fest, dass Banane, Pflaume, Weintraube jeweils keinen der Indizes 1, 2, 4 haben können,
weil jeder dieser Obstsorten ein Index aus der Menge $Z_2$ zugewiesen wird. 
Wir entfernen alle Kanten in $G$ zwischen allen $x \in F_2$ und allen $y \in B \setminus Z_2$.
Unter anderem wurden die Kanten (Banane, 1) und\\ (Banane, 4) entfernt.
Wir stellen fest, dass die Kardinlität des Knotens „Banane“ 1 beträgt --- er ist nur mit dem Knoten 5 verbunden.
Ebenfalls ist der Knoten 5 nur mit dem Knoten „Banane“ verbunden.
Dies bedeutet, dass es nur eine einzige Möglichkeit gibt, diesen Knoten mit einem Index zu verbinden.
Somit wurde der Index von Banane gefunden. Wir kennen schon eine Obstsorte: $o(\text{Banane}, 5)$.\\
Wir stellen auch fest, dass Apfel, Brombeere und Erdbeere jeweils keinen der Indizes 3, 5, 6 haben können,
da sie nur den Obstsorten aus $F_2$ zugewiesen werden dürfen.
Insbesondere wissen wir schon, dass 5 mit Banane verbunden wird.
Deshalb dürfen wir alle Kanten in $G$ zwischen allen $p \in A \setminus F_2$ und allen $q \in Z_2$ entfernen.
Wir bemerken, dass die Kardinlität des Knotens „Erdbeere“ nun 1 beträgt, der nur mit dem Knoten 2 verbunden ist.
Ebenfalls ist der Knoten 2 mit keinem anderen verbunden. So steht fest:\\ $o$(Erdbeere, 2).
Auf der Abbildung \ref{fig:example0-3} wird der Graph $G$ nach der Verarbeitung der 2. Spießkombination
dargestellt. Die fetten Kanten zeigen an, dass die zwei Endknoten bereits verbunden sind, also, dass 
diesen Obstsorten ihre Indizes zugewiesen wurden.\\

Wir analysieren die 3. Spießkombination:
\minibox[frame]{$F_3 = \{\text{Apfel, Brombeere, Erdbeere}\}, Z_3 = \{1, 2, 4\}$}.\\
An dieser Stelle wurde die Zuweisung für Erdbeere bereits gefunden.
Die Knoten „Apfel“ und „Brombeere“ besitzen keine Kanten mehr als die Kanten, die sie jeweils mit 1 und 4 verbinden. 
Ebenfalls wurde der Index 5 Banane zugewiesen.
Die Knoten „Pflaume“ und „Weintraube“ sind jeweils nur mit 3 und 6 verbunden.
So können wir keine der übrigen Kanten zwischen irgendwelchen zwei Knoten in $G$ entfernen.\\

Wir analysieren die 4. Spießkombination:
\minibox[frame]{$F_4 = \{\text{Erdbeere, Pflaume}\}, Z_4 = \{2, 6\}$}.\\
In der Menge $F_4$ tritt Erdbeere auf, deren Index bereits gefunden wurde.
Somit wissen wir, dass der Index von Pflaume 6 sein muss. 
So entdecken wir eine neue Zuweisung: $o$(Pflaume, 6).
Wir können deshalb alle übrigen Kanten zwischen Pflaume und allen anderen Knoten entfernen.
So bleibt der Knoten „Weintraube“ mit nur einer Kante übrig.
Der einzelne Nachbar von diesem Knoten ist 3. So entdecken wir wieder eine neue Zuweisung: $o$(Weintraube, 3).\\
Wir stellen auch fest, dass keine Kanten mehr entfernt werden können.
Es bleibt immer noch ein Paar von Indizes und ein Paar von Obstsorten ohne eindeutige Zuweisung:
\{Apfel, Brombeere\} und $\{1, 4\}$.\\

An dieser Stelle schauen wir die Wunschliste an: \framebox{Apfel, Brombeere, Weintraube}.\\
Der Index von Weintraube ist erfolgreich gefunden, aber die Indizes der übrigen Obstsorten nicht.
Allerdings soll die Lösung der Aufgabe eine Menge an Indizes der gewünschten Obstsorten sein ---
es müssen keine konkreten Zuweisungen ausgegeben werden. 
Dies wurde erfolgreich gefunden, da die Indizes 1 und 4 nur Apfel oder Brombeere gehören können, weil
keine anderen Kanten aus den Knoten 1 und 4 führen.
Auf der Abbildung \ref{fig:example0-4} wurden alle gefundenen Zuweisungen durch fette Kanten dargestellt
und alle Wünsche mit ihren Indizes wurden entsprechend \colorbox{black!30!green}{\textcolor{white}{grün}} und \colorbox{black!5!blue}{\textcolor{white}{blau}} markiert.

\begin{figure}[H]
\centering
\begin{adjustbox}{minipage=\linewidth,scale=0.85}
\begin{subfigure}[t]{.24\textwidth}
\centering
\begin{tikzpicture}
    \node[vertex] (1) {$A$};
    \node[vertex] (2) [below = 0.4cm of 1] {$B$};
    \node[vertex] (3) [below = 0.4cm of 2] {$Br$};
    \node[vertex] (4) [below = 0.4cm of 3] {$E$};
    \node[vertex] (5) [below = 0.4cm of 4] {$P$};
    \node[vertex] (6) [below = 0.4cm of 5] {$W$};
    \node[vertex] (7) [right = 1.5cm of 1] {$1$};
    \node[vertex] (8) [right = 1.5cm of 2] {$2$};
    \node[vertex] (9) [right = 1.5cm of 3] {$3$};
    \node[vertex] (10) [right = 1.5cm of 4] {$4$};
    \node[vertex] (11) [right = 1.5cm of 5] {$5$};
    \node[vertex] (12) [right = 1.5cm of 6] {$6$};

    \path[draw,thick]
    (1) edge node {} (7)
    (1) edge node {} (8)
    (1) edge node {} (9)
    (1) edge node {} (10)
    (1) edge node {} (11)
    (1) edge node {} (12)
    (2) edge node {} (7)
    (2) edge node {} (8)
    (2) edge node {} (9)
    (2) edge node {} (10)
    (2) edge node {} (11)
    (2) edge node {} (12)
    (3) edge node {} (7)
    (3) edge node {} (8)
    (3) edge node {} (9)
    (3) edge node {} (10)
    (3) edge node {} (11)
    (3) edge node {} (12)
    (4) edge node {} (7)
    (4) edge node {} (8)
    (4) edge node {} (9)
    (4) edge node {} (10)
    (4) edge node {} (11)
    (4) edge node {} (12)
    (5) edge node {} (7)
    (5) edge node {} (8)
    (5) edge node {} (9)
    (5) edge node {} (10)
    (5) edge node {} (11)
    (5) edge node {} (12)
    (6) edge node {} (7)
    (6) edge node {} (8)
    (6) edge node {} (9)
    (6) edge node {} (10)
    (6) edge node {} (11)
    (6) edge node {} (12);
\end{tikzpicture}

\caption{}
\label{fig:example0-1}
\end{subfigure}\hfill
\begin{subfigure}[t]{.24\textwidth}
\centering
\begin{tikzpicture}
    \node[vertex] (A) {$A$};
    \node[vertex] (B) [below = 0.4cm of A] {$B$};
    \node[vertex] (Br) [below = 0.4cm of B] {$Br$};
    \node[vertex] (E) [below = 0.4cm of Br] {$E$};
    \node[vertex] (P) [below = 0.4cm of E] {$P$};
    \node[vertex] (W) [below = 0.4cm of P] {$W$};
    \node[vertex] (1) [right = 1.5cm of A] {$1$};
    \node[vertex] (2) [right = 1.5cm of B] {$2$};
    \node[vertex] (3) [right = 1.5cm of Br] {$3$};
    \node[vertex] (4) [right = 1.5cm of E] {$4$};
    \node[vertex] (5) [right = 1.5cm of P] {$5$};
    \node[vertex] (6) [right = 1.5cm of W] {$6$};

    \path[draw,thick]
    (A) edge node {} (1)
    (A) edge node {} (4)
    (A) edge node {} (5)
    (B) edge node {} (1)
    (B) edge node {} (4)
    (B) edge node {} (5)
    (Br) edge node {} (1)
    (Br) edge node {} (4)
    (Br) edge node {} (5)
    (E) edge node {} (2)
    (E) edge node {} (3)
    (E) edge node {} (6)
    (P) edge node {} (2)
    (P) edge node {} (3)
    (P) edge node {} (6)
    (W) edge node {} (2)
    (W) edge node {} (3)
    (W) edge node {} (6);
\end{tikzpicture}
\caption{}
\label{fig:example0-2}
\end{subfigure}
\begin{subfigure}[t]{.24\textwidth}
\centering
\input{./tex/tikz/2.spiess.tex}
\caption{}
\label{fig:example0-3}
\end{subfigure}\hfill
\begin{subfigure}[t]{.24\textwidth}
\centering
\begin{tikzpicture}
    \node[vertex,text=white,fill=black!30!green] (A) {$A$};
    \node[vertex] (B) [below = 0.4cm of A] {$B$};
    \node[vertex,text=white,fill=black!30!green] (Br) [below = 0.4cm of B] {$Br$};
    \node[vertex] (E) [below = 0.4cm of Br] {$E$};
    \node[vertex] (P) [below = 0.4cm of E] {$P$};
    \node[vertex,text=white,fill=black!30!green] (W) [below = 0.4cm of P] {$W$};
    \node[vertex,text=white,fill=black!5!blue] (1) [right = 1.5cm of A] {$1$};
    \node[vertex] (2) [right = 1.5cm of B] {$2$};
    \node[vertex,text=white,fill=black!5!blue] (3) [right = 1.5cm of Br] {$3$};
    \node[vertex,text=white,fill=black!5!blue] (4) [right = 1.5cm of E] {$4$};
    \node[vertex] (5) [right = 1.5cm of P] {$5$};
    \node[vertex] (6) [right = 1.5cm of W] {$6$};

    \path[draw,thick]
    (A) edge node {} (1)
    (A) edge node {} (4)
    (B) edge[line width=2pt] node {} (5)
    (Br) edge node {} (1)
    (Br) edge node {} (4)
    (E) edge[line width=2pt] node {} (2)
    (P) edge[line width=2pt] node {} (6)
    (W) edge[line width=2pt] node {} (3);
\end{tikzpicture}
\caption{}
\label{fig:example0-4}
\end{subfigure}
\end{adjustbox}
\caption{}
\label{fig:example0}
\end{figure}



\subsection{Beispiel 1 (BWINF)}\label{example:1}
Textdatei: \texttt{spiesse1.txt}\\

\noindent
Wünsche: \framebox{Clementine, Erdbeere, Grapefruit, Himbeere, Johannisbeere}\\

\noindent
\framebox{1, 2, 4, 5, 7}

\subsection{Beispiel 2 (BWINF)}\label{example:2}
Textdatei: \texttt{spiesse2.txt}
\vspace{0.25cm}

\noindent
Wünsche: \framebox{Apfel, Banane, Clementine, Himbeere, Kiwi, Litschi}
\vspace{0.25cm}

\noindent
\framebox{1, 5, 6, 7, 10, 11}

\subsection{Beispiel 3 (BWINF)}\label{example:3}
Textdatei: \texttt{spiesse3.txt}\\

\noindent
Wünsche: \framebox{Clementine, Erdbeere, Feige, Himbeere, Ingwer, Kiwi, Litschi}\\

\noindent
\framebox{Dieses Beispiel ist unlösbar.}
\begin{verbatim}
Für die folgenden Obstsorten konnte keine eindeutige Zuweisung gefunden werden.
Komponente: Grapefruit Litschi 
	--> Nicht auf der Wunschliste: Grapefruit 
\end{verbatim}

\subsection{Beispiel 4 (BWINF)}\label{example:4}
Textdatei: \texttt{spiesse4.txt}
\vspace{0.25cm}

\noindent
Wünsche: \framebox{Apfel, Feige, Grapefruit, Ingwer, Kiwi, Nektarine, Orange, Pflaume}
\vspace{0.25cm}

\noindent
\framebox{2, 6, 7, 8, 9, 12, 13, 14}

\newpage
\subsection{Beispiel 5 (BWINF)}\label{example:5}
Textdatei: \texttt{spiesse5.txt}
\vspace{0.25cm}

\noindent
Wünsche: \minibox[frame]{Apfel, Banane, Clementine, Dattel, Grapefruit, Himbeere, Mango,\\ Nektarine, Orange, Pflaume, Quitte, Sauerkirsche, Tamarinde}
\vspace{0.25cm}

\noindent
\framebox{1, 2, 3, 4, 5, 6, 9, 10, 12, 14, 16, 19, 20}

\subsection{Beispiel 6 (BWINF)}\label{example:6}
Textdatei: \texttt{spiesse6.txt}\\

\noindent
\framebox{Clementine, Erdbeere, Himbeere, Orange, Quitte, Rosine, Ugli, Vogelbeere}\\

\noindent
\framebox{4, 6, 7, 10, 11, 15, 18, 20}

\subsection{Beispiel 7 (BWINF)}\label{example:7}
Textdatei: \texttt{spiesse7.txt}\vspace{0.25cm}

\noindent
Wünsche: \minibox[frame]{Apfel, Clementine, Dattel, Grapefruit, Mango, Sauerkirsche, Tamarinde, Ugli,\\ Vogelbeere, Xenia, Yuzu, Zitrone}
\vspace{0.25cm}

\noindent
\framebox{Dieses Beispiel ist unlösbar.}
\begin{verbatim}
Für die folgenden Obstsorten konnte keine eindeutige Zuweisung gefunden werden.
Komponente: Apfel Grapefruit Litschi Xenia 
	--> Nicht auf der Wunschliste: Litschi 
Komponente: Banane Ugli 
	--> Nicht auf der Wunschliste: Banane 
\end{verbatim}


\subsection{Beispiel 8}\label{example:8}
Textdatei: \texttt{spiesse8.txt}\vspace{10pt}\\
\texttt{
\noindent
4\\
Dattel Apfel Banane\\
3\\
1 2 3\\
Apfel Banane Dattel\\
1 2\\
Apfel Banane\\
1 3\\
Clementine Dattel}\\

\noindent
Wünsche: \framebox{Dattel, Apfel, Banane}\\

\noindent
\framebox{Error: Es gibt Fehler in der Eingabedatei.}\\

Die erste Spießkombination legt fest, dass Apfel, Banane und Dattel einen der folgenden Indizes
besitzen: $\{1, 2, 3\}$.  Das bedeutet auch, dass Clementine --- die einzelne übrige Obstsorte ---
den Index 4 besitzt. Jedoch widerspricht dieser Zuweisung die dritte Spießkombination,
laut der Clementine den Index 1 oder 3 hat. 

\newpage
\subsection{Beispiel 9}\label{example:9}
Textdatei: \texttt{spiesse9.txt}\\
Besonderheit: Ein Beispiel für den Fall \ref{probleme2} $\rightarrow$ \ref{fall2}, s. Teil \ref{sec:korrektheit-eingabe}.
\begin{verbatim}
5
Apfel Erdbeere Banane
2
2 3
Banane Clementine
4 5
Banane Dattel
\end{verbatim}

\noindent
Wünsche: \framebox{Apfel, Erdbeere, Banane}
\vspace{0.25cm}

\noindent
\framebox{Error: Es gibt Fehler in der Eingabedatei.}
\vspace{0.25cm}

Die erste Spießkombination legt fest, dass nur Banane und Clementine die Indizes 2 und 3 besitzen dürfen.
Allerdings sollen die Indizes 4 und 5 nach der zweiten Spießkombination den Obstsorten Banane und Dattel 
gehören. Die Obstsorte Banane darf keine zwei unterschiedliche Indizes besitzen.
Deshalb kommt es zu einem Widerspruch --- das Axiom \ref{ax:obstsorte-index} ist verletzt.
\subsection{Beispiel 10}\label{example:10}
Textdatei: \texttt{spiesse10.txt}\\
Besonderheit: ein worst-case, in dem alle $m = 28$ Spießkombinationen alle $n = 26$ Obstsorten beinhalten. Die Wunschliste beinhaltet alle $n$ Obstsorten.\\

\noindent
Wünsche: \minibox[frame]{Apfel, Banane, Clementine, Dattel, Erdbeere, Feige, Grapefruit, Himbeere, Ingwer,\\
Johannisbeere, Kiwi, Litschi, Mango, Nektarine, Orange, Pflaume, Quitte, Rosine,\\
Sauerkirsche, Tamarinde, Ugli, Vogelbeere, Weintraube, Xenia, Yuzu, Zitrone}\\
\vspace{7pt}

\noindent
\minibox[frame]{1, 2, 3, 4, 5, 6, 7, 8, 9, 10, 11, 12, 13, 14, 15, 16, 17, 18, 19, 20, 21, 22, 23, 24, 25, 26}

\subsection{Beispiel 11}\label{example:11}
Textdatei: \texttt{spiesse11.txt}\\
Besonderheit: ein worst-case, in dem alle $m = 28$ Spießkombinationen alle $n = 26$ Obstsorten beinhalten. Die Wunschliste beinhaltet 24 Obstsorten.
\vspace{0.25cm}

\noindent
Wünsche: \minibox[frame]{Apfel, Banane, Clementine, Dattel, Feige, Grapefruit, Himbeere, Ingwer,\\
Johannisbeere, Kiwi, Litschi, Mango, Nektarine, Orange, Pflaume, Rosine,\\
Sauerkirsche, Tamarinde, Ugli, Vogelbeere, Weintraube, Xenia, Yuzu, Zitrone}\\
\vspace{0.25cm}

\noindent
\framebox{Dieses Beispiel ist unlösbar.}
\begin{verbatim}
Für die folgenden Obstsorten konnte keine eindeutige Zuweisung gefunden werden.
Komponente: Apfel Banane Clementine Dattel Erdbeere Feige Grapefruit Himbeere Ingwer
Johannisbeere Kiwi Litschi Mango Nektarine Orange Pflaume Quitte Rosine Sauerkirsche
Tamarinde Ugli Vogelbeere Weintraube Xenia Yuzu Zitrone 
	--> Nicht auf der Wunschliste: Erdbeere Quitte
\end{verbatim}

\newpage
\subsection{Beispiel 12}\label{example:12}
Textdatei: \texttt{spiesse12.txt}\\
Besonderheit: Es entstehen $\frac{n}{2}$ Zusammenhangskomponenten in $G$. Keiner Obstsorte kann ein Index eindeutig zugeordnet werden.
\begin{verbatim}
8
Feige Dattel Clementine Grapefruit
4
5 1 3 2 8 6
Dattel Banane Feige Erdbeere Clementine Himbeere
1 5 8 6
Himbeere Dattel Feige Banane
3 2 7 4
Apfel Clementine Erdbeere Grapefruit
8 7 1 4
Banane Apfel Grapefruit Dattel
\end{verbatim}

\noindent
Wünsche: \framebox{Clementine, Dattel, Feige, Grapefruit}\\

\noindent
\framebox{Dieses Beispiel ist unlösbar.}
\begin{verbatim}
Für die folgenden Obstsorten konnte keine eideutige Zuweisung gefunden werden.
Komponente: Clementine Erdbeere 
	--> Nicht auf der Wunschliste: Erdbeere 
Komponente: Banane Dattel 
	--> Nicht auf der Wunschliste: Banane 
Komponente: Feige Himbeere 
	--> Nicht auf der Wunschliste: Himbeere 
Komponente: Apfel Grapefruit 
	--> Nicht auf der Wunschliste: Apfel
\end{verbatim}
\subsection{Beispiel 13}\label{example:13}
Textdatei: \texttt{spiesse13.txt}\\
Besonderheit: Es gibt $m = n$ Spießkombinationen. Die Mächtigkeiten der Mengen dieser Spießkombinationen entsprechen:
$n, n-1, ..., 1$. Die Spießkombinationen werden aufeinander aufgebaut.
\begin{verbatim}
8
Apfel Himbeere Grapefruit Banane
8
5 1 3 2 4 8 6 7 
Dattel Grapefruit Banane Feige Erdbeere Clementine Apfel Himbeere
1 5 4 6 8 2 7
Himbeere Dattel Feige Banane Apfel Gp0rapefruit Clementine
4 8 2 5 1 7 
Apfel Dattel Clementine Himbeere Banane Grapefruit
8 7 2 4 1
Clementine Banane Apfel Grapefruit Dattel
1 4 2 7
Grapefruit Clementine Dattel Apfel
7 2 1
Dattel Grapefruit Clementine
2 7
Clementine Grapefruit
2
Clementine
\end{verbatim}

\noindent
Wünsche: \framebox{Apfel, Himbeere, Grapefruit, Banane}
\vspace{0.25cm}

\noindent
\framebox{4, 5, 7, 8}
\subsection{Beispiel 14}\label{example:14}
Textdatei: \texttt{spiesse14.txt}\\
Besonderheit: Ein Beispiel für den Fall \ref{probleme1} $\rightarrow$ \ref{fall2}, s. Teil \ref{sec:korrektheit-eingabe}.
\begin{verbatim}
6
Apfel Clementine Feige
2
1 4 5
Apfel Banane Dattel
4 5 6
Dattel Erdbeere Feige
\end{verbatim}

\noindent
Wünsche: \framebox{Apfel, Clementine, Feige}
\vspace{0.25cm}

\noindent
\framebox{Error: Es gibt Fehler in der Eingabedatei.}
\vspace{0.25cm}

Die erste Spießkombination legt fest, dass die Obstsorten Apfel, Banane, Dattel
jeweils einen der Indizes $\{1, 4, 5\}$ haben können. Dennoch die zweite Spießkombination 
legt im Widerspruch zur ersten fest, dass die Indizes $\{4, 5, 6\}$ jeweils einer der
Obstsorten Dattel, Erdbeere, Feige gehören können. In diesem Fall decken sich 4 und Dattel
in beiden Spießkombinationen, aber 5 kann keiner Obstsorte zugeordnet werden.
\subsection{Beispiel 15}\label{example:15}
Textdatei: \texttt{spiesse15.txt}\\
Besonderheit: Die obere Schranke an Obstsorten ist sehr groß im Vergleich zu der Anzahl der 
in den Spießkombinationen und in der Wunschliste erwähnten Obstsorten. Dieses Beispiel betont, dass
mein Programm problemlos so eine große obere Schranke behandeln kann. Außerdem zeigt sich hier der
Vorteil der Verwendung von Bitmasken im Vergleich zu einer „Standard“--Adjazenzmatrix.
\begin{verbatim}
26
Apfel Dattel Feige
3
1 2 4
Apfel Banane Dattel
2 3 6
Dattel Erdbeere Feige
1 6
Banane Erdbeere
\end{verbatim}

\noindent
Wünsche: \framebox{Apfel, Dattel, Feige}\\

\noindent
\framebox{2, 3, 4}

\section{Quellcode}
\lstinputlisting[language=C++]{./tex/spiesse.m}

\end{document}
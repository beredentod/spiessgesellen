\documentclass[a4paper,10pt,ngerman]{scrartcl}
\usepackage{babel}
\usepackage[T1]{fontenc}
\usepackage[utf8]{inputenc}
\usepackage{textcomp}
\usepackage[a4paper,margin=2.5cm,footskip=0.5cm]{geometry}

% Die nächsten drei Felder bitte anpassen:
\newcommand{\Aufgabe}{Aufgabe 2: Spießgesellen} % Aufgabennummer und Aufgabennamen angeben
\newcommand{\TeilnahmeId}{55628}       % Teilnahme-Id angeben
\newcommand{\Namen}{Michal Boron} % Namen der Bearbeiter/-innen dieser Aufgabe angeben
 
% Kopf- und Fußzeilen
\usepackage{scrlayer-scrpage, lastpage}
\setkomafont{pageheadfoot}{\large\textrm}
\lohead{\Aufgabe}
\rohead{Teilnahme-Id: \TeilnahmeId}
\cfoot*{\thepage{}/\pageref{LastPage}}

% Position des Titels
\usepackage{titling}
\setlength{\droptitle}{-1.0cm}
\usepackage{seqsplit}
\usepackage{verbatim}

% Für mathematische Befehle und Symbole
\usepackage{amsmath}
\usepackage{amssymb}
%\usepackage{cite}

\usepackage[backend=bibtex]{biblatex}
\addbibresource{stromrallye.bib}

\usepackage{hyperref}
\hypersetup{
    colorlinks=false,
    linkcolor=blue,
    filecolor=magenta,      
    urlcolor=cyan,
}
% Für Bilder
\usepackage{graphicx}
\usepackage[all]{xy}
\usepackage{svg}
\graphicspath{ {./images/} }

% Für Algorithmen
\usepackage{algpseudocode}
\usepackage{algorithm}
\usepackage{gensymb}
\usepackage{tikz}
\usepackage{caption}
\usepackage{subcaption}

\usepackage[backgroundcolor=lightgray]{todonotes}
\usepackage{minibox}

\usepackage{amsthm}
\usepackage{enumitem}

% Für Quelltext
\usepackage{listings}
\usepackage{color}
\definecolor{mygreen}{rgb}{0,0.6,0}
\definecolor{mygray}{rgb}{0.5,0.5,0.5}
\definecolor{mymauve}{rgb}{0.58,0,0.82}
\lstset{
  keywordstyle=\color{blue},commentstyle=\color{mygreen},
  stringstyle=\color{mymauve},rulecolor=\color{black},
  basicstyle=\footnotesize\ttfamily,numberstyle=\tiny\color{mygray},
  captionpos=b, % sets the caption-position to bottom
  keepspaces=true, % keeps spaces in text
  numbers=left, numbersep=5pt, showspaces=false,showstringspaces=true,
  showtabs=false, stepnumber=2, tabsize=2, title=\lstname
}

% Diese beiden Pakete müssen zuletzt geladen werden
%\usepackage{hyperref} % Anklickbare Links im Dokument
\usepackage{cleveref}
%\newtheorem{lemma}{Lemma}
%\newenvironment{proof}{\paragraph{Beweis:}}{\hfill$\square$}
\newtheorem{lemma}{Lemma}
\newtheorem{definition}{Definition}
\newtheorem{satz}{Satz}
\newtheorem{axiom}{Axiom}
%\renewcommand*{\proofname}{Solution}

\newcommand{\TODO}[1]{\todo[inline]{TODO: #1}}
\newcommand{\mb}[1]{{\color{red}[MB: #1]}}
\newcommand{\tbf}[1]{\textbf{#1}}
\newcommand{\ttt}[1]{\texttt{#1}}

\usetikzlibrary{fit,backgrounds,positioning}
\tikzset{vertex/.style={circle,draw,minimum size=0.8cm,inner sep=1pt,fill=white}}

% Daten für die Titelseite
\title{\textbf{\Huge\Aufgabe}}
\author{\LARGE Teilnahme-Id: \LARGE \TeilnahmeId \\\\
	    \LARGE Bearbeiter dieser Aufgabe: \\ 
	    \LARGE \Namen\\\\}
\date{\LARGE April 2021}

\begin{document}

\maketitle
\tableofcontents

\section{Lösungsidee}
\subsection{Formulierung des Problems}\label{sec:formulierung}
\begin{axiom}\label{ax:obstsorte-index}
Jeder \textbf{Obstsorte} wird genau ein einzigartiger natürlicher Index zugewiesen.

Man schreibt: $o(x, i)$ --- eine Obstsorte $x$ besitzt einen Index $i$.
\end{axiom}


Gegeben sind eine Menge von $n$ Obstsorten $A$ und eine Menge von $n$ ganzen Zahlen
$B = \{1, 2, ..., n\}$, zu der die Indizes der Obstsorten aus $A$ gehören.

\begin{definition}[Spießkombination]\label{def:spiesskomb}
Als eine \textbf{Spießkombination} $K = (F, Z)$ bezeichnet man eine Veknüpfung von zwei Mengen 
$F \subseteq A$ und $Z$, wobei $Z = \{i \in B \,|\, \forall x \in F : o(x, i)\}.$
\end{definition}


Gegeben sind auch $m$ Spießkombinationen, wobei jede $i$--te Spießkombination
aus einer Menge von Obstsorten $F_i \subseteq A$ und einer Menge der Indizes $Z_i \subseteq B$ besteht. 
Nach der Definition \ref{def:spiesskomb} besteht die Menge $Z_i$ nur aus
den in $B$ enthaltenen Indizes, die zu den Obstsorten in $F_i$ gehören, deshalb sind die beiden Mengen
$F_i$ und $Z_i$ auch gleichmächtig.\\
\indent Außerdem gegeben ist auch eine \textit{\textbf{Wunschliste}} $W \subseteq A$.\\

Die Aufgabe ist ein Entscheidungsproblem. Es soll entschieden werden,
ob die Menge der Indizes der in $W$ enthaltenen Obstsorten $W' \subseteq B$ anhand der $m$ 
Spießkombinationen eindeutig bestimmt werden kann. Falls ja, soll sie auch ausgegeben werden.\\

In den folgenden Überlegungen wird angenommen, dass das Axiom \ref{ax:obstsorte-index} für alle
Obstsorten in der Eingabe gilt.
Es ist aber möglich, dass die Spießkombination in einer Eingabe diesem Axiom nicht folgen, das heißt,
es an einer Stelle einen Widerspruch gibt.
Laut der Aufgabenstellung ist ein solcher Fall nicht ausgeschlossen. 
Um diesen Fall zu verhindern, muss man die Korrektheit der Eingabe überprüfen. 
Mehr dazu folgt im Teil \ref{sec:korrektheit-eingabe}.


\subsection{Bipartiter Graph}
Man kann die beiden Mengen $A$ und $B$ zu Knoten eines bipartiten Graphen $G = (A \cup B = V, E)$ umwandeln.
Die Menge der Kanten $E$ wird im Folgenden festgelegt.
Man stellt den Graphen als eine Adjazenzmatrix $M$ der Größe $n \times n$ dar. 
Als $M_i$ bezeichnet wird die Liste der Länge $n$,
die die Beziehungen des Knotens
$i \in A$ zu jedem Knoten $j \in B$ als 1 (Kante) oder 0 (keine Kante) enthält.
Als $M_{i, j}$ bezeichnet wird die $j$--te Stelle in der $i$--ten Liste der Matrix.

Nach Axiom \ref{ax:obstsorte-index} gehört zu jeder Obstsorte aus $A$ genau ein Index aus $B$.
Dennoch kann man am Anfang keiner Obsorte einen Index zuweisen.
Deshalb wird zunächst jeder Knoten aus $A$ mit jedem Knoten aus $B$ durch eine Kante verbunden:
\[
E = A\times B = \{(x, y) \mid  x \in A \text{ und } y \in B\}.
\]

\begin{figure}[H]
\centering
\begin{subfigure}[b]{.49\textwidth}
\centering
\begin{tikzpicture}
    \node[vertex] (A) {$A$};
    \node[vertex] (B) [below = 0.4cm of A] {$B$};
    \node[vertex] (Br) [below = 0.4cm of B] {$Br$};
    \node[vertex] (E) [below = 0.4cm of Br] {$E$};
    \node[vertex] (P) [below = 0.4cm of E] {$P$};
    \node[vertex] (W) [below = 0.4cm of P] {$W$};
    \node[vertex] (1) [right = 1.5cm of A] {$1$};
    \node[vertex] (2) [right = 1.5cm of B] {$2$};
    \node[vertex] (3) [right = 1.5cm of Br] {$3$};
    \node[vertex] (4) [right = 1.5cm of E] {$4$};
    \node[vertex] (5) [right = 1.5cm of P] {$5$};
    \node[vertex] (6) [right = 1.5cm of W] {$6$};
    \begin{scope}[on background layer]
        \node[draw=blue!20,fill=blue,fill opacity=0.2,fit=(A) (B) (Br) (E) (P) (W)] [label=left:A] {};
        \node[draw=red!20,fill=red,fill opacity=0.2,fit=(1) (2) (3) (4) (5) (6)] [label=right:B] {};
    \end{scope}
\end{tikzpicture}

\caption{Die entsprechenden Mengen des Graphen}
\label{fig:graph-anfang}
\end{subfigure}
\begin{subfigure}[b]{.49\textwidth}
\centering
\begin{tikzpicture}
    \node[vertex] (1) {$A$};
    \node[vertex] (2) [below = 0.4cm of 1] {$B$};
    \node[vertex] (3) [below = 0.4cm of 2] {$Br$};
    \node[vertex] (4) [below = 0.4cm of 3] {$E$};
    \node[vertex] (5) [below = 0.4cm of 4] {$P$};
    \node[vertex] (6) [below = 0.4cm of 5] {$W$};
    \node[vertex] (7) [right = 1.5cm of 1] {$1$};
    \node[vertex] (8) [right = 1.5cm of 2] {$2$};
    \node[vertex] (9) [right = 1.5cm of 3] {$3$};
    \node[vertex] (10) [right = 1.5cm of 4] {$4$};
    \node[vertex] (11) [right = 1.5cm of 5] {$5$};
    \node[vertex] (12) [right = 1.5cm of 6] {$6$};

    \path[draw,thick]
    (1) edge node {} (7)
    (1) edge node {} (8)
    (1) edge node {} (9)
    (1) edge node {} (10)
    (1) edge node {} (11)
    (1) edge node {} (12)
    (2) edge node {} (7)
    (2) edge node {} (8)
    (2) edge node {} (9)
    (2) edge node {} (10)
    (2) edge node {} (11)
    (2) edge node {} (12)
    (3) edge node {} (7)
    (3) edge node {} (8)
    (3) edge node {} (9)
    (3) edge node {} (10)
    (3) edge node {} (11)
    (3) edge node {} (12)
    (4) edge node {} (7)
    (4) edge node {} (8)
    (4) edge node {} (9)
    (4) edge node {} (10)
    (4) edge node {} (11)
    (4) edge node {} (12)
    (5) edge node {} (7)
    (5) edge node {} (8)
    (5) edge node {} (9)
    (5) edge node {} (10)
    (5) edge node {} (11)
    (5) edge node {} (12)
    (6) edge node {} (7)
    (6) edge node {} (8)
    (6) edge node {} (9)
    (6) edge node {} (10)
    (6) edge node {} (11)
    (6) edge node {} (12);
\end{tikzpicture}

\caption{Der Graph am Anfang}
\label{fig:graph-full}
\end{subfigure}
\caption{Beide Abbildungen stellen den Graphen für das Beispiel aus der Aufgabenstellung dar.\\
Die Buchstaben stehen für die entsprechenden Obstsorten aus diesem Beispiel (s. auch \ref{example:0}).}
\end{figure}


Am Anfang ist $M$ dementsprechend voll mit Einsen.
Bei der Erstellung der Adjazenzmatrix kann man den Vorteil nutzen, dass die 
Liste der Nachbarn eines Knotens $x \in A$ nur aus Nullen und Einsen besteht, indem man
diese Liste als eine Bitmaske darstellt (mehr dazu in der \nameref{sec:umsetzung}).

Jede $i$--te Spießkombination $K_i = (F_i, Z_i)$ bringt Informationen über die Obstsorten in $F_i$.
%Als $a \leadsto b$ bezeichnet man, dass $a$ den Index $b$ haben kann. 
Man kann Folgendes festellen. 

\begin{lemma} \label{lem:spiess-numbers}
Sei $K = (F, Z)$ eine Spießkombination. Für jede Obstsorte $o(x, i)$, wobei $x \in F$, gilt:
\begin{enumerate}[label={\upshape(\roman*)}]
  %\item $\forall x \in F\, \forall y \in Z: x \leadsto y$
  %\item $\nexists p \in F\, \forall q \in B \setminus Z: p \leadsto q$.
  \item $i \in Z$, \label{lem:spiess-numbers1}
  \item $i \notin B \setminus Z$. \label{lem:spiess-numbers2}
\end{enumerate}   
\end{lemma}

\begin{proof}
Nach Definition \ref{def:spiesskomb} gilt \ref{lem:spiess-numbers1}. 
Nach Axiom \ref{ax:obstsorte-index} besitzt jede Obstsorte einen einzigartigen Index $i$,
deshalb kann $i$ nicht gleichzeitig zu $Z$ und $B \setminus Z$ gehören \ref{lem:spiess-numbers2}.
\end{proof}


\begin{definition}[Zusammenhangskomponente]\label{def:komponente}
Ein ungerichteter Graph $\mathcal{G} = (\mathcal{V}, \mathcal{E})$ heißt zusammenhängend, wenn es von jedem Knoten $u$ zu jedem anderen Knoten $v$ mindestens einen Pfad gibt.
Ein maximaler zusammenhängender Teilgraph eines ungerichteten Graphen $\mathcal{G}$ heißt \textbf{Zusammenhangskomponente} $C = (V_c \subseteq \mathcal{V}, E_c \subseteq \mathcal{E})$ von $\mathcal{G}$. 
\end{definition}


\noindent Aus Lemma \ref{lem:spiess-numbers} ergibt sich direkt auch eine andere Beobachtung.

\begin{korollar}\label{kor:komponente-mengen}
Sei $C = (L_c \cup R_c, E_c)$ eine Zusammenhangskomponente in $G$.
Sei $K = (F, Z)$ eine Spießkombiantion.
%Falls $F \subsetneq L_c$ gilt, dann:
Falls $F \subseteq L_c$ gilt, dann gilt für jede Obstsorte $o(x, i)$, wobei $x \in F$:
\begin{enumerate}[label={\upshape(\roman*)}]
	\item $i \in Z$,
	\item $i \notin R_c \setminus Z$.
  %\item Für jede Obstsorte $o(p, i)$, wobei $p \in F$, gilt: $i \in Z \land i \notin R_c \setminus Z$,\label{lem:komponente-mengen1}
  %\item Für jede Obstsorte $o(q, j)$, wobei $q \in L_c \setminus F$, gilt: 
  %$(j \in R_c \setminus Z) \land j \notin Z$.\label{lem:komponente-mengen2}
\end{enumerate}
Deshalb werden alle Kanten, die aus jedem Knoten $x \in F$ 
zu jedem Knoten $y \in R_c \setminus Z$ führen,\\ aus $E$ entfernt.
\end{korollar}

\begin{comment}
\begin{proof}
\ref{lem:komponente-mengen1} gilt nach Defintion \ref{def:spiesskomb}, Axiom \ref{ax:obstsorte-index}
und Lemma \ref{lem:spiess-numbers}.\\
\ref{lem:komponente-mengen2} gilt aus dem Grund, dass $L_c$ und $R_c$ gleichmächtig sind.
(Sonst, könnte man nicht allen $x \in A$ einen $y \in B$ zuweisen.)
\TODO{Beweis zu Ende}
\end{proof}
\end{comment}





\subsection{Logik}\label{sec:logik}

Betrachten wir eine Spießkombination $s = (F_s, Z_s)$.
%die aus den Mengen $F_s \subseteq A$ und $Z_s \subseteq B$ besteht. 
Wir erstellen 3 Bitmasken $bf, bn$ und $br$ jeweils der Länge $n$.
Die Bitmaske $bf$ besteht aus $n$ 1--en.
In der Maske $bn$ stehen die 1--Bits an allen Stellen, die den Indizes in $Z_s$ entsprechen.
Die Bitmaske $br$ wird auf folgende Weise definiert:
\[
br := \neg(bn) \land bf.
\]
\noindent So können wir auf allen Listen $M_i$, wobei $i \in F_s$, die AND--Operation mit der Maske $bn$ 
durchführen:
\[
M_i := M_i \land bn.
\]
Analog führen wir die AND--Operation mit der Maske $br$ auf allen Listen $M_j$,
wobei $j \in A \setminus F_s$, durch:
\[
M_j := M_j \land br.
\]

\begin{figure}[H]
\vspace{-0.7cm}
\caption{Beide Abbildungen stellen die Adjazenzmatrix für das Beispiel aus der Aufgabenstellung dar.
Die Buchstaben in der ersten Spalte stehen für die entsprechenden Obstsorten
und die Zahlen in der ersten Zeile stehen für die Indizes aus demselben Beispiel (s. auch \ref{example:0}).\\
Auf der Abb. \ref{fig:matrix-danach} stehen $bn$ und $br$ für die entsprechenden Bitmasken.}
\begin{subfigure}[b]{.39\textwidth}
\centering
\begin{tabular}{>{\itshape}l|c|c|c|c|c|c|}
 & 6 & 5 & 4 & 3 & 2 & 1 \\ \hline
A & 0 & 1 & 1 &0 & 0 & 1 \\ \hline 
B & 0 & 1 & 1 &0 & 0 & 1 \\ \hline 
Br & 0 & 1 & 1 &0 & 0 & 1 \\ \hline 
E & 1 & 0 & 0 & 1 & 1 & 0 \\ \hline 
P & 1 & 0 & 0 & 1 & 1 & 0 \\ \hline 
W & 1 & 0 & 0 & 1 & 1 & 0 \\ \hline 
\end{tabular}
\caption{$M$ vor der neuen Spießkombination}
\label{fig:matrix-anfang}
\end{subfigure}
\begin{subfigure}[b]{.59\textwidth}
\vspace{0.25cm}
\begin{tabular}{lll}
Spießkombination: & F =&\{Banane, Pflaume, Weintraube\} \\
 & Z =&\{3, 5, 6\} \\
\end{tabular}\\
\centering

\begin{tabular}{>{\itshape}l|c|c|c|c|c|c|}
 & 6 & 5 & 4 & 3 & 2 & 1 \\ \hline
\cellcolor{lightblue}bn & 1 & 1 & 0 & 1 & 0 & 0 \\ \hline
\cellcolor{lightred}br & 0 & 0 & 1 & 0 & 1 & 1 \\ \hline
\end{tabular}\\
\vspace{0.5cm}
\begin{tabular}{>{\itshape}l|c|c|c|c|c|c|}
 & 6 & 5 & 4 & 3 & 2 & 1 \\ \hline
\cellcolor{lightred}A & 0 & {\color{red} 0} & 1 &0 & 0 & 1 \\ \hline 
\cellcolor{lightblue}B & 0 & 1 & {\color{red} 0} &0 & 0 & {\color{red} 0} \\ \hline 
\cellcolor{lightred}Br & 0 & {\color{red} 0} & 1 &0 & 0 & 1 \\ \hline 
\cellcolor{lightred}E & {\color{red} 0} & 0 & 0 & {\color{red} 0} & 1 & 0 \\ \hline 
\cellcolor{lightblue}P & 1 & 0 & 0 & 1 & {\color{red} 0} & 0 \\ \hline 
\cellcolor{lightblue}W & 1 & 0 & 0 & 1 & {\color{red} 0} & 0 \\ \hline
\end{tabular}
\caption{$M$ nach der Verarbeitung der beschriebenen Spießkombination.}
\label{fig:matrix-danach}
\end{subfigure}
\end{figure}

Auf der obigen Abbildung werden \colorbox{lightblue}{blau} und \colorbox{lightred}{rot} die entsprechenden
Listen gekennzeichnet, auf denen die AND--Operation mit der entsprechenden Bitmaske durchgefüht wurde.
{\color{red} Rot} werden die Bits gekennzeichnet, die sich nach der Verarbeitung der Spießkombination veränderten.\\

Was die beschriebenen Operationen verursachen, wird anhand der folgenden Fallunterscheidung erläutert.
\begin{enumerate}
  \item Falls es sich um einen Knoten $x \in F_s$ handelt, betrachten wir dazu die entsprechende
  Liste $M_x$ und einen Knoten $y \in B$.
  \begin{enumerate}
   \item Falls der Knoten $y$ zu $Z_s$ gehört, aber an der Stelle $M_{x,y}$ 0 steht, bleibt es auch 0.
   %Allerdings ergibt sich laut Lemma \ref {lem:spiess-numbers} ein Widerspruch.
   %Später wird dieser Widerspruch bei der Prüfung der Korrektheit der Eingabe entdeckt.
   \item Falls der Knoten $y$ zu $Z_s$ gehört und an der Stelle $M_{x,y}$ 1 steht, bleibt es auch 1.
   \item Falls der Knoten $y$ zu $Z_s$ nicht gehört und an der Stelle $M_{x,y}$ 0 steht, bleibt es auch 0.
   \item Falls der Knoten $y$ zu $Z_s$ nicht gehört, aber an der Stelle $M_{x,y}$ 1 steht, 
    wird die Stelle $M_{x,y}$ zu 0.
  \end{enumerate}
  \item Falls es sich um einen Knoten $x \in A \setminus F_s$ handelt, betrachten wir dazu die entsprechende
  Liste $M_x$ und einen Knoten $y \in B$.
  \begin{enumerate}
    \item Falls der Knoten $y$ zu $Z_s$ nicht gehört, aber an der Stelle $M_{x,y}$ 0 steht, bleibt es auch 0.
    %Allerdings ergibt sich laut Lemma \ref {lem:spiess-numbers} ein Widerspruch.
    %Später wird dieser Widerspruch bei der Prüfung der Korrektheit der Eingabe entdeckt.
    \item Falls der Knoten $y$ zu nicht $Z_s$ gehört und an der Stelle $M_{x,y}$ 1 steht, bleibt es auch 1.
    \item Falls der Knoten $y$ zu $Z_s$ gehört, aber an der Stelle $M_{x,y}$ 1, 
      wird die Stelle $M_{x,y}$ zu 0.
    \item Falls der Knoten $y$ zu $Z_s$ gehört und an der Stelle $M_{x,y}$ 0 steht, bleibt es auch 0.
  \end{enumerate}
\end{enumerate}



\begin{comment}
\begin{lemma}\label{lem:neue-komponenten}
Sei $C = (V_c, E_c) \subseteq G$ eine Zusammenhangskomponente.
Sei $K = (F, Z)$ eine Spießkombiantion.
Wir betrachten, was mit $G$ nach der Verarbeitung von $K$ passiert.
\begin{enumerate}[label={\upshape(\roman*)}]
  \item Falls gilt: $F \cup Z = V_c$, dann entsteht keine neue Zusammenhangskomponente in $G$. \label{lem:neue-komponenten1}
  \item Falls gilt: $F \cup Z \subsetneq V_c$,%\, \land\, F \cup Z \not\subset V \setminus V_c$, 
  dann entstehen zwei neue Zusammenhangskomponenten in $G$ --- $C$ wird in zwei Komponenten gespalten.\label{lem:neue-komponenten2}
  \item Seien $C_1, C_2, ..., C_k \subset G$ voneinander unterschiedliche Zusammenhangskomponenten.\\
  Falls $F \cup Z$ aus mehreren Teilmengen aus $C_1, C_2, ..., C_k$ besteht, gelten für jede Komponente $C_i$ ebenfalls \ref{lem:neue-komponenten1} und \ref{lem:neue-komponenten2}.
\end{enumerate}

\end{lemma}

\begin{proof}
\TODO{was machen wir mit dem Beweis?}
Die Beweise für die entsprechenden Punkte:
\begin{enumerate}[label={\upshape(\roman*)}]
  \item Nach der Definition einer Zusammenhangskomponente gilt: $\nexists x \in V_c: x \in V \setminus V_c$.
  Bei Bearbeiteung von $K$ werden nach Lemma \ref{lem:spiess-numbers} alle Kanten zwischen 
  $x \in V \setminus V_c$ und $y \in V_c$ aus $E$ entfernt.
  Deshalb werden bei so einer Spießkombination $K$ keine Kanten entfernt.
  \item blablabla
  %Laut dieser Bedingung gilt: $\exists x \in A \cap V_c: x \notin F$ und
  %$\exists y \in B \cap V_c: y \notin Z$.\\
\end{enumerate}
\end{proof}
\end{comment}

\begin{lemma}\label{lem:komponente-complete}
Sei $C = (V_c, E_c)$ eine beliebige Zusammenhangskomponente in $G$. 
Dann bildet $C$ nach Verarbeitung jeder $k$--ten Spießkombination
selbt einen vollständigen, bipartiten Graphen.
\end{lemma}

\begin{proof}
Diese Aussage kann durch die vollständige Induktion für jedes $k \in \mathbb{N}$ bewiesen werden.\\
\noindent
\tbf{Induktionsanfang}: 
Die beiden Mengen $A$ und $B$ sind gleichmächtig und ganz am Anfang ist $G$ vollständig.
Sei die erste Spießkombination $K_1 = (F_1, Z_1)$, wobei $F_1 \neq A$. (Falls $F_1 = A$,
dann gilt die Aussage sofort für $k = 1$.)
Nach der Verarbeitung von $K_1$ 
entstehen zwei Zusammenhangskomponenten: $C_1 \cup C_2 = G$, wobei o.B.d.A $C_1 = F \cup Z$.
Dann sind $C_1 \cap A$ und $C_1 \cap B$ nach Definition \ref{def:spiesskomb} auch gleichmächtig. 
Ebenfalls sind dann $C_2 \cap A$ und $C_2 \cap B$ gleichmächtig.
Nach \ref{lem:spiess-numbers} gilt, dass alle Kanten zwischen $C_1$ und $C_2$
aus $E$ entfernt wurden, aber alle innerhalb von $C_1$ und innerhalb von $C_2$
beibehalten wurden.
Dies bedeutet, dass die Komponenten $C_1$ und $C_2$ selbst vollständige, bipartite Graphen sind.\\
Damit ist die Aussage für $k = 1$ bewiesen und der Induktionsanfang erledigt.\\

\noindent
\tbf{Induktionsschritt}: Es gelte die Aussage, also die \tbf{Induktionsannahme},
für $k \in \mathbb{N}$, d.h., es gelte,
dass jede Zusammenhangskomponente in $G$ nach Verarbeitung von $k$ Spießkombinationen selbst
einen vollständigen, bipartiten Graphen bildet.\\

Zu zeigen ist die Aussage für $k + 1$, also, dass jede Zusammenhangskomponente in $G$ nach Verarbeitung
von $k + 1$ Spießkombinationen selbst einen vollständigen, bipartiten Graphen bildet.\\

Sei $K_i = (F_i, Z_i)$ die $k+1$--te Spießkombination.
Zu untersuchen ist die folgende Fallunterscheidung:
\begin{enumerate}[label={\upshape(\roman*)}]
  \item Sei $D = (V_D, E_D)$ eine Zusammenhangskomponente in $G$. Sei $F_i \cup Z_i = V_D$.
  Da alle Knoten der Spießkombination sich mit allen Knoten von $D$ decken,
  können, nach Lemma \ref{lem:komponente-mengen}, keine Kanten aus $E$ entfernt werden, deshalb
  entstet keine neue Zusammenhangskomponente, also ist jede Zusammenhangskomponente
  nach der Induktionsannahme ein vollständiger, bipartiter Graph.\label{lem:komponente-complete1} 
  %Damit ist der Induktionsschritt für diesen Fall vollzogen und die Behauptung gilt für jedes
  %$k \in \mathbb{N}$.
  \item Sei $D = (V_D, E_D)$ eine Zusammenhangskomponente in $G$. Sei $F_i \cup Z_i \subsetneq V_D
  \land (F_i \cup Z_i) \not\subset (A \cup B)$, also $F_i \cup Z_i$ gehört nur zu einer
  Zusammenhangskomponente in $G$.
  $D$ ist laut Induktionsannahme selbt ein vollständiger, bipartiter Graph.
  Nach Lemma \ref{lem:komponente-mengen} werden alle Kanten zwischen 
  allen $x \in F_i$ und allen $y \in V_D \cap Z_i$ entfernt.
  So entstehen zwei neue Zusammenhangskomponenten:
  $C_1 = F_i \cup Z_i$ und $C_2 = V_D \setminus (F_i \cup Z_i)$, die ebenfalls selbt 
  vollständige, bipartite Grpahen sind.
  Jede andere Zusammenhangskomponente in $G$ ist
  nach der Induktionsannahme ein vollständiger, bipartiter Graph.\label{lem:komponente-complete2} 
  \item Sei $1 \leqslant p \leqslant n$ beliebig, aber fest.
  Seien $C_1, C_2, ..., C_p$ untereinander unterschiedliche Zusammnhangskomponenten in $G$.
  Gehöre $(F_i \cup Z_i)$ zu mehreren Komponenten $C_p, ..., C_q$.
  Dann gilt für jede Zusammenhangskomponente $C_i$ entweder \ref{lem:komponente-complete1} 
  oder \ref{lem:komponente-complete2}, abhängig davon, ob $C_i$ vollständig zu $F_i \cup Z_i$
  gehört oder nur zum Teil. Das bedeutet, entweder entsteht keine neue Zusammenhangskomponente 
  \ref{lem:komponente-complete1} oder $C_i$ wird in zwei neue Zusammenhangskomponenten gespalten
  \ref{lem:komponente-complete2}.
\end{enumerate}

Da alle mögliche untersucht wurden, ist der Induktionsschritt vollzogen und die Behauptung gilt für jedes
$k \in \mathbb{N}$.
\end{proof}


\subsection{Zusammenhangskomponenten}
Nach der Verarbeitung der allen $m$ Spießkombinationen verfügen wir über den Graphen $G$,
in dem viele Kanten in $E$ entfernet wurden.
Auf diese Weise können wir schon anfangen, die Indizes der Obstsorten aus $W$ festzulegen.
Definieren wir zunächst, was generell ein \tbf{Matching} ist.

\begin{definition}[Matching]\label{def:matching}
Sei $\mathcal{G} = (\mathcal{V}, \mathcal{E})$ ein ungerichteter Graph.
Als ein \textbf{Matching} bezeichnen wir eine Teilmenge $\mathcal{S} \subseteq \mathcal{E}$,
sodass für alle $v \in \mathcal{V}$ gilt, dass höchstens eine Kante 
aus $\mathcal{S}$ inzident zu $v$ ist.\\
Wir bezeichnen einen Knoten $v \in \mathcal{V}$ als in $\mathcal{S}$ \textbf{gematcht},
wenn eine Kante aus $\mathcal{S}$ inzident zu $v$ ist.\\\textnormal{\cite[S.~732]{cormen:matchings}}.
\end{definition}


\noindent
Zwischen verschiedenen Typen des Matchings unterscheidet man auch das \tbf{perfekte Matching}.

\begin{definition}[Perfektes Matching]\label{def:perfect-matching}
Sei $\mathcal{G} = (\mathcal{V}, \mathcal{E})$ ein ungerichteter Graph. Ein \textbf{perfektes Matching} 
 ist so ein Matching, in dem alle Knoten aus $\mathcal{V}$ gematcht sind.
\end{definition}


\noindent
Um die Aufgabe in der Form zu lösen, eignet sich gut der \tbf{Satz von Hall},
der als ein Ausgangspunkt der ganzen Matching--Theorie gilt. 
Um sich dieses Satzes zu bedienen, muss man noch den Begriff der \tbf{Nachbarschaft} einführen.

\begin{definition}[Nachbarschaft]\label{def:nachbarschaft}
Sei $\mathcal{G} = (\mathcal{V}, \mathcal{E})$ ein ungerichteter Graph.
Für alle $X \subseteq \mathcal{V}$ definieren wir die \textbf{Nachbarschaft}
von $X$ als $N(X) = \{y \in \mathcal{V} \,|\, \forall x \in X : (x, y) \in \mathcal{E}\}$.
\end{definition}


\begin{satz}[Satz von Hall]
Sei $\mathcal{G} = (\mathcal{L} \cup \mathcal{R}, \mathcal{E})$ ein bipartiter, ungerichteter Graph.
Es existiert ein perfektes Matching genau dann,
wenn für alle Teilmengen $\mathcal{K} \subseteq \mathcal{L}$ gilt: $|\mathcal{K}| \leqslant |N(\mathcal{K})|$.\textnormal{\cite[S.~736, Übung]{cormen:matchings}}
\end{satz}

\begin{proof}
Auf den
\href{https://homes.cs.washington.edu/~anuprao/pubs/CSE599sExtremal/lecture6.pdf}{Beweis}
\footnote{s. etwa: Anup Rao. Lecture 6 Hall’s Theorem. October 17, 2011. University of Washington. [Zugang 21.01.2021]\\
\url{https://homes.cs.washington.edu/~anuprao/pubs/CSE599sExtremal/lecture6.pdf}}
verzichten wir.
\end{proof}


\begin{lemma}\label{lem:grad-1}
Seien $x \in A$ ein Knoten in $G$, seine Kardinalität $\Delta(x) = 1$,
und der einzelne Nachbar von $x$ sei $y \in B$.
Dann gilt: $o(x, y)$. 
\end{lemma} 
\begin{proof}
Nach dem Satz von Hall ist für $x \in A$ die Bedingung $|x| = \Delta(x) = 1 \leqslant |N(x)| = 1$ erfüllt.
Deshalb existiert ein perfektes Matching für $x \in A$ und das ist auch das einzelne mögliche Matching.\\
Nach Axiom \ref{ax:obstsorte-index} hat jede Obstsorte genau einen einzigartigen Index, also ist der Index der Obstsorte
$x$ somit gefunden.
\end{proof}



\begin{lemma}\label{lem:grad-groesser1}
Seien $x \in A$ ein Knoten in $G$ und seine Kardinalität $\Delta(x) = k > 1$.
Dann gehört $x$ zu einer Zusammenhangskomponente $C = (V_c, E_c)$, wobei
die Menge $V_c$ aus insgesamt $2k$ Knoten $x_1, ..., x_k \in A$ 
und $y_1, ..., y_k \in B$ besteht. Für die Menge $E_c$ gilt:
$E_c = \{(x_i, y_j) \,|\, $für alle $ 1 \leqslant i,j \leqslant k \}$.
$C$ ist deshalb selbt ein vollständiger, bipartiter Graph.
\end{lemma}
\begin{proof}
Der Beweis erfolgt durch Widerspruch.
\TODO{Beweis} 
\end{proof}

\TODO{Lemmata über Zuweisungen (Satz von Hall) in Komponenten:\\
eine Zuweisung ist immer möglich\\
alle auf der Wunschliste\\
mind. ein nicht auf der Wunschliste}

\begin{lemma}\label{lem:komponente-matching}
Sei $C = (L_c \cup R_c, E_c)$ eine beliebige
Zusammenhangskomponente in $G$. Dann existiert immer ein perfektes Matching zu $C$.
\end{lemma}
\begin{proof} 
Nach Lemma \ref{lem:komponente-complete} ist jede Zusammenhangskomponente in $G$ ein vollständiger,
bipartiter Graph. Nach Satz von Hall existiert ein perfektes Matching, wenn
für alle Teilmengen $K \subseteq L_c$ gilt: $|K| \leqslant |N(K)|$.
Die obere Behauptung kann für beliebig große Mächtigkeiten $|K| = k \in \mathbb{N}$
durch die vollständige Induktion bewiesen werden.\\

\noindent
\tbf{Induktionsanfang:} Für $k = 1$ hat der einzelne Knoten $x \in K \subseteq L_c$
die Kardinalität $\Delta(x) = 1$.
Deshalb gilt: $|K| = 1 \leqslant |N(x)| = 1$. Damit stimmt die Behauptung für $k = 1$ und der Induktionsanfang ist erledigt.\\

\noindent
\tbf{Induktionsschritt:} Es gelte die Aussage für ein beliebiges $k \in \mathbb{N}$, also für eine Teilemenge
$K \subseteq L_c$, die aus $k$ Knoten besteht und in der jeder Knoten $x \in K$ die Kardinalität
$\Delta(x) = k$ hat.\\ Es gelte also: $|K| \leqslant |N(K)|$.\\
Zu zeigen ist die Aussage für $k + 1$, also für eine Teilmenge $K' \subseteq L_c$ der Mächtigkeit 
$|K'| = k+1$:
\[
|K'| \leqslant |N(K')|.
\] 
Wir verifizieren:

Jeder Knoten in $C$ hat den Grad $k+1$, also: $|K'| = k + 1 \leqslant |N(K')| = (k+1)^2 = k^2 + 2k + 1$.\\
Folglich stimmt die Behauptung für $k+1$.\\

Der Induktionsschritt ist damit vollzogen und es wurde bewiesen, dass die Behauptung für beliebige
Mächtigkeit von $K$ gilt.
Dadurch wurde auch bewiesen, dass es in einer Zusammenhangskomponente in $G$
 immer ein perfektes Matching gibt.
\end{proof}


\begin{lemma}\label{}
Sei $C = (V_c, E_c)$ eine Zusammenhangskomponente in $G$.
Wenn gilt: $\forall x \in A \cap V_c : x \in W$, dann werden alle $y \in B \cap V_c$ in $W'$ hinzugefügt.
\end{lemma}
\begin{proof} 
Nach Axiom \ref{ax:obstsorte-index} besitzt jede Obstsorte genau einen einzigartigen Index.
Die Zusammenhangskomponente $C$ beschreibt nach Lemmata \ref{lem:grad-groesser1} und \ref{lem:komponente-matching},
dass jede Obstsorte $x \in A \cap V_c$ jeden Index $y \in B \cap V_c$ haben kann, weil
ein perfektes Matching stets existiert und $C$ ein vollständiger, bipartiter Graph ist.

Dadurch, dass $\forall x \in A \cap V_c : x \in W$ gilt,
ist ohne Bedeutung, welchen Index die jeweilige Obstsorte besitzt, da
 die Lösung des Problems eine Menge $W'$ mit den Indizes der Obstsorten aus $W$ sein soll.
Dadurch, dass $A \cap V_c \subseteq W$ auch gilt: $B \cap V_c \subseteq W'$. 
\end{proof}

\begin{lemma}
Sei $C = (V_c, E_c)$ eine Zusammenhangskomponente in $G$.
Wenn gilt: $\exists x \in A \cap V_c : x \notin W$, dann kann die Menge $W'$ nicht eindeutig
festgelegt werden.
\end{lemma}
\begin{proof}
Nach Axiom \ref{ax:obstsorte-index} besitzt jede Obstsorte genau einen einzigartigen Index.
Die Zusammenhangskomponente $C$ beschreibt nach Lemmata \ref{lem:grad-groesser1} und \ref{lem:komponente-matching},
dass jede Obstsorte $x \in A \cap V_c$ jeden Index $y \in B \cap V_c$ haben kann, weil
ein perfektes Matching stets existiert und $C$ ein vollständiger, bipartiter Graph ist.

Angenommen, $\exists p \in A \cap V_c \land p \notin W$. Dann ist es unmöglich, festzustellen,
welcher Index aus $B \cap V_c$ der Obstsorte $p$ gehört.
Also ist es auch unmöglich, festzustellen, welche Indizes in $W'$ hinzugefügt werden sollen.
Deshalb, unabhängig von allen anderen Zusammenhangskomponenten des Graphen $G$,
ist unmöglich, eine eindeutige Menge der Indizes der gewünschten Obstsorten festzulegen.
Dadurch gibt es keine eindeutige Lösung zu diesem Problem für diese Eingabe.
\end{proof}


\begin{lemma}
\TODO{umschreiben}
Wenn alle Zusammenhangskomponten, die die Wünsche aus $W$ beinhalten, nur aus den Wüschen
aus $W$ bestehen, kann $W'$ eindeutig und vollständig festgelegt werden.
\end{lemma}
\begin{proof} 
\end{proof}


\begin{figure}[h]
\centering
\begin{subfigure}[c]{.49\textwidth}
\centering
\begin{tikzpicture}
    \node[vertex] (A) {$A$};
    \node[vertex] (B) [below = 0.4cm of A] {$B$};
    \node[vertex] (Br) [below = 0.4cm of B] {$Br$};
    \node[vertex] (E) [below = 0.4cm of Br] {$E$};
    \node[vertex] (P) [below = 0.4cm of E] {$P$};
    \node[vertex] (W) [below = 0.4cm of P] {$W$};
    \node[vertex] (1) [right = 1.5cm of A] {$1$};
    \node[vertex] (2) [right = 1.5cm of B] {$2$};
    \node[vertex] (3) [right = 1.5cm of Br] {$3$};
    \node[vertex] (4) [right = 1.5cm of E] {$4$};
    \node[vertex] (5) [right = 1.5cm of P] {$5$};
    \node[vertex] (6) [right = 1.5cm of W] {$6$};

    \path[draw,thick]
    (A) edge node {} (1)
    (A) edge node {} (4)
    (B) edge node {} (5)
    (Br) edge node {} (1)
    (Br) edge node {} (4)
    (E) edge node {} (2)
    (P) edge node {} (6)
    (W) edge node {} (3);
\end{tikzpicture}

\caption{Der Graph nach der Analyse der allen Spießkombinationen}
\label{fig:graph-after-infos}
\end{subfigure}
\begin{subfigure}[c]{.49\textwidth}
\centering
\begin{tikzpicture}
    \node[vertex] (A) {$A$};
    \node[vertex] (Br) [below = 0.4cm of A] {$Br$};
    \node[vertex] (1) [right = 1.5cm of A] {$1$};
    \node[vertex] (4) [right = 1.5cm of Br] {$4$};

    \path[draw,thick]
    (A) edge node {} (1)
    (A) edge node {} (4)
    (Br) edge node {} (1)
    (Br) edge node {} (4);
\end{tikzpicture}

\caption{Die ürbige Zusammenhangskomponente}
\label{fig:component-left}
\end{subfigure}
\end{figure}

\subsection{Prüfung auf Korrektheit der Eingabe}\label{sec:korrektheit-eingabe}
\subsection{Laufzeit}\label{sec:laufzeit}

\section{Umsetzung}\label{sec:umsetzung}
\subsection{Klasse \ttt{Solver}}
\subsection{Klasse \ttt{Graph}}

\section{Beispiele}

\subsection{Beispiel 0 (Aufgabenstellung --- Teil a)}\label{example:0}
Textdatei: \texttt{spiesse0.txt}\\
\noindent
\framebox{Apfel, Brombeere, Weintraube}\\

\noindent
\framebox{1, 3, 4}\\

Auf Papier kann man das Beispiel auf folgende Weise lösen.
Am Anfang weiß man nichts über die Obstsorten in $A$ und die Indizes in $B$.
Wir zeichnen deshalb einen vollständigen, bipartiten Graphen $G$ (Abb. \ref{fig:example0-1}).
Jede Kante steht für eine mögliche Zuweisung eines Index einer Obstsorte.\\

Wir analysieren die 1. Spießkombination:
\minibox[frame]{$F_1 = \{\text{Apfel, Banane, Brombeere}\}, Z_1 = \{1, 4, 5\}$}.\\
Wir stellen fest, dass Apfel, Banane und Brombeere jeweils keine Indizes 2, 3, 6 haben können,
weil jeder dieser Obstsorten ein Index aus der Menge $Z_1$ zugewiesen wird.
Gleichzeitig merken wir, dass Erdbeere, Pflaume und Weintraube jeweils keinen der Indizes
1, 4, 5 besitzen können, weil diese ausschließlich den Obstsorten aus der Menge $F_1$ zugewiesen werden.
Somit stellen wir fest, dass die Anzahl der möglichen Zuweisungen schrumpft.
Deshalb dürfen wir alle Kanten in $G$ zwischen allen $x \in F_1$ und allen $y \in B \setminus Z_1$,
sowie zwischen allen $p \in A \setminus F_1$ und allen $q \in Z_1$ entfernen (Abb. \ref{fig:example0-2}).\\

Wir analysieren die 2. Spießkombination:
\minibox[frame]{$F_2 = \{\text{Banane, Pflaume, Weintraube}\}, Z_2 = \{3, 5, 6\}$}.\\
Wir stellen fest, dass Banane, Pflaume, Weintraube jeweils keinen der Indizes 1, 2, 4 haben können,
weil jeder dieser Obstsorten ein Index aus der Menge $Z_2$ zugewiesen wird. 
Wir entfernen alle Kanten in $G$ zwischen allen $x \in F_2$ und allen $y \in B \setminus Z_2$.
Unter anderem wurden die Kanten (Banane, 1) und\\ (Banane, 4) entfernt.
Wir stellen fest, dass die Kardinlität des Knotens „Banane“ 1 beträgt --- er ist nur mit dem Knoten 5 verbunden.
Ebenfalls ist der Knoten 5 nur mit dem Knoten „Banane“ verbunden.
Dies bedeutet, dass es nur eine einzige Möglichkeit gibt, diesen Knoten mit einem Index zu verbinden.
Somit wurde der Index von Banane gefunden. Wir kennen schon eine Obstsorte: $o(\text{Banane}, 5)$.\\
Wir stellen auch fest, dass Apfel, Brombeere und Erdbeere jeweils keinen der Indizes 3, 5, 6 haben können,
da sie nur den Obstsorten aus $F_2$ zugewiesen werden dürfen.
Insbesondere wissen wir schon, dass 5 mit Banane verbunden wird.
Deshalb dürfen wir alle Kanten in $G$ zwischen allen $p \in A \setminus F_2$ und allen $q \in Z_2$ entfernen.
Wir bemerken, dass die Kardinlität des Knotens „Erdbeere“ nun 1 beträgt, der nur mit dem Knoten 2 verbunden ist.
Ebenfalls ist der Knoten 2 mit keinem anderen verbunden. So steht fest:\\ $o$(Erdbeere, 2).
Auf der Abbildung \ref{fig:example0-3} wird der Graph $G$ nach der Verarbeitung der 2. Spießkombination
dargestellt. Die fetten Kanten zeigen an, dass die zwei Endknoten bereits verbunden sind, also, dass 
diesen Obstsorten ihre Indizes zugewiesen wurden.\\

Wir analysieren die 3. Spießkombination:
\minibox[frame]{$F_3 = \{\text{Apfel, Brombeere, Erdbeere}\}, Z_3 = \{1, 2, 4\}$}.\\
An dieser Stelle wurde die Zuweisung für Erdbeere bereits gefunden.
Die Knoten „Apfel“ und „Brombeere“ besitzen keine Kanten mehr als die Kanten, die sie jeweils mit 1 und 4 verbinden. 
Ebenfalls wurde der Index 5 Banane zugewiesen.
Die Knoten „Pflaume“ und „Weintraube“ sind jeweils nur mit 3 und 6 verbunden.
So können wir keine der übrigen Kanten zwischen irgendwelchen zwei Knoten in $G$ entfernen.\\

Wir analysieren die 4. Spießkombination:
\minibox[frame]{$F_4 = \{\text{Erdbeere, Pflaume}\}, Z_4 = \{2, 6\}$}.\\
In der Menge $F_4$ tritt Erdbeere auf, deren Index bereits gefunden wurde.
Somit wissen wir, dass der Index von Pflaume 6 sein muss. 
So entdecken wir eine neue Zuweisung: $o$(Pflaume, 6).
Wir können deshalb alle übrigen Kanten zwischen Pflaume und allen anderen Knoten entfernen.
So bleibt der Knoten „Weintraube“ mit nur einer Kante übrig.
Der einzelne Nachbar von diesem Knoten ist 3. So entdecken wir wieder eine neue Zuweisung: $o$(Weintraube, 3).\\
Wir stellen auch fest, dass keine Kanten mehr entfernt werden können.
Es bleibt immer noch ein Paar von Indizes und ein Paar von Obstsorten ohne eindeutige Zuweisung:
\{Apfel, Brombeere\} und $\{1, 4\}$.\\

An dieser Stelle schauen wir die Wunschliste an: \framebox{Apfel, Brombeere, Weintraube}.\\
Der Index von Weintraube ist erfolgreich gefunden, aber die Indizes der übrigen Obstsorten nicht.
Allerdings soll die Lösung der Aufgabe eine Menge an Indizes der gewünschten Obstsorten sein ---
es müssen keine konkreten Zuweisungen ausgegeben werden. 
Dies wurde erfolgreich gefunden, da die Indizes 1 und 4 nur Apfel oder Brombeere gehören können, weil
keine anderen Kanten aus den Knoten 1 und 4 führen.
Auf der Abbildung \ref{fig:example0-4} wurden alle gefundenen Zuweisungen durch fette Kanten dargestellt
und alle Wünsche mit ihren Indizes wurden entsprechend \colorbox{black!30!green}{\textcolor{white}{grün}} und \colorbox{black!5!blue}{\textcolor{white}{blau}} markiert.

\begin{figure}[H]
\centering
\begin{adjustbox}{minipage=\linewidth,scale=0.85}
\begin{subfigure}[t]{.24\textwidth}
\centering
\begin{tikzpicture}
    \node[vertex] (1) {$A$};
    \node[vertex] (2) [below = 0.4cm of 1] {$B$};
    \node[vertex] (3) [below = 0.4cm of 2] {$Br$};
    \node[vertex] (4) [below = 0.4cm of 3] {$E$};
    \node[vertex] (5) [below = 0.4cm of 4] {$P$};
    \node[vertex] (6) [below = 0.4cm of 5] {$W$};
    \node[vertex] (7) [right = 1.5cm of 1] {$1$};
    \node[vertex] (8) [right = 1.5cm of 2] {$2$};
    \node[vertex] (9) [right = 1.5cm of 3] {$3$};
    \node[vertex] (10) [right = 1.5cm of 4] {$4$};
    \node[vertex] (11) [right = 1.5cm of 5] {$5$};
    \node[vertex] (12) [right = 1.5cm of 6] {$6$};

    \path[draw,thick]
    (1) edge node {} (7)
    (1) edge node {} (8)
    (1) edge node {} (9)
    (1) edge node {} (10)
    (1) edge node {} (11)
    (1) edge node {} (12)
    (2) edge node {} (7)
    (2) edge node {} (8)
    (2) edge node {} (9)
    (2) edge node {} (10)
    (2) edge node {} (11)
    (2) edge node {} (12)
    (3) edge node {} (7)
    (3) edge node {} (8)
    (3) edge node {} (9)
    (3) edge node {} (10)
    (3) edge node {} (11)
    (3) edge node {} (12)
    (4) edge node {} (7)
    (4) edge node {} (8)
    (4) edge node {} (9)
    (4) edge node {} (10)
    (4) edge node {} (11)
    (4) edge node {} (12)
    (5) edge node {} (7)
    (5) edge node {} (8)
    (5) edge node {} (9)
    (5) edge node {} (10)
    (5) edge node {} (11)
    (5) edge node {} (12)
    (6) edge node {} (7)
    (6) edge node {} (8)
    (6) edge node {} (9)
    (6) edge node {} (10)
    (6) edge node {} (11)
    (6) edge node {} (12);
\end{tikzpicture}

\caption{}
\label{fig:example0-1}
\end{subfigure}\hfill
\begin{subfigure}[t]{.24\textwidth}
\centering
\begin{tikzpicture}
    \node[vertex] (A) {$A$};
    \node[vertex] (B) [below = 0.4cm of A] {$B$};
    \node[vertex] (Br) [below = 0.4cm of B] {$Br$};
    \node[vertex] (E) [below = 0.4cm of Br] {$E$};
    \node[vertex] (P) [below = 0.4cm of E] {$P$};
    \node[vertex] (W) [below = 0.4cm of P] {$W$};
    \node[vertex] (1) [right = 1.5cm of A] {$1$};
    \node[vertex] (2) [right = 1.5cm of B] {$2$};
    \node[vertex] (3) [right = 1.5cm of Br] {$3$};
    \node[vertex] (4) [right = 1.5cm of E] {$4$};
    \node[vertex] (5) [right = 1.5cm of P] {$5$};
    \node[vertex] (6) [right = 1.5cm of W] {$6$};

    \path[draw,thick]
    (A) edge node {} (1)
    (A) edge node {} (4)
    (A) edge node {} (5)
    (B) edge node {} (1)
    (B) edge node {} (4)
    (B) edge node {} (5)
    (Br) edge node {} (1)
    (Br) edge node {} (4)
    (Br) edge node {} (5)
    (E) edge node {} (2)
    (E) edge node {} (3)
    (E) edge node {} (6)
    (P) edge node {} (2)
    (P) edge node {} (3)
    (P) edge node {} (6)
    (W) edge node {} (2)
    (W) edge node {} (3)
    (W) edge node {} (6);
\end{tikzpicture}
\caption{}
\label{fig:example0-2}
\end{subfigure}
\begin{subfigure}[t]{.24\textwidth}
\centering
\input{./tex/tikz/2.spiess.tex}
\caption{}
\label{fig:example0-3}
\end{subfigure}\hfill
\begin{subfigure}[t]{.24\textwidth}
\centering
\begin{tikzpicture}
    \node[vertex,text=white,fill=black!30!green] (A) {$A$};
    \node[vertex] (B) [below = 0.4cm of A] {$B$};
    \node[vertex,text=white,fill=black!30!green] (Br) [below = 0.4cm of B] {$Br$};
    \node[vertex] (E) [below = 0.4cm of Br] {$E$};
    \node[vertex] (P) [below = 0.4cm of E] {$P$};
    \node[vertex,text=white,fill=black!30!green] (W) [below = 0.4cm of P] {$W$};
    \node[vertex,text=white,fill=black!5!blue] (1) [right = 1.5cm of A] {$1$};
    \node[vertex] (2) [right = 1.5cm of B] {$2$};
    \node[vertex,text=white,fill=black!5!blue] (3) [right = 1.5cm of Br] {$3$};
    \node[vertex,text=white,fill=black!5!blue] (4) [right = 1.5cm of E] {$4$};
    \node[vertex] (5) [right = 1.5cm of P] {$5$};
    \node[vertex] (6) [right = 1.5cm of W] {$6$};

    \path[draw,thick]
    (A) edge node {} (1)
    (A) edge node {} (4)
    (B) edge[line width=2pt] node {} (5)
    (Br) edge node {} (1)
    (Br) edge node {} (4)
    (E) edge[line width=2pt] node {} (2)
    (P) edge[line width=2pt] node {} (6)
    (W) edge[line width=2pt] node {} (3);
\end{tikzpicture}
\caption{}
\label{fig:example0-4}
\end{subfigure}
\end{adjustbox}
\caption{}
\label{fig:example0}
\end{figure}



\subsection{Beispiel 1 (BWINF)}\label{example:1}
Textdatei: \texttt{spiesse1.txt}\\

\noindent
Wünsche: \framebox{Clementine, Erdbeere, Grapefruit, Himbeere, Johannisbeere}\\

\noindent
\framebox{1, 2, 4, 5, 7}

\subsection{Beispiel 2 (BWINF)}\label{example:2}
Textdatei: \texttt{spiesse2.txt}
\vspace{0.25cm}

\noindent
Wünsche: \framebox{Apfel, Banane, Clementine, Himbeere, Kiwi, Litschi}
\vspace{0.25cm}

\noindent
\framebox{1, 5, 6, 7, 10, 11}

\subsection{Beispiel 3 (BWINF)}\label{example:3}
Textdatei: \texttt{spiesse3.txt}\\

\noindent
Wünsche: \framebox{Clementine, Erdbeere, Feige, Himbeere, Ingwer, Kiwi, Litschi}\\

\noindent
\framebox{Dieses Beispiel ist unlösbar.}
\begin{verbatim}
Für die folgenden Obstsorten konnte keine eindeutige Zuweisung gefunden werden.
Komponente: Grapefruit Litschi 
	--> Nicht auf der Wunschliste: Grapefruit 
\end{verbatim}

\subsection{Beispiel 4 (BWINF)}\label{example:4}
Textdatei: \texttt{spiesse4.txt}
\vspace{0.25cm}

\noindent
Wünsche: \framebox{Apfel, Feige, Grapefruit, Ingwer, Kiwi, Nektarine, Orange, Pflaume}
\vspace{0.25cm}

\noindent
\framebox{2, 6, 7, 8, 9, 12, 13, 14}

\newpage
\subsection{Beispiel 5 (BWINF)}\label{example:5}
Textdatei: \texttt{spiesse5.txt}
\vspace{0.25cm}

\noindent
Wünsche: \minibox[frame]{Apfel, Banane, Clementine, Dattel, Grapefruit, Himbeere, Mango,\\ Nektarine, Orange, Pflaume, Quitte, Sauerkirsche, Tamarinde}
\vspace{0.25cm}

\noindent
\framebox{1, 2, 3, 4, 5, 6, 9, 10, 12, 14, 16, 19, 20}

\subsection{Beispiel 6 (BWINF)}\label{example:6}
Textdatei: \texttt{spiesse6.txt}\\

\noindent
\framebox{Clementine, Erdbeere, Himbeere, Orange, Quitte, Rosine, Ugli, Vogelbeere}\\

\noindent
\framebox{4, 6, 7, 10, 11, 15, 18, 20}

\subsection{Beispiel 7 (BWINF)}\label{example:7}
Textdatei: \texttt{spiesse7.txt}\vspace{0.25cm}

\noindent
Wünsche: \minibox[frame]{Apfel, Clementine, Dattel, Grapefruit, Mango, Sauerkirsche, Tamarinde, Ugli,\\ Vogelbeere, Xenia, Yuzu, Zitrone}
\vspace{0.25cm}

\noindent
\framebox{Dieses Beispiel ist unlösbar.}
\begin{verbatim}
Für die folgenden Obstsorten konnte keine eindeutige Zuweisung gefunden werden.
Komponente: Apfel Grapefruit Litschi Xenia 
	--> Nicht auf der Wunschliste: Litschi 
Komponente: Banane Ugli 
	--> Nicht auf der Wunschliste: Banane 
\end{verbatim}


\subsection{Beispiel 8}\label{example:8}
Textdatei: \texttt{spiesse8.txt}\vspace{10pt}\\
\texttt{
\noindent
4\\
Dattel Apfel Banane\\
3\\
1 2 3\\
Apfel Banane Dattel\\
1 2\\
Apfel Banane\\
1 3\\
Clementine Dattel}\\

\noindent
Wünsche: \framebox{Dattel, Apfel, Banane}\\

\noindent
\framebox{Error: Es gibt Fehler in der Eingabedatei.}\\

Die erste Spießkombination legt fest, dass Apfel, Banane und Dattel einen der folgenden Indizes
besitzen: $\{1, 2, 3\}$.  Das bedeutet auch, dass Clementine --- die einzelne übrige Obstsorte ---
den Index 4 besitzt. Jedoch widerspricht dieser Zuweisung die dritte Spießkombination,
laut der Clementine den Index 1 oder 3 hat. 

\newpage
\subsection{Beispiel 9}\label{example:9}
Textdatei: \texttt{spiesse9.txt}\\
Besonderheit: Ein Beispiel für den Fall \ref{probleme2} $\rightarrow$ \ref{fall2}, s. Teil \ref{sec:korrektheit-eingabe}.
\begin{verbatim}
5
Apfel Erdbeere Banane
2
2 3
Banane Clementine
4 5
Banane Dattel
\end{verbatim}

\noindent
Wünsche: \framebox{Apfel, Erdbeere, Banane}
\vspace{0.25cm}

\noindent
\framebox{Error: Es gibt Fehler in der Eingabedatei.}
\vspace{0.25cm}

Die erste Spießkombination legt fest, dass nur Banane und Clementine die Indizes 2 und 3 besitzen dürfen.
Allerdings sollen die Indizes 4 und 5 nach der zweiten Spießkombination den Obstsorten Banane und Dattel 
gehören. Die Obstsorte Banane darf keine zwei unterschiedliche Indizes besitzen.
Deshalb kommt es zu einem Widerspruch --- das Axiom \ref{ax:obstsorte-index} ist verletzt.
\subsection{Beispiel 10}\label{example:10}
Textdatei: \texttt{spiesse10.txt}\\
Besonderheit: ein worst-case, in dem alle $m = 28$ Spießkombinationen alle $n = 26$ Obstsorten beinhalten. Die Wunschliste beinhaltet alle $n$ Obstsorten.\\

\noindent
Wünsche: \minibox[frame]{Apfel, Banane, Clementine, Dattel, Erdbeere, Feige, Grapefruit, Himbeere, Ingwer,\\
Johannisbeere, Kiwi, Litschi, Mango, Nektarine, Orange, Pflaume, Quitte, Rosine,\\
Sauerkirsche, Tamarinde, Ugli, Vogelbeere, Weintraube, Xenia, Yuzu, Zitrone}\\
\vspace{7pt}

\noindent
\minibox[frame]{1, 2, 3, 4, 5, 6, 7, 8, 9, 10, 11, 12, 13, 14, 15, 16, 17, 18, 19, 20, 21, 22, 23, 24, 25, 26}

\subsection{Beispiel 11}\label{example:11}
Textdatei: \texttt{spiesse11.txt}\\
Besonderheit: ein worst-case, in dem alle $m = 28$ Spießkombinationen alle $n = 26$ Obstsorten beinhalten. Die Wunschliste beinhaltet 24 Obstsorten.
\vspace{0.25cm}

\noindent
Wünsche: \minibox[frame]{Apfel, Banane, Clementine, Dattel, Feige, Grapefruit, Himbeere, Ingwer,\\
Johannisbeere, Kiwi, Litschi, Mango, Nektarine, Orange, Pflaume, Rosine,\\
Sauerkirsche, Tamarinde, Ugli, Vogelbeere, Weintraube, Xenia, Yuzu, Zitrone}\\
\vspace{0.25cm}

\noindent
\framebox{Dieses Beispiel ist unlösbar.}
\begin{verbatim}
Für die folgenden Obstsorten konnte keine eindeutige Zuweisung gefunden werden.
Komponente: Apfel Banane Clementine Dattel Erdbeere Feige Grapefruit Himbeere Ingwer
Johannisbeere Kiwi Litschi Mango Nektarine Orange Pflaume Quitte Rosine Sauerkirsche
Tamarinde Ugli Vogelbeere Weintraube Xenia Yuzu Zitrone 
	--> Nicht auf der Wunschliste: Erdbeere Quitte
\end{verbatim}

\newpage
\subsection{Beispiel 12}\label{example:12}
Textdatei: \texttt{spiesse12.txt}\\
Besonderheit: Es entstehen $\frac{n}{2}$ Zusammenhangskomponenten in $G$. Keiner Obstsorte kann ein Index eindeutig zugeordnet werden.
\begin{verbatim}
8
Feige Dattel Clementine Grapefruit
4
5 1 3 2 8 6
Dattel Banane Feige Erdbeere Clementine Himbeere
1 5 8 6
Himbeere Dattel Feige Banane
3 2 7 4
Apfel Clementine Erdbeere Grapefruit
8 7 1 4
Banane Apfel Grapefruit Dattel
\end{verbatim}

\noindent
Wünsche: \framebox{Clementine, Dattel, Feige, Grapefruit}\\

\noindent
\framebox{Dieses Beispiel ist unlösbar.}
\begin{verbatim}
Für die folgenden Obstsorten konnte keine eideutige Zuweisung gefunden werden.
Komponente: Clementine Erdbeere 
	--> Nicht auf der Wunschliste: Erdbeere 
Komponente: Banane Dattel 
	--> Nicht auf der Wunschliste: Banane 
Komponente: Feige Himbeere 
	--> Nicht auf der Wunschliste: Himbeere 
Komponente: Apfel Grapefruit 
	--> Nicht auf der Wunschliste: Apfel
\end{verbatim}
\subsection{Beispiel 13}\label{example:13}
Textdatei: \texttt{spiesse13.txt}\\
Besonderheit: Es gibt $m = n$ Spießkombinationen. Die Mächtigkeiten der Mengen dieser Spießkombinationen entsprechen:
$n, n-1, ..., 1$. Die Spießkombinationen werden aufeinander aufgebaut.
\begin{verbatim}
8
Apfel Himbeere Grapefruit Banane
8
5 1 3 2 4 8 6 7 
Dattel Grapefruit Banane Feige Erdbeere Clementine Apfel Himbeere
1 5 4 6 8 2 7
Himbeere Dattel Feige Banane Apfel Gp0rapefruit Clementine
4 8 2 5 1 7 
Apfel Dattel Clementine Himbeere Banane Grapefruit
8 7 2 4 1
Clementine Banane Apfel Grapefruit Dattel
1 4 2 7
Grapefruit Clementine Dattel Apfel
7 2 1
Dattel Grapefruit Clementine
2 7
Clementine Grapefruit
2
Clementine
\end{verbatim}

\noindent
Wünsche: \framebox{Apfel, Himbeere, Grapefruit, Banane}
\vspace{0.25cm}

\noindent
\framebox{4, 5, 7, 8}
\subsection{Beispiel 14}\label{example:14}
Textdatei: \texttt{spiesse14.txt}\\
Besonderheit: Ein Beispiel für den Fall \ref{probleme1} $\rightarrow$ \ref{fall2}, s. Teil \ref{sec:korrektheit-eingabe}.
\begin{verbatim}
6
Apfel Clementine Feige
2
1 4 5
Apfel Banane Dattel
4 5 6
Dattel Erdbeere Feige
\end{verbatim}

\noindent
Wünsche: \framebox{Apfel, Clementine, Feige}
\vspace{0.25cm}

\noindent
\framebox{Error: Es gibt Fehler in der Eingabedatei.}
\vspace{0.25cm}

Die erste Spießkombination legt fest, dass die Obstsorten Apfel, Banane, Dattel
jeweils einen der Indizes $\{1, 4, 5\}$ haben können. Dennoch die zweite Spießkombination 
legt im Widerspruch zur ersten fest, dass die Indizes $\{4, 5, 6\}$ jeweils einer der
Obstsorten Dattel, Erdbeere, Feige gehören können. In diesem Fall decken sich 4 und Dattel
in beiden Spießkombinationen, aber 5 kann keiner Obstsorte zugeordnet werden.
\subsection{Beispiel 15}\label{example:15}
Textdatei: \texttt{spiesse15.txt}\\
Besonderheit: Die obere Schranke an Obstsorten ist sehr groß im Vergleich zu der Anzahl der 
in den Spießkombinationen und in der Wunschliste erwähnten Obstsorten. Dieses Beispiel betont, dass
mein Programm problemlos so eine große obere Schranke behandeln kann. Außerdem zeigt sich hier der
Vorteil der Verwendung von Bitmasken im Vergleich zu einer „Standard“--Adjazenzmatrix.
\begin{verbatim}
26
Apfel Dattel Feige
3
1 2 4
Apfel Banane Dattel
2 3 6
Dattel Erdbeere Feige
1 6
Banane Erdbeere
\end{verbatim}

\noindent
Wünsche: \framebox{Apfel, Dattel, Feige}\\

\noindent
\framebox{2, 3, 4}

\section{Quellcode}
\lstinputlisting[language=C++]{./tex/spiesse.m}

\end{document}